\documentclass{article}

\usepackage[francais]{babel}
\usepackage[T1]{fontenc}
\usepackage[utf8]{inputenc}
\usepackage{graphicx}
\usepackage{float}
\usepackage{setspace}

\usepackage{xcolor}
\usepackage[font=small,margin=1mm]{caption}
\usepackage[toc,page]{appendix}
\usepackage{tocloft}
\addtocontents{toc}{\cftpagenumbersoff{part}}
\addtocontents{toc}{\cftpagenumbersoff{subsection}}
\renewcommand\cftsecfont{\bfseries}
\renewcommand{\cftaftertoctitle}{\vspace{5pt} \hrule}
\usepackage[nottoc,notindex]{tocbibind}
\usepackage[french]{nomencl}
\makenomenclature
\renewcommand{\nomname}{Glossaire}
\usepackage[pdftex,backref=false,pdfborder={0 0 0}]{hyperref}
\usepackage[nonumberlist]{glossaries}
\makeglossaries
\usepackage{textcomp}
\usepackage{enumitem}
\renewcommand{\theenumi}{{\scriptsize \textbf{\emph{\thesection.\arabic{enumi}}}}}
%pas d'indice dans la bibliographie
\makeatletter
\renewcommand{\@biblabel}[1]{}
\makeatother
\pagecolor{yellow!10}
\title{%\textsf{\Huge \textbf{Open Composition}}
}
\author{\textsf{\textbf{Yann Ics}}}
\date{}
%\setlength{\oddsidemargin}{10pt} % Marge gauche sur pages impaires
%\setlength{\evensidemargin}{10pt} % Marge gauche sur pages paires
%\setlength{\marginparwidth}{84pt} % Largeur de note dans la marge
%\setlength{\textwidth}{381pt} % Largeur de la zone de texte (17cm)
\setcounter{secnumdepth}{0} 
\usepackage[modulo]{lineno}
\usepackage{enotez}
% https://tex.stackexchange.com/questions/202805/refer-to-an-endnote-with-the-enotez-package-points-to-the-wrong-note
\renewcommand{\footnote}{\endnote}
\newcommand{\footref}[1]{%
  \enotezwritemark{\enmarkstyle\ref{#1}}%
}
%\let\endnote=\endnote
\newcommand{\eo}{\textsl{etou}}
\begin{document}

%\begin{minipage}{.033\textwidth}\includegraphics[width=\linewidth]{1123}\end{minipage}

\makeatletter
  \begin{titlepage}
  \centering
  {\Large \@author} \\
  \rule{\linewidth}{.5pt}
  \vspace{4cm}
  
  {\LARGE \textbf{\@title}} \\
   \vspace{6 mm}
   {\color{brown} {\Large \textbf{\texttt{\, ... à la recherche d'un nouveau}}}}\\
   \vspace{2mm}
   {\color{brown}{\Large \textbf{\texttt{\qquad paradigme de musique vivante.}}}}\\
  \vfill
  \rule{\linewidth}{.5pt}
%  in search of a new paradigm of living music.
     {\Large \textsf{\textsl{}}}\\ 
          {\Large \textsf{\textsl{}}}\\ 

      {\Large \textsf{Essai}}\\

 \end{titlepage}
\makeatother

%\tableofcontents

\pagecolor{white}

%% blank page
%\newpage
%\thispagestyle{empty}
%\mbox{}
%\newpage

%\vspace*{,5cm}

\renewcommand\thelinenumber{\color{purple}\arabic{linenumber}}
\linenumbers
  
\section{Avertissement}

Le but de cet essai est d'exposer et de contextualiser une pensée simple dont la mise à l'écrit s'est révélé d'une complexité déroutante. Aussi, je revendique un style d'écriture qui se veut concis et aussi précis que possible, à l'encontre d'une quelconque vulgarisation,  et qui requiert parfois l'utilisation d'une terminologie spécifique. En effet, il existe des mots, qui synthétisent, subsument, condensent et cristallisent des concepts, et qui révèlent un monde aussi complexe que fascinant, pour lesquels j'estime opportun d'en illustrer la pertinence. J'avoue prendre le risque de ne pas être lu ou compris, mais à l'heure où l'accessibilité du savoir est à la portée de tous -- ou presque --, j'invite le lecteur à rechercher, disséquer, au delà de mon propre propos, le sens de certain concepts, mots, peu ou pas développer à dessein, et de ne pas hésiter à déconstruire mon modeste édifice, usant pour les plus aventureux de la grammatologie déridienne.

\bigskip

\section{...}
Suite à une longue reflexion au sujet d'accéder, de pratiquer, de créer et de partager le phénomène sonore en tant qu'objet musical, et à une expérience de vie dans laquelle je suis toujours en quête par ailleurs d'un ailleurs, 
% et à cause d'une tendance misanthropique  
 je me propose d'exposer sous la forme d'un essai ma vision et mes espérances sur notre propre existence et notre propre ontologie, et ce à travers le prisme de l'expérience acoustique et musical. En effet, si il est devenu évident que la musique soit un produit de consommation courante, avec ceux qui la font et ceux qui la consomme, au regard de ceux qui la vive au sein d'une recherche introspective ou d'un rituel,
 %d'une communion propre à un environnement propice
il m'est possible de ressentir un vide entre le fait de ne pas souscrire à la logique consumériste et de ne pas avoir d'ascendant sur l'experience musicale en tant que pratique collective.
% -- certes, excepté les quelques élus à la gloire de l'excellence pour lesquels on se doit de s'incliner. 
Serait-ce l'horizon d'un nouveau paradigme, \textit{à la limite du pays fertile}. Sans aucun doute, et l'objet de cet essai est d'en identifier le substrat et d'en décrire les modalités praxéologiques\endnote{ De \textit{Praxis}, qui désigne la pratique d'une activité visant au développement du sujet.}\label{efn:pr}. 

\bigskip
Ceci est purement théorique, car toutes mes tentatives en ce sens ont été des échecs, et se sont limitées à une espèce de bruit ressenti comme incontrôlable et incontrôlé qui n'a pas enthousiasmé les participants. Probablement dû à un manque de pédagogie de ma part et aussi à une absence de linéaments -- quoique cela aurait du être définit par le groupe --, en bref une préparation inadéquate\endnote{L'idée première était d'initier un processus par la prise de décision par consensus, lequel devait être introduit conceptuellement, demandant un `effort intellectuel' que l'audience n'était pas prêt, ni préparé, à exercer.\\ \indent La prise de décision par consensus est un processus impliquant tous les participants afin de résoudre une problématique pour le bénéfice de tous. Cela implique l'analyse du projet pour en comprendre les tenants et les aboutissants, afin d’être discuter jusqu'à ce que chaque personne soit d'accord. Si une personne désapprouve la résolution, le groupe se doit de proposer une alternative jusqu'à ce que tout le monde soit d'accord. Bien sûr, cela peut impliquer certaines règles en cas de désaccord afin de maintenir la cohésion du groupe.}. Cependant, ce `nouveau paradigme' doit se construire selon un mouvement consensuel des participants qui ne souffre ni leader, ni  maitre, et doit émerger d'une édification systémique. %(1) du moins de façon permanente. De plus, comme toutes \textit{praxis}, celle-ci doit s'inscrire dans le temps 
Bien sûr, cela doit s'inscrire dans le temps -- \textsc{je dirais pour ma part au moins une session par semaine sur une période d'au moins 6 mois pour en apprécier les perspectives} -- et demande une assiduité et un volontarisme (bref une véritable \textit{praxis}\footref{efn:pr}), qui dans le cadre proposé -- c'est à dire libre\endnote{Un des points important dans le déroulement des activités était la possibilité d’un accès libre, afin de créer son propre calendrier, en fonction bien sûr des ressources et des disponibilités du lieux et des personnes en termes de possibilités pratiques telles que l’espace, les horaires et la nature du processus en cours. Dans tout les cas de figures, c’est le partage d’expériences et de compétences sans apriori et le respect d'un calendrier convenu qui devait constituer le dessein de cet `atelier'.} -- n'a tout simplement pas trouvé, ne serait-ce que par curiosité, son `public'. Les musiciens sortent difficilement du cadre tonal, de leurs habitudes, et les autres se déclarent ineptes à toutes pratiques musicales. En bref, les uns produisent, les autres consomment, point. 

% 
%This is purely theoretical, because all my attempts in this direction have been unsuccessful, and have been limited to a kind of free jazz jamming session, which did not thrill the participants. Probably due to a lack of pedagogy on my part and also to a real lack of charisma. In addition, this musical practice must take place over time -- I would say at least over a period of 6 months with one session per week to appreciate the perspectives -- and requires diligence and voluntarism, which in the proposed framework -- that is to say free -- has simply not found, if only out of curiosity, its `audience'. The musicians have difficulty getting out of the tonal framework and the others declare themselves inept at all musical practices. In short, some produce, others consume, period.

Il faut sans doute être fou pour vouloir changer cela. Peut être ne le suis-je pas assez moi-même, et que cet écrit est finalement destiné à en inspirer certains afin de permettre à d'autres -- y compris moi-même -- d'en être les acteurs et les témoins ...
%You probably have to be crazy to want to change that. Maybe I am not enough myself, and that this writing is ultimately intended to inspire some ...

 
%\section{Propos}
\section{\textit{Purpose}}
\textit{It appears that our globalist civilisation has failed, and finally  be only a consumerist purpose to serve the} \textsl{Brave New World}\endnote{Le \textit{Meilleur des mondes} est une projection à terme -- d’aucuns parleront de société parfaite -- basé sur le conditionnement des castes et la compartimentation du savoir et des compétences.%\\   \indent L'aspect politique -- en tant qu'état d'une possible transition -- ne sera pas trop développé au cours de cet essai, mais il est parfois nécessaire de préciser ce qui se cache derrière ce besoin de changement, ce qu'il faut changer et peut-être comment et pourquoi. Ces digressions se feront au besoin ici ou là en note.% -- preceded by the symbol  
%\\ \indent Présentement, il importe de souligner la dérive totalitaire de la spécialisation et de la professionnalisation dans nos sociétés. Le premier réduit les connaissances à une ou quelques thématiques contre toute observation holistique, et par induction le second pervertit les relations humaines car le professionnel agit de manière routinière, standardisée, en tant que spécialiste. Cela signifie une dépendance excessive du système et de ses institutions, notamment par le cloisonnement des responsabilités.
\\ \indent \textit{The political aspect -- as a possible transitional state -- won't be developed too much during this writing, but it needs sometimes to precise what behind the need for change and what to change and maybe how and why. These digressions will be done here and there, when required as a note.}% -- preceded by the symbol \anar
\\ \indent \textit{Presently,  it is important to emphasise the totalitarian drift of the specialisation and professionalisation in our societies. The first one reduces the knowledge to one or few topics against all holistic observations, and by induction the second one perverts human relationships because the professionnel acts a routine as a specialist. This means an over-reliance of the system and its institutions, notably by partitioning responsibilities.}
} \textit{described by Aldous Huxley in 1932, under the `approving gaze' of} \textsl{Big Brother}\endnote{\textit{Big Brother} -- en référence au roman de George Orwell \textsl{1984} publié en 1949 -- est un système de surveillance et de propagande omniprésente -- incarnant le `parti' et qui encourage la délation jusqu’au sein de la famille --, et par extension tout dispositif, notamment par le biais de l'intelligence artificielle connectée, qui nuit à la liberté individuelle et à la vie privé, généralement sous le prétexte fallacieux de sécurité nationale.}. 
\textit{There is definitely no future for this world}\endnote{\textit{The inevitable collapse of our civilisation concerns the hegemony or even the supremacy of money. Indeed, Money -- i.e. the haves, the super-rich -- rules the world against Nature, promoting notably competitiveness instead of sharing resources.}}. \textit{We have to build from scratch a new world. There is no miracle recipe, but some alternatives exist}\endnote{\textit{Some of them require -- in the context of political transition -- to be mentioned such as: 
\begin{itemize}
\item A Mondial negative growth is obviously needed -- at least according to the actual standard, which means positive growth on different markers.
\item An Unconditional Universal Income, notion imagined and developed by the french sociologist and economist Bernard Friot.
\end{itemize}}} \textit{and can be the daylight of our hopes. 
For instance, at our modest level, the restructuration from tribes, clans or any small communities should do consensus for a New World based on the respect of life and by extension Nature, and the development of inner culture to each group}\endnote{\textit{At the national level for instance is definitely too wide for cultural identity, and with the globalisation, this level tends to be the same for every country -- at least for connected people. The idea here is to deconstruct the concepts of nation, state and country.
\\ \indent Obviously, this approach goes against the established order, but this one way if we want to evolve with awareness.}}. 

\textit{Obviously, music can play a determining role inside a group of persons and between groups as embodied beings.
Moreover, music promotes interrelationship and create a living communion with each other.
The `ultimate' aim is to raise awareness of another way to think and to practice music, and by this way stimulate our own creativity and fantasy.
We need to consider the didactic approach which has to be designed to fit into a larger context such as a community or societal renewal, where each individual must participate in one way or another in the musical workshop. 
I believe that everybody can be part of musical practice in a playful way, for the benefit of  both the individual and the community level.
%  
%En effet, la musique permet un developpement cognitif significatif créant un lien social fort et permet aussi pour chacun de bénéficier de ses nombreuses vertus thérapeutiques.
Indeed, music allows significant cognitive development creating a strong social bond and also allows everyone to benefit from its many
therapeutic virtues on the body and mind}\endnote{...}.

 
%Indeed, music allows a significant cognitive development creating a strong social bond and also allows for everyone to benefit from its recognised therapeutic virtues. 
%Of course, a minimum of framing structure is necessary and can be inspired by the functioning of some experimental schools, but without age or social discrimination.

%Also, the creative aspect is far from neglected, and must be the engine of the workshop. In this respect, the creative perspectives allow for an innovative approach both in the composition process and in the sound result, especially if one adds the possibility of creating a « meta-workshop » grouping together several « communities » within the framework of an annual event for example. 
%Finally, the recognition of creative ideas in terms of paternity must be framed for the benefit of all. 

%I wish wholeheartedly to be the actor and the witness of this musical and human revolution.

\bigskip

%The `ultimate' aim is to raise awareness of another way to think and to practice music, and by this way stimulate our own creativity and fantasy. I believe that everybody can be part of musical practice in a playful way, for the benefit of a significant cognitive development at both the individual and the community level.

\textit{Beforehand, we mean to redefine music, historically and for this purpose as a collective work within a never-ending rhizomatic learning}\endnote{\textit{Le rhizomatic learning} est un modèle où la recherche d'un résultat est effectuée par les participants de manière dynamique en fonction du contexte. Il met l'accent sur l'interconnexion des compétences de chacun par consensus.\\ \indent Le terme rhizome est attribué à Gilles Deleuze et Félix Guattari évoqué dans \textit{Mille plateaux }  -- édité en 1980 -- dans lequel ils dévelopent l'idée d'une `structure organique' définie par les principes suivants : \begin{itemize}%[label=\textbullet]
\item « \textit{principe de connexion et d'hétérogénéité} [...] \textit{n'importe quel point d'un rhizome peut être connecté à un autre, et doit l'être} »
\item « \textit{principe de multiplicité} [...] \textit{l'organisation propre du multiple en tant que tel, qui n'a nullement besoin de l'unité pour former un système} »
\item « \textit{principe de rupture assignifiante}  [...]  \textit{un rhizome peut être rompu, brisé en un endroit quelconque} »
\item « \textit{principe de cartographie et de décalcomanie} » (la carte est l'exploration de la structure, c'est à dire \textit{une expérimentation en prise sur le réel}, tandis que les calques sont les éléments structuraux et identifiés comme tels.)
\end{itemize}
%\\ -- « \textit{principe de connexion et d'hétérogénéité} [...] \textit{n'importe quel point d'un rhizome peut être connecté à un autre, et doit l'être} »\\ -- « \textit{principe de multiplicité} [...] \textit{l'organisation propre du multiple en tant que tel, qui n'a nullement besoin de l'unité pour former un système} »\\ -- « \textit{principe de rupture assignifiante}  [...]  \textit{un rhizome peut être rompu, brisé en un endroit quelconque} »\\ -- « \textit{principe de cartographie et de décalcomanie} » (la carte est l'exploration de la structure, c'est à dire \textit{une expérimentation en prise sur le réel}, tandis que les calques sont les éléments structuraux et identifiés comme tels.)
}. \textit{This means} `pensez la musique autrement', \textit{or in other words, get out of our mercantile logic and rid of fanatic postures.
There is no need to retrace our steps or copying tribal lifestyles, but rather to make music our own, based on its} \textsl{épistémè}\endnote{Contrairement à l'épistémologie qui relate de la connaissance diachroniquement suivant une logique généalogique, l'\textit{épistémè} est la contextualisation synchrone de la connaissance relatant de la complexité d'un système dans l'espace et le temps, en d'autre terme, le contexte historique et social dans lequel la connaissance est acquise.
%\textit{\textbf{Épistémè}}: unlike epistemology which relates knowledge diachronically in a genealogical tree, \textit{épistémè} is the synchronous contextualisation of knowledge relating complexity of a system of thought in space and time, in other words, the societal and historical context in which knowledge is spread.
} 
\textit{within the group.}
%Il ne s'agit pas de faire un retour en arrière ou de copier des modes de vies tribaux, mais bien de s'approprier `notre' musique en tenant compte de son \textit{épistémè}\endnote{Contrairement à l'épistémologie qui relate de la connaissance diachroniquement suivant une généalogie, l\textit{épistémè} est la contextualisation synchrone de la connaissance relatant la complexité d'une système dans l'espace et le temps, en d'autre terme, le contexte historique et social dans lequel la connaissance est acquise.}.

\bigskip

%    
  
\section{Contexte historique}

D'un point de vue historique et formel, le fait musical fut décrit -- et l'est encore -- la plupart du temps dans un rapport  d'opposition paradigmatique en terme de discrimination social tel que la musique profane versus la musique sacré, la musique traditionnelle versus la musique savante, et la musique populaire versus la musique élitiste.
%- musique profane versus musique sacré -- introduisant les notions de monophonie et de polyphonie;
%- musique traditionnelle versus musique savante -- par extension danses et développement de la forme;
%- musique populaire versus musique bruitiste -- comprenant l'industrie musicale et la conceptualisation de l'objet sonore.
En fait d’opposition et d’un point de vue purement musicale, la disjonction
de cette dualité paradigmatique est la plupart des cas beaucoup plus diffuse que ce que les
termes en eux-même pourraient laisser entendre. Cela dit, il s’agit bien du rôle
que la musique tient dans la vie sociale qui en détermine véritablement sa nature,
indépendamment de l’objet musical lui-même. Autrement dit, l’interprétation
musicale s’inscrit dans une démarche herméneutique\endnote{L'herméneutique est l'\textit{ensemble des connaissances et des techniques qui permettent de faire parler les signes et de découvrir leur sens} -- selon la définition de Michel Foucault dans \textit{Les Mots et les choses}, édité en 1966.}.

%Ces deux paradigmes s’inscrivent alors et aussi dans le temps en termes de synchronie et de diachronie. 
\bigskip

Dans un même ordre d'idée et d'un point de vue structurel, nous pouvons identifier trois paradigmes musicaux respectivement inclusifs tels que la musique de sons\endnote{Cette dénomination mérite d'être préciser. La musique de sons se référe présentement à l'objet sonore en tant que tel et plus spécifiquement à une ontologie du phénomène vibratoire compris comme un fait musical.}, la musique modale\endnote{La modalité est un système de construction scalaire permettant une variété de mélodie et de configurations motiviques dans son contexte culturel et compositionnel.  Autrement dit, la modalité est le résultat d’une selection de sons -- pour un instrument donné, notamment en fonction de sa facture et/ou de son intonation --  dans un contexte culturel donné, et qui constitue les `phonèmes' d’une phraséologie musicale codifiée.%    La période comprend la musique de la grèce antique – à partir de l'épitaphe de Seikilos – et les chants grégorien transcrit principalement en neumes. ...
} et la musique tonale\endnote{La tonalité est un cas particulier de la modalité dont la hiérarchisation des degrés crée des pôles d'attraction, notamment de la sensible vers la tonique, selon les deux modes diatoniques %heptatoniques 
mineur et majeur -- le premier étant le relatif du second. Ainsi, la tonalité repose sur la dualité tension/détente, dissonance/consonance définit respectivement par la cadence du cinquième au premier degré du ton considéré.\\ \indent La tonalité est un compromis entre une volonté d’unifier les échelles modales selon les lois de l’acoustique -- i.e. relatif aux harmoniques du monocorde -- et d’universalisme, en passant par le temperament égal, et qui constitue encore de nos jours le standard tant en pratique qu’en théorie.% La majeur partie des musiques dites commerciales sont de nature tonale.%    reposant sur la dualité tension/détente ... tonique/dominante sensible etc ...
}, lesquelles s'articulent autour de trois périodes historiques. %La première articule la modalité vers la tonalité, par l

La première articulation pourrait commencer par la découverte d'instruments préhistorique dont le caractère modal est avéré tel que les flûtes datant du paléolithique supérieur -- soit il y a environ 35\,000 ans\endnote{Il s'agit %pour les plus anciens artefacts -- datées au carbone 14 -- 
de flûtes en os et en ivoire, découvertes en septembre 2008 dans la grotte de Hohle Fels, située dans le Jura Souabe à Bade-Wurtemberg en Allemagne.} --, ou bien aux premières traces de musiques écrites sous forme de textes accompagnés et/ou chantés, il y a 3\,400 ans pour les plus anciennes\endnote{Il s'agit de chants Hourrites accompagnés d'instructions pour instrument à neuf cordes, gravés sur tablette d'argile en écriture cunéiforme, découverts en 1950 dans l'ancienne cité d'Ougarit en Syrie.}.%, ou bien encore ... monocorde pythagore, epitaph de seikilos ...

La deuxième articulation concerne l'émergence des tonalités majeure et mineure durant la Renaissance, incluant l'établissement du tempérament égal\endnote{Le tempérament égal divise l'octave en douze demi-tons égaux, permettant de moduler vers n'importe quelle tonalité; c'est un compromis à l'encontre de l'intonation juste. Le tempérament égal constitue depuis le socle normatif de la musique en général.} à la période Baroque. Le fait marquant fut le perfectionnement de l'imprimerie, notamment en Europe\endnote{Utilisant une technique d'impression dite de typographie à caractères mobiles, l'amélioration concerne le mode de fabrication de ces caractères mobiles en alliage de plomb, la presse à vis et l'encre grasse à base d'huile de lin, mise au point notamment par Gutenberg vers 1450.}, permettant l'édition et la diffusion de partitions et d'ouvrages théorique sur la musique, qui contribua largement à l'affirmation de la tonalité que connaitra l'ère classique, puis le romantisme du \textsc{xix}\ieme ~siècle, et en maintenant la modalité par la standardisation induite de la transcription musicale des œuvres du passé.

Et enfin la troisième articulation couvrant largement le \textsc{xx}\ieme ~siècle va déconstruire\endnote{La déconstruction concerna l'art dans son ensemble depuis les origines, de l’art pariétal paléolithique au monochrome de Kasimir Malevich, du \textit{musilangage} au silence de John Cage, des Venus préhistoriques à l’urinoir de Marcel Duchamp; autrement dit de l’œuvre sans signature à la signature sans œuvre.} la tonalité, puis la modalité pour re-découvrir la musicalité du phénomène sonore, élément constitutif de la musique de sons. Le point de départ se situe durant la révolution industrielle avec l'invention de la téléphonie et de l'enregistrement sonore vers la fin du \textsc{xix}\ieme ~siècle, qui va littéralement `déterritorialisé' %\textit{déterritorialisé}\endnote{\textit{\textbf{Déterritorialisation}}:  concept créé par Gilles Deleuze et Félix Guattari en 1972 ...}
 l'objet sonore dans l'espace et dans le temps. Ce phénomène va amplifier -- dans tous les sens du terme -- et %démocratiser
 diffuser la tonalité comme standard musical, avec le tempérament égal, ainsi que la modalité avec l’ouverture aux musiques du monde, %, initiée par les compositeurs du \textsc{xx}\ieme ~siècle, depuis 
 à partir de l’exposition universelle de Paris de 1889, et entériner par la diffusion discographique et radiophonique durant les années 60, ce qui a permis d’élargir l’horizon musical en termes de pratique et d’écoute au delà de nos propres cultures.%musicales. 

Dans le même temps, de nouveaux instruments dits analogiques\endnote{Par définition, les instruments `dits' analogiques se caractérisent par le ou les traitements en fonction continu du temps d'un signal électrique.} ouvrent un potentiel inouï tant au niveau du timbre que de la technique compositionnelle.

À cela, s'ajoute la révolution informatique à la fin du \textsc{xx}\ieme ~siècle,  par la démocratisation grâce à l'internet des ressources numériques -- i.e. documentations, informations et applications -- permettant à tout à chacun de produire et de partager sa propre musique avec le monde entier, avec de nouvelles pratiques compositionnelles, des techniques de diffusions inouïes et des qui restent à inventer -- du moins pour ce qui concerne le monde `connecté'.

%À cela, s'ajoute la révolution informatique -- dit aussi numérique -- qui permet d'accéder et de partager la musique dans le monde entier, avec de nouvelles pratiques compositionnelles, des techniques de diffusions inouïes et des qui restent à inventer. 

%de nouvelles techniques de diffusions inédites et de nouveaux horizons qui restent à inventer. %Le côté obscure de l'internet est la tendance à aplanir les différences culturelles 
%Il pourrait s'agir d'une quatrième articulation, mais nous n'avons pas encore le recul nécessaire pour le dire et nous ne savons pas encore vers quel paradigme celui-ci  



% préciser dans une dynamique "tribale" au sens de micro societé (à développer) ...

\bigskip
%pouvant s'inscrire dans une potentielle et nouvelle articulation.

Voilà pour l'essentiel. Pour illustrer notre propos, il nous faudra revenir sur la déconstruction musicale évoquée dans la troisième articulation. Cette dernière semble nous mener vers ce que nous pourrions appeler un cycle cumulatif qui se clos sur un retour à la musique de sons, dans un contexte bien différent en termes de connaissances acquises, de pratiques et de productions sonores. %Bien que nous n'ayons pas le recul nécessaire, nous pouvons observer   
C'est bien dans cette perspective holistique que nous pouvons créer une véritable rupture, en nous réappropriant le monde sonore vers un objectif commun; un nouveau paradigme de musique vivante.

\bigskip

%    
  

\section{L'œuvre ouverte}

Il faut dire que durant le \textsc{xx}\ieme ~siècle, il y eut beaucoup d'effervescence artistique et musical, et la musique de sons prit des formes aussi variés que pu l'être l'art en général, et la déconstruction fut une véritable épopée.

Il serait vain ici d'en faire une description %-- de surcroit exhaustive -- 
tant la terminologie stylistique\endnote{La terminologie est loin d'être clair. En effet, d'aucuns parleront de genre ou de mouvement, mais la différence -- voire la \textit{différance} derridienne -- est d'ordre typologique et chronologique, si bien que l'on pourrait en établir la généalogie. %Du prototype à l'archétype jusqu'au stéréotype, la nuance est subordonnée au régime doxique et épistémique.
Pour ma part, il est plus intelligible de parler de prototype pour une particularité admise au sein d'une communauté, d'archétype pour la modélisation d'un ou de plusieurs prototypes et/ou archétypes, et de stéréotype pour la `normalisation' doxique et épistémique d'un ou de plusieurs archétypes.\\ \indent À noter que la \textit{doxa} est devenu de plus en plus manipulable et conditionnée par les mass média et les réseaux sociaux. De plus, l'\textit{épistémè} s'opère  traditionnellement par voie institutionnelle, bien que la tendance s'amenuise grâce notamment à l'accès et au partage du savoir par l'internet.}  multiplie les dénominations et se prête volontiers à une polysémie, parfois confuse, inclusive, dont les contours restent incertains, subtils, aussi bien pour les musicologues que pour les acteurs se réclamant, ou ne se réclamant pas, de telle ou telle mouvance.

Cela dit, il est possible d'esquisser, selon l'analyse paradigmatique précédemment évoquée et selon l'\textit{instrumentarium}\endnote{L'\textit{Instrumentarium} se rapporte à un ensemble d'instruments, à une formation, et qui se caractérise par son organologie; laquelle 
 consiste essentiellement à décrire tout objet destiné à produire des sons par le comment, le pourquoi, depuis quant, par qui ces sons sont-ils produits, pour jouer quelle musique, et à quelle fins; et par extension à répertorier ces objets pour une culture donnée et dans son ensemble à l'échelle du monde entier.}, les différentes tendances musicales de ce \textsc{xx}\ieme ~siècle.
 Sachant que la musique tonale est une interprétation hiérarchique de la musique modale, laquelle s'inscrit dans la musique de sons, l'\textit{instrumentarium} peut aussi se résumer de manière inclusive par les instruments respectivement  %-- et par extension tout objet -- 
 à hauteurs indéterminés, à hauteurs déterminés et tempérés, dont la manipulation compositionnelle s'effectue de manière scripturale, analogique, ou numérique. 
 Ainsi, selon ces critères, la combinaison de ces éléments caractérisant l'objet musical résume assez bien les spécificités inouïes de la musique du \textsc{xx}\ieme ~siècle. 
 
 Ceci vaut pour le phénomène sonore en tant qu'objet musical perçu. Au delà ou en deçà , il y a  le contexte situationnel ou rituel dans lequel la structure et la forme, c'est à dire le déploiement en temps selon des règles et des objectifs circonstanciels prédéfinis, vont prendre une dimension sociétale et communautaire %-- voire identitaire -- 
tacite, implicite, convenu, pour asseoir une autorité quelconque, faire école, faire partie d'une esthétique lié à un mouvement artistique, et que sais-je encore?  


Tout cela s'est construit, si l'on peut dire, sur la déconstruction de la tonalité. En effet, celle-ci va déterminer ce que sera la musique de sons d'aujourd'hui, notamment  par la valorisation des sons dits extra-musicaux, par l'invention de systèmes dits atonales jusqu'à l'impasse que fut le sérialisme intégral, par une approche analytique de plus en plus précise allant au cœur du son, et par les tentatives de taxonomie axiomatique du phénomène sonore et musical. De fait, la `reconstruction' prendra des formes les plus inattendues et souvent les plus surprenantes, comme par exemple des concerts de `bruits', de percussions, de synthétiseurs de toutes sortes selon leur degré de technicité, des installations et des happenings mettant à rude épreuve les habitudes du public plus ou moins mélomane, provoquant tantôt indignation et fanatisme, tantôt incrédulité et fascination. %Ces comportements extrêmes s'estompèrent progressivement, par la connaissance et l'habituation, une forme de `déjà vu' ou `entendu' les désamorçant, par l'épuisement des possibilités combinatoires inouïes\endnote{Si tant est que cela soit possible, il doit bien en rester quelques unes à découvrir ou à re-découvrir.}, même si la musique reste un puissant véhicule d'affects.  

%tout cela ne surprend plus grand monde de nos jours, même si la musique reste un puissant véhicule d'affects.  

%à ce propos il serait intéressant  d'établir une généalogie des genres et des styles
%Pour mémoire, et dans les grandes lignes, la déconstruction de l'ultrachromatisme wagnerien débouchera sur le sérialisme integral de Stockhausen, en passant par le dodécaphonisme shoenbergien, le bruit au sens de son complexe à hauteur indéterminé s'invitera de plus en plus dans le champ du possible, de Russolo et de Varèse jusqu'à Lachemann, en passant par le spectralisme  de Grisey, pour les musiques écrites.
\bigskip

%À la marge de tous ces événements, 
%L'œuvre ouverte est un pan
C'est dans les années 60 que quelques artistes envisagent le concept d'œuvre ouverte. Par définition, l'œuvre ouverte est la somme d'événements, voire la somme d'individu, d'une situation donnée, comme l'expression de cette situation par la médiation de l'art en question. C'est un processus, intégrant \textit{a priori} et \textit{in situ}, une potentielle ou effective contextualisation.

Ainsi, dans l'art sonore, certains compositeurs ont tenté avec plus ou moins de succès une reconstruction interactive avec les interprètes et parfois même avec le public.

%L'enregistrement sonore va permettre des techniques de transf
\bigskip\bigskip
   [à préciser -- de quelles tentatives il est question, etc.]
\bigskip

  
Bien que des tentatives en ce sens ont déjà été réalisé dans le cadre pédagogique pour les écoles de musique, grâce notamment aux initiatives de Jean-Yves Bosseur consignées dans l'un de ses livres textes-partitions \textit{Le temps de le prendre}, ce mode de pratique musicale n'a pas pu transcender nos habitudes culturelles\endnote{Ces habitudes sont ancrés en nous selon un mode atavique systémique de soumission pour lequel il est difficile, dans un premier temps d'en avoir conscience, et dans un second temps, de s'en émanciper.} tant le lobby de l'industrie musicale sous tout ses aspects privilégie l'attitude mercantile du fait musical. 


Cela dit, cette démarche était encadrée dans un context institutionnel permettant de promouvoir une pratique musicale \textit{hic}  et \textit{nunc} et une approche compositionnelle autre pour enfin créer des liens avec l'enseignement traditionnel. Bien que cela n'est pas fait consensus dans le temps et dans l'espace, Il en reste des traces de nos jours dans les ateliers dit d'éveil musical pratiqués dans certaines classes de musique au collège ou dans les écoles de musique. 

%\bigskip
%
%Ceci constitue une des raisons pour laquelle cela ne peut se faire qu'en dehors des institutions.

\bigskip

%\textit{Cette approche est liée à une écoute, aussi affinée que possible, du monde sonore qui nous entoure, à ce que peuvent impliquer des échanges musicaux à l'intérieur d'une collectivité lorsque la musique n'est plus seulement propulsée comme produit à consommer, ou conçue comme recherche de laboratoire (ce qui représente, pour beaucoup, l'image de marque de la musique dite contemporaine), mais plutôt comme processus de sensibilisation et de questionnement, modelable selon les nécessités et facultés d'invention de chacun.} J.-Y. \textsc{Bosseur}, \textit{Le temps de le prendre}, Paris, Éditions Kimé, 1997, p. 202.

%\bigskip
%Malheureusement cela n'a pas fait consensus dans le temps et dans l'espace, si bien qu'aujourd'hui il n'en reste qu'une anecdote -- dans le meilleur des cas -- dans la littérature consacrée à l'histoire de la musique du XX\up{e} siècle.
\bigskip\bigskip\bigskip
   [à reformuler...]

%  
Pour revenir à notre propos, il va sans dire qu'il ne s'agit pas de remettre en cause des pratiques musicales s'inscrivant dans la tradition, mais plutôt de réinventez un monde à partir de nos acquis dans une perspective de recherche continuelle et sans cesse renouvelée de notre monde sonore impliquant les forces vives de chacun.

Il est alors question ici de se repositionner par rapport à ces tentatives, de modéliser une pratique issue de ces expériences, qui n'ont en aucun cas était des échecs, mais plutôt une avant-garde que l'histoire nous pousse à réitérer aujourd'hui, dans un contexte sociétal qui se doit d'évoluer; pour lors, ce nouveau paradigme a pour vocation d'initier un mouvement culturel en tant que tel et social éthologiquement -- voire ethnologiquement  -- parlant au regard du contexte historique.%\footref{efn:pr}. 

%The next endnote\endnote{This endnote gets a label.}\label{en:test} has  the number~\ref{en:test}. Let's now test endnotemark\endnotemark[\ref{en:test}].

\bigskip

%    
  

\section{Qu’est-ce que la musique?}

...

\bigskip

%    
  

\section{Pensez la musique autrement}

Vous l'avez compris, la musique ne peut se résumer au système tonal et tempéré, à une mélodie que l'on siffle ou que l'on chante sous la douche, ou à une pulsation cardio-emotive.

Maintenant, on peut appréhender la musique, dans une continuité dite de tradition -- perpétuation d'une codification --, ou bien dans une logique de rupture -- recherche de nouveaux paradigmes. Bien entendu, la rupture se fait selon la catastrophe thomienne, dans une sorte de continuité, pour le moins en temps. 

On peut aussi l'appréhender respectivement comme acte d'expression ou de créativité, lesquels ne sont évidemment pas exclusifs.

À cela, j'ajouterai le développement personnel et collectif, ce qui constitue, au delà de l'art pour l'art, ou au delà d'une posture artistique consciente, un pas vers l'émancipation du sujet vers sa réalisation, faisant partie d'un tout.


%on peut se laisser porter par la musique ...


\bigskip

%    
  

\section{Objet sonore}

...

\newpage
  
\nolinenumbers
\setenotez{list-name = {Notes}}
\printendnotes

\newpage

\end{document}