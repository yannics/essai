\documentclass{article}

%% count number of words:
% $ pdftotext essai.pdf - | wc -w

\usepackage[french]{babel}
\usepackage[T1]{fontenc}
\usepackage[utf8]{inputenc}
\usepackage{setspace}
\usepackage{longtable}
\usepackage{xcolor}
\usepackage[font=small,margin=1mm]{caption}
\usepackage[toc,page]{appendix}
\usepackage{tocloft}
\addtocontents{toc}{\cftpagenumbersoff{part}}
\addtocontents{toc}{\cftpagenumbersoff{subsection}}
\renewcommand\cftsecfont{\bfseries}
\renewcommand{\cftaftertoctitle}{\vspace{5pt} \hrule}
\usepackage[nottoc,notindex]{tocbibind}
\usepackage[pdftex,backref=false,pdfborder={0 0 0}]{hyperref}
\usepackage{textcomp}
\makeatletter
\renewcommand{\@biblabel}[1]{}
\makeatother
\pagecolor{yellow!10}
\title{%\textsf{\Huge \textbf{Open Composition}}
}
\author{\textsf{\textbf{Yann Ics}}}
\date{}
%\setlength{\oddsidemargin}{10pt} % Marge gauche sur pages impaires
%\setlength{\evensidemargin}{10pt} % Marge gauche sur pages paires
%\setlength{\marginparwidth}{84pt} % Largeur de note dans la marge
%\setlength{\textwidth}{381pt} % Largeur de la zone de texte (17cm)
\setcounter{secnumdepth}{0} 
\usepackage[modulo]{lineno}
%\linenumbers

%%******************* added <<<<<<<<<<<<<<<
\makeatletter
\renewcommand*\l@section{\@dottedtocline{1}{1.0ex}{3.5ex}} % add dots to TOC
\def\addcontentsline#1#2#3{%
    \addtocounter{linenumber}{-1} %  the line number of the section name
    \addtocontents{#1}{\protect\contentsline{#2}{#3}{\thelinenumber}{}%
   % \protected@file@percent
    }
    \stepcounter{linenumber} % back to right linenumber
}
\makeatother

\renewcommand{\contentsname}{Contents~\hfill\textbf{Line}}
\usepackage{enotez}
% https://tex.stackexchange.com/questions/202805/refer-to-an-endnote-with-the-enotez-package-points-to-the-wrong-note
\renewcommand{\footnote}{\endnote}
\newcommand{\footref}[1]{%
  \enotezwritemark{\enmarkstyle\ref{#1}}%
}
%\let\endnote=\endnote
\newcommand{\eo}{\textsl{etou}}
\definecolor{DarkRed}{RGB}{139,0,0}
\pagenumbering{gobble} 
\usepackage{enumitem}
\usepackage{contour}
\usepackage{ulem}
\AtBeginDocument{\addtocontents{toc}{\protect\thispagestyle{empty}}} 

\renewcommand{\ULdepth}{1.8pt}
\contourlength{0.8pt}

\newcommand{\myuline}[1]{%
  \uline{\phantom{#1}}%
  \llap{\contour{white}{#1}}%
}

\newcommand{\myulineyellow}[1]{%
  \uline{\phantom{#1}}%
  \llap{\contour{yellow!10}{#1}}%
}
\begin{document}

%\begin{minipage}{.033\textwidth}\includegraphics[width=\linewidth]{1123}\end{minipage}

\makeatletter
  \begin{titlepage}
  \centering
  {\Large \@author} \\
  \rule{\linewidth}{.5pt}
  \vspace{4cm}
  
  {\LARGE \textbf{\@title}} \\
   \vspace{6 mm}
   {\color{brown} {\Large \textbf{\texttt{\, ... à la recherche d'un nouveau}}}}\\
   \vspace{2mm}
   {\color{brown}{\Large \textbf{\texttt{\qquad paradigme de musique vivante.}}}}\\
%    \vspace{12mm}
%   {\color{black}{\normalsize \qquad \textsl{\texttt{The truth is 
%   \textup{\myulineyellow{until proven otherwise}}
%  %\textup{\myulineyellow{fortunately always}} 
%   out there}}$^*$}}\\
  \vfill
  \rule{\linewidth}{.5pt}
%  in search of a new paradigm of living music.
     {\Large \textsf{\textsl{}}}\\ 
          {\Large \textsf{\textsl{}}}\\ 

      {\Large \textsf{Essai}}\\

 \end{titlepage}
\makeatother
\setcounter{tocdepth}{1}
\renewcommand{\contentsname}{Sommaire\\}

{\color{DarkRed}\tableofcontents
%\addtocontents{toc}{~\hfill\textbf{Ligne}\par}
\noindent\hrulefill}

\pagecolor{white}

%% blank page
%\newpage
%\thispagestyle{empty}
%\mbox{}
%\newpage

%\vspace*{,5cm}

\renewcommand\thelinenumber{\color{purple}\arabic{linenumber}}
\linenumbers
  
\section{Avertissement}

Le but de cet essai est d'exposer et de contextualiser une pensée simple dont la mise à l'écrit s'est révélé d'une complexité déroutante, tant une question en appelle une autre, et tant la réponse peut soulever des concepts inattendus, pluridisciplinaires, reliant parfois des savoirs disparates. Aussi, 
je revendique un style d'écriture qui se veut concis et aussi précis que possible -- à la limite de l'aphoristique --, à l'encontre d'une quelconque vulgarisation, et qui requiert parfois l'utilisation d'une terminologie spécifique. En effet, il existe des mots, qui synthétisent, subsument, condensent et cristallisent des concepts, 
et d'autres, plus diffus, parfois en cours de conceptualisation, paradoxalement éloquent, 
et qui révèlent un monde aussi complexe que fascinant, pour lesquels j'estime opportun d'en illustrer la pertinence.
%, et selon ma propre sensibilité. 
J'avoue prendre le risque de ne pas être lu ou compris, mais à l'heure où l'accessibilité du savoir est à la portée de tous -- ou presque --, j'invite le lecteur, le cas échéant, à rechercher, disséquer, au delà de mon propre propos, le sens de certain concepts, mots, peu ou pas développer à dessein, et de ne pas hésiter à déconstruire mon modeste édifice, usant pour les plus aventureux de la grammatologie déridienne\endnote{Jacques Derrida écrit en 1967 \textit{De la grammatologie}, %
% SOURCE [MHRA style] -- Contributeurs à Wikipedia, 'De la grammatologie',  Wikipédia, l'encyclopédie libre, 14 juillet 2018, 10:54 UTC, <https://fr.wikipedia.org/wiki/De_la_grammatologie?oldformat=true#Port%C3%A9e_de_la_grammatologie&oldid=150370940> [Page consultée le 19 septembre 2021]
ouvrage dans lequel il vise à décrire, à comprendre, les relations entre signifiant et signifié d'un texte, \textit{à savoir une science qui ne cesse de revenir sur ses conditions de possibilité, une science qui n'avance qu'en régressant vers le doute qui est son fondement}. Autrement dit, la grammatologie deridienne est un concept qui consiste à déconstruire le language à la manière d'un horloger afin d'en comprendre les mécanismes épistémiques\footref{efn:ep}. \textit{Si la science et la philosophie ont la vérité comme objet visé, la grammatologie voit en revanche dans l’écriture un phénomène sans vérité}.}. 
% SOURCE ---> https://hal.archives-ouvertes.fr/hal-01504272/document -- PAR-DELÀ LE SENS, L’ECRITURE ET LE TEXTE. JACQUES DERRIDA Par Joseph-Igor MOULENDA
%\bigskip

\section{…}
Suite à une longue reflexion au sujet d'accéder, de pratiquer, de créer et de partager le phénomène sonore en tant qu'objet musical, et à une expérience de vie dans laquelle je suis toujours en quête par ailleurs d'un ailleurs, 
% et à cause d'une tendance misanthropique  
 je me propose d'exposer sous la forme d'un essai ma vision et mes espérances sur notre propre existence et notre propre ontologie, et ce à travers le prisme de l'expérience acoustique et musical. En effet, si il est devenu évident que la musique soit un produit de consommation courante, avec ceux qui la font et ceux qui la consomme, au regard de ceux qui la vive au sein d'une recherche introspective ou d'un rituel,
 %d'une communion propre à un environnement propice
il m'est possible de ressentir un vide entre le fait de ne pas souscrire à la logique consumériste et de ne pas avoir d'ascendant sur l'experience musicale en tant que pratique collective.
% -- certes, excepté les quelques élus à la gloire de l'excellence pour lesquels on se doit de s'incliner. 
Serait-ce l'horizon d'un nouveau paradigme, \textit{à la limite du pays fertile}. %
% SOURCE -- --> Paul Klee, --> Pierre Boulez
% les tenants et les aboutissants
Sans aucun doute, et l'objet de cet essai est d'en identifier le substrat et d'en décrire les modalités praxéologiques\endnote{ De \textit{Praxis}, qui désigne la pratique d'une activité visant au développement du sujet.}\label{efn:pr}. 

\bigskip
Ceci est purement théorique, car toutes mes tentatives en ce sens ont été des échecs, et se sont limitées à une espèce de bruit ressenti comme incontrôlable et incontrôlé qui n'a pas enthousiasmé les participants. Probablement dû à une éloquence défaillante de ma part et aussi à une absence de linéaments -- quoique cela aurait du être définit par le groupe --, en bref une préparation inadéquate\endnote{L'idée première était d'initier un processus par la prise de décision par consensus, lequel devait être introduit conceptuellement, demandant un `effort intellectuel' que l'audience n'était pas prêt, ni préparé, à exercer.\\ \indent La prise de décision par consensus est un processus impliquant tous les participants afin de résoudre une problématique pour le bénéfice de tous. Cela implique l'analyse du projet pour en comprendre les tenants et les aboutissants, afin d’être discuter jusqu'à ce que chaque personne soit d'accord. Si une personne désapprouve la résolution, le groupe se doit de proposer une alternative jusqu'à ce que tout le monde soit d'accord. Bien sûr, cela peut impliquer certaines règles en cas de désaccord afin de maintenir la cohésion du groupe. \\ \indent À noter que ces règles sont aussi proposées par consensus et peuvent par conséquent évoluées, être adaptées ou changées, soulevant la problématique d'initier ou de proposer les prémisses opératoires selon la propre sensibilité du groupe, et ce sans être sous l'influence d'un modèle quelconque, tant que faire se peut.}$^{,}$\endnote{Ceci constitue le pourquoi de cet essai, dont le dessein est justement, dans un premier temps, de préparer les potentiels protagonistes à une expérimentation \textit{in vivo}, afin d'en percevoir les effets sur l'accomplissement personnel et collectif dans un second temps.}. Cependant, ce `nouveau paradigme' doit se construire selon un mouvement consensuel des participants qui ne souffre ni leader, ni  maitre, et doit émerger d'une édification systémique. %(1) du moins de façon permanente. De plus, comme toutes \textit{praxis}, celle-ci doit s'inscrire dans le temps 
Bien sûr, cela doit s'inscrire dans le temps -- \textsc{je dirais pour ma part au moins une session par semaine sur une période d'au moins 6 mois pour en apprécier les perspectives}, avec une `performance' au sens rituel marquant la fin ou le début de chaque mois -- et demande une assiduité et un volontarisme (bref une véritable \textit{praxis}%\footref{efn:pr}
), qui dans le cadre proposé -- c'est à dire libre\endnote{Un des points important dans le déroulement des activités était la possibilité d’un accès libre, afin de créer son propre calendrier, en fonction bien sûr des ressources et des disponibilités du lieux et des personnes en termes de possibilités pratiques telles que l’espace, les horaires et la nature du processus en cours. Dans tout les cas de figures, c’est le partage d’expériences et de compétences sans apriori et le respect d'un calendrier convenu qui devait constituer le dessein de cet `atelier'.} -- n'a tout simplement pas trouvé, ne serait-ce que par curiosité, son `public'. Les musiciens sortent difficilement du cadre tonal, de leurs habitudes, et les autres se déclarent ineptes à toutes pratiques musicales. En bref, les uns produisent, les autres consomment, point. 

% 
%This is purely theoretical, because all my attempts in this direction have been unsuccessful, and have been limited to a kind of free jazz jamming session, which did not thrill the participants. Probably due to a lack of pedagogy on my part and also to a real lack of charisma. In addition, this musical practice must take place over time -- I would say at least over a period of 6 months with one session per week to appreciate the perspectives -- and requires diligence and voluntarism, which in the proposed framework -- that is to say free -- has simply not found, if only out of curiosity, its `audience'. The musicians have difficulty getting out of the tonal framework and the others declare themselves inept at all musical practices. In short, some produce, others consume, period.

Il faut sans doute être fou pour vouloir changer cela. Peut être ne le suis-je pas assez moi-même, et que cet écrit est finalement destiné à en inspirer certains afin de permettre à d'autres -- y compris moi-même -- d'en être les acteurs et les témoins ...
%You probably have to be crazy to want to change that. Maybe I am not enough myself, and that this writing is ultimately intended to inspire some ...

 
%\section{Propos}
\section{\textsl{Purpose}}
\textsl{It appears that our globalist civilisation has failed, and finally  be only a consumerist purpose to serve the} \textit{Brave New World}\endnote{Le \textit{Meilleur des mondes} est une projection à terme -- d’aucuns parleront de société parfaite -- basé sur le conditionnement des castes et la compartimentation du savoir et des compétences.%\\   \indent L'aspect politique -- en tant qu'état d'une possible transition -- ne sera pas trop développé au cours de cet essai, mais il est parfois nécessaire de préciser ce qui se cache derrière ce besoin de changement, ce qu'il faut changer et peut-être comment et pourquoi. Ces digressions se feront au besoin ici ou là en note.% -- preceded by the symbol  
%\\ \indent Présentement, il importe de souligner la dérive totalitaire de la spécialisation et de la professionnalisation dans nos sociétés. Le premier réduit les connaissances à une ou quelques thématiques contre toute observation holistique, et par induction le second pervertit les relations humaines car le professionnel agit de manière routinière, standardisée, en tant que spécialiste. Cela signifie une dépendance excessive du système et de ses institutions, notamment par le cloisonnement des responsabilités.
\\ \indent \textsl{The political aspect -- as a possible transitional state -- won't be developed too much during this writing, but it needs sometimes to precise what behind the need for change and what to change and maybe how and why. These digressions will be done here and there, when required as a note.}% -- preceded by the symbol \anar
\\ \indent \textsl{Presently,  it is important to emphasise the totalitarian drift of the specialisation and professionalisation in our societies. The first one reduces the knowledge to one or few topics against all holistic observations, and by induction the second one perverts human relationships because the professionnel acts a routine as a specialist. This means an over-reliance of the system and its institutions, notably by partitioning responsibilities.}
} \textsl{described by Aldous Huxley in 1932, under the `approving gaze' of} \textit{Big Brother}\endnote{\textit{Big Brother} -- en référence au roman de George Orwell \textsl{1984} publié en 1949 -- est un système de surveillance et de propagande omniprésente -- incarnant le `parti' et qui encourage la délation jusqu’au sein de la famille --, et par extension tout dispositif, notamment par le biais de l'intelligence artificielle connectée, qui nuit à la liberté individuelle et à la vie privé, généralement sous le prétexte fallacieux de sécurité nationale.}. 
\textsl{There is definitely no future for this world}\endnote{\textsl{The inevitable collapse of our civilisation concerns the hegemony or even the supremacy of money. Indeed, Money -- \textit{i.e.} the haves, the super-rich -- rules the world against Nature, promoting notably competitiveness instead of sharing resources.}\\ \indent \textit{L’obligation de produire aliène la passion de créer.}}. \textsl{We have to build from scratch a new world. There is no miracle recipe, but some alternatives exist}\endnote{\textsl{Some of them require -- in the context of political transition -- to be mentioned such as: 
\begin{itemize}
\item A Mondial negative growth is obviously needed -- at least according to the actual standard, which means positive growth on different markers.
\item An Unconditional Universal Income, notion imagined and developed by the french sociologist and economist Bernard Friot.
\end{itemize}}} \textsl{and can be the daylight of our hopes. 
For instance, at our modest level, the restructuration from tribes, clans or any small communities should do consensus for a New World based on the respect of life and by extension Nature, and the development of inner culture to each group}\endnote{\textsl{At the national level for instance is definitely too wide for cultural identity, and with the globalisation, this level tends to be the same for every country -- at least for connected people. The idea here is to deconstruct the concepts of nation, state and country.
\\ \indent Obviously, this approach goes against the established order, but this one way if we want to evolve with awareness.}}. 

\textsl{Obviously, music can play a determining role inside a group of persons and between groups as embodied beings.
Moreover, music promotes interrelationship and create a living communion with each other.
The `ultimate' aim is to raise awareness of another way to think and to practice music, and by this way stimulate our own creativity and fantasy.
We need to consider the didactic approach which has to be designed to fit into a larger context such as a community or societal renewal, where each individual must participate in one way or another in the musical workshop. 
I believe that everybody can be part of musical practice in a playful way, for the benefit of  both the individual and the community level.
%  
%En effet, la musique permet un developpement cognitif significatif créant un lien social fort et permet aussi pour chacun de bénéficier de ses nombreuses vertus thérapeutiques.
Indeed, music allows significant cognitive development creating a strong social bond and also allows everyone to benefit from its many
therapeutic virtues on the body and mind}\endnote{La pratique de la musique est un vecteur de plasticité anatomique et fonctionnelle en termes de registres perceptif, cognitif (mémoire et langage), émotionnel et moteur, impliquant le réseau neuronal dans son entièreté et ses répercussions sur l'ensemble du corps. En d'autres termes, la musique aide -- en améliorant de façon significative -- à soigner ou à soulager des sujets dans le cadre de la thérapeutique, ou tout simplement à mieux vivre, pour soi et avec les autres.}.

 
%Indeed, music allows a significant cognitive development creating a strong social bond and also allows for everyone to benefit from its recognised therapeutic virtues. 
%Of course, a minimum of framing structure is necessary and can be inspired by the functioning of some experimental schools, but without age or social discrimination.

%Also, the creative aspect is far from neglected, and must be the engine of the workshop. In this respect, the creative perspectives allow for an innovative approach both in the composition process and in the sound result, especially if one adds the possibility of creating a « meta-workshop » grouping together several « communities » within the framework of an annual event for example. 
%Finally, the recognition of creative ideas in terms of paternity must be framed for the benefit of all. 

%I wish wholeheartedly to be the actor and the witness of this musical and human revolution.

\bigskip

%The `ultimate' aim is to raise awareness of another way to think and to practice music, and by this way stimulate our own creativity and fantasy. I believe that everybody can be part of musical practice in a playful way, for the benefit of a significant cognitive development at both the individual and the community level.

\textsl{Beforehand, we mean to redefine music, historically and for this purpose as a collective work within a never-ending rhizomatic learning}\endnote{Le \textit{rhizomatic learning} est un modèle où la recherche d'un résultat est effectuée par les participants de manière dynamique en fonction du contexte. Il met l'accent sur l'interconnexion des compétences de chacun par consensus.\\ \indent Le terme rhizome est attribué à Gilles Deleuze et Félix Guattari évoqué dans \textit{Mille plateaux }  -- édité en 1980 -- dans lequel ils dévelopent l'idée d'une `structure organique' définie par les principes suivants : \begin{itemize}%[label=\textbullet]
\item « \textit{principe de connexion et d'hétérogénéité} [ … ] \textit{n'importe quel point d'un rhizome peut être connecté à un autre, et doit l'être} »
\item « \textit{principe de multiplicité} [ … ] \textit{l'organisation propre du multiple en tant que tel, qui n'a nullement besoin de l'unité pour former un système} »
\item « \textit{principe de rupture assignifiante}  [ … ]  \textit{un rhizome peut être rompu, brisé en un endroit quelconque} »
\item « \textit{principe de cartographie et de décalcomanie} » (la carte est l'exploration de la structure, c'est à dire \textit{une expérimentation en prise sur le réel}, tandis que les calques sont les éléments structuraux et identifiés comme tels.)
\end{itemize}
%\\ -- « \textit{principe de connexion et d'hétérogénéité} [...] \textit{n'importe quel point d'un rhizome peut être connecté à un autre, et doit l'être} »\\ -- « \textit{principe de multiplicité} [...] \textit{l'organisation propre du multiple en tant que tel, qui n'a nullement besoin de l'unité pour former un système} »\\ -- « \textit{principe de rupture assignifiante}  [...]  \textit{un rhizome peut être rompu, brisé en un endroit quelconque} »\\ -- « \textit{principe de cartographie et de décalcomanie} » (la carte est l'exploration de la structure, c'est à dire \textit{une expérimentation en prise sur le réel}, tandis que les calques sont les éléments structuraux et identifiés comme tels.)
}. \textsl{This means} `pensez la musique autrement', \textsl{or in other words, get out of our mercantile logic and rid of fanatic postures.
There is no need to retrace our steps or copying tribal lifestyles, but rather to make music our own, based on its} \textit{épistémè}\endnote{Contrairement à l'épistémologie qui relate de la connaissance diachroniquement suivant une logique généalogique, l'\textit{épistémè} est la contextualisation synchrone de la connaissance relatant de la complexité d'un système dans l'espace et le temps, en d'autre termes, le contexte historique et social dans lequel la connaissance est acquise.
%\textit{\textbf{Épistémè}}: unlike epistemology which relates knowledge diachronically in a genealogical tree, \textit{épistémè} is the synchronous contextualisation of knowledge relating complexity of a system of thought in space and time, in other words, the societal and historical context in which knowledge is spread.
}\label{efn:ep}
\textsl{within the group.}
%Il ne s'agit pas de faire un retour en arrière ou de copier des modes de vies tribaux, mais bien de s'approprier `notre' musique en tenant compte de son \textit{épistémè}\endnote{Contrairement à l'épistémologie qui relate de la connaissance diachroniquement suivant une généalogie, l\textit{épistémè} est la contextualisation synchrone de la connaissance relatant la complexité d'une système dans l'espace et le temps, en d'autre terme, le contexte historique et social dans lequel la connaissance est acquise.}.

%\bigskip

%    
  
\section{Contexte historique}

D'un point de vue historique et formel, le fait musical fut décrit -- et l'est encore -- la plupart du temps dans un rapport  d'opposition paradigmatique en termes de discrimination sociale tel que la musique profane versus la musique sacré, la musique traditionnelle versus la musique savante, et la musique populaire versus la musique élitiste.
%- musique profane versus musique sacré -- introduisant les notions de monophonie et de polyphonie;
%- musique traditionnelle versus musique savante -- par extension danses et développement de la forme;
%- musique populaire versus musique bruitiste -- comprenant l'industrie musicale et la conceptualisation de l'objet sonore.
On pourrait ajouter, à partir du \textsc{xx}\ieme ~siècle, en termes de discrimination sociétale\endnote{\textsl{The difference between social and societal is that `social' has to do with human relationships and groups of people within a society often class-based while `societal' has to do with society as a whole, as a system, by the way it is organised.}}, la musique `\textit{underground}' et de recherche versus la musique `\textit{mainstream}' dite aussi de masse. %diffusée en boucle dans les mass média et soutenue par les institutions en place.
En fait d’opposition et d’un point de vue purement musicale, la disjonction
de ces dualités paradigmatiques est la plupart des cas beaucoup plus diffuse que ce que les
termes en eux-même pourraient laisser entendre\endnote{Dans ce contexte nominalisme, la discrimination et la typologie, nécessaire à l'intelligibilité d'une phénoménologie,  permettent d’accéder par l'esprit à une réalité souvent complexe, à en saisir la problématique, mais créent dans le même temps des tensions, des confusions, voire des impossibilités sur les inévitables recouvrements.}. Cela dit, il s’agit bien du rôle
que la musique tient dans la vie sociale qui en détermine véritablement sa nature,
indépendamment de l’objet musical lui-même. Autrement dit, l’interprétation
musicale en ces termes s’inscrit dans une démarche herméneutique\endnote{L'herméneutique est l'\textit{ensemble des connaissances et des techniques qui permettent de faire parler les signes et de découvrir leur sens} -- selon la définition de Michel Foucault dans \textit{Les Mots et les choses}, édité en 1966.}.

%Ces deux paradigmes s’inscrivent alors et aussi dans le temps en termes de synchronie et de diachronie. 
\bigskip

Dans un même ordre d'idée et d'un point de vue structurel, nous pouvons identifier trois paradigmes musicaux respectivement inclusifs tels que la musique de sons\endnote{La musique de sons se réfère présentement à l'objet sonore en tant que tel et plus spécifiquement à une ontologie du phénomène vibratoire compris comme un fait musical.}, la musique modale\endnote{La modalité est un système de construction scalaire prédéterminé permettant une variété de mélodie et de configurations motiviques dans son contexte culturel et compositionnel.  Autrement dit, la modalité est le résultat d’une selection de sons -- corréler pour un instrument donné à sa facture ou à son intonation --  dans un contexte culturel donné, et qui constitue les `phonèmes' d’une phraséologie musicale codifiée.%    La période comprend la musique de la grèce antique – à partir de l'épitaphe de Seikilos – et les chants grégorien transcrit principalement en neumes. ...
} et la musique tonale\endnote{La tonalité est un cas particulier de la modalité, héritée de la gamme pythagoricienne, dont la hiérarchisation des degrés crée des pôles d'attraction, %notamment de la sensible vers la tonique, selon les deux modes diatoniques %heptatoniques 
%mineur et majeur -- le premier étant le relatif du second. Ainsi, la tonalité repose
reposant sur la dualité tension/détente, dissonance/consonance, définit respectivement par la cadence parfaite du cinquième au premier degré du ton considéré.\\ \indent La tonalité est un compromis entre une volonté d’unifier les échelles modales selon les lois de l’acoustique -- \textit{i.e.} relatif aux harmoniques du monocorde pythagoricien -- et d’universalisme, en passant par le temperament égal, et qui constitue encore de nos jours le standard tant en pratique qu’en théorie.% La majeur partie des musiques dites commerciales sont de nature tonale.%    reposant sur la dualité tension/détente ... tonique/dominante sensible etc ...
}, lesquelles s'articulent autour de trois périodes historiques. %La première articule la modalité vers la tonalité, par l

La première articulation pourrait commencer par la découverte d'instruments préhistorique dont le caractère modal est avéré tel que les flûtes datant du paléolithique supérieur -- soit il y a environ 35\,000 ans\endnote{Il s'agit %pour les plus anciens artefacts -- datées au carbone 14 -- 
de flûtes en os et en ivoire, découvertes en septembre 2008 dans la grotte de Hohle Fels, située dans le Jura Souabe à Bade-Wurtemberg en Allemagne.} --, ou bien aux premières traces de musiques écrites sous forme de textes accompagnés et/ou chantés, il y a 3\,400 ans pour les plus anciennes\endnote{Il s'agit de chants Hourrites accompagnés d'instructions pour instrument à neuf cordes, gravés sur tablette d'argile en écriture cunéiforme, découverts en 1950 dans l'ancienne cité d'Ougarit en Syrie.}.%, ou bien encore ... monocorde pythagore, epitaph de seikilos ...

La deuxième articulation concerne, à partir de la Renaissance et plus particulièrement en Europe, l'émergence de la tonalité en tant que système musical,  
%des tonalités majeure et mineure 
incluant l'établissement du tempérament égal\endnote{Le tempérament égal divise l'octave en douze demi-tons égaux, permettant de moduler vers n'importe quelle tonalité; c'est un compromis à l'encontre de l'intonation juste. Le tempérament égal constitue depuis le socle normatif de la musique en général.} à la période Baroque. Le fait marquant et déterminant  fut certainement le perfectionnement de l'imprimerie\endnote{En imprimerie, deux techniques font leur apparition, l'une utilisant une technique d'impression dite de typographie à caractères mobiles. L'amélioration concerne le mode de fabrication de ces caractères mobiles en alliage de plomb, la presse à vis et l'encre grasse à base d'huile de lin, mise au point notamment par Gutenberg vers 1450. L'autre est la technique dite de taille-douce, découverte par hasard par le graveur florentin Maso Finiguerra en 1452, qui consiste à graver en creux une plaque de métal, souvent en cuivre, puis rempli de noir de fumée avant d'y appliquer et presser une feuille de papier humide.} et le début d'une industrialisation papetière grâce à l'énergie hydraulique. Cela permit l'édition et la diffusion de partitions et d'ouvrages théorique sur la musique, qui contribua largement à l'affirmation de la tonalité que connaitra l'ère classique, puis le romantisme du \textsc{xix}\ieme ~siècle, tout en maintenant la modalité par la standardisation typographique induite de la transcription musicale des œuvres du passé.

Et enfin la troisième articulation couvrant largement le \textsc{xx}\ieme ~siècle va déconstruire\endnote{La déconstruction concerna l'art dans son ensemble depuis les origines, de l’art pariétal paléolithique au monochrome de Kasimir Malevich, du \textit{musilangage} au silence de John Cage, des Venus préhistoriques à l’urinoir de Marcel Duchamp; autrement dit de l’œuvre sans signature à la signature sans œuvre -- voire du transcendant à l'immanent, de l'art à l'artiste, de l'acteur/spectateur au spectateur/acteur, du signifiant au signifié.\\ \indent Noter que le \textit{musilangage} est une hypothèse de recherche transdisciplinaire qui défend l'idée d'une origine commune entre la musique et le langage; \textit{la  musique  aurait  ainsi  une  origine  vocale,  et  le  langage  une origine  musicale}.} %
% SOURCE -- Caroline Boe. Le musilangage : une hypothèse d’origine commune entre la proto-musique et le proto- langage. 2019.  hal-02004782v2 
la tonalité, puis la modalité pour re-découvrir la musicalité du phénomène sonore, élément constitutif de la musique de sons. Le point de départ se situe durant la révolution industrielle avec l'invention de la téléphonie et de l'enregistrement sonore vers la fin du \textsc{xix}\ieme ~siècle, qui va littéralement `déterritorialisé' %\textit{déterritorialisé}\endnote{\textit{\textbf{Déterritorialisation}}:  concept créé par Gilles Deleuze et Félix Guattari en 1972 ...}
 l'objet sonore dans l'espace et dans le temps. Ce phénomène va amplifier -- dans tous les sens du terme -- et %démocratiser
 diffuser la tonalité comme standard musical, avec le tempérament égal, ainsi que la modalité avec l’ouverture aux musiques du monde, %, initiée par les compositeurs du \textsc{xx}\ieme ~siècle, depuis 
 à partir de l’exposition universelle de Paris de 1889, et entériner par la diffusion discographique et radiophonique durant les années 60, ce qui a permis d’élargir l’horizon musical en termes de pratique et d’écoute au delà de nos propres cultures.%musicales. 

Dans le même temps, de nouveaux instruments dits analogiques\endnote{Par définition, les instruments `dits' analogiques se caractérisent par le ou les traitements en fonction continu du temps d'un signal électrique.} ouvrent un potentiel inouï tant au niveau du timbre que de la technique compositionnelle.

À cela, s'ajoute la révolution informatique à la fin du \textsc{xx}\ieme ~siècle,  par la démocratisation, grâce à l'internet, des ressources numériques -- \textit{i.e.} documentations, informations et applications -- permettant à tout à chacun de produire et de partager sa propre musique avec le monde entier, avec de nouvelles pratiques compositionnelles, des techniques de diffusions inouïes et des qui restent à inventer -- du moins pour ce qui concerne le monde `connecté'.

%À cela, s'ajoute la révolution informatique -- dit aussi numérique -- qui permet d'accéder et de partager la musique dans le monde entier, avec de nouvelles pratiques compositionnelles, des techniques de diffusions inouïes et des qui restent à inventer. 

%de nouvelles techniques de diffusions inédites et de nouveaux horizons qui restent à inventer. %Le côté obscure de l'internet est la tendance à aplanir les différences culturelles 
%Il pourrait s'agir d'une quatrième articulation, mais nous n'avons pas encore le recul nécessaire pour le dire et nous ne savons pas encore vers quel paradigme celui-ci  



% préciser dans une dynamique "tribale" au sens de micro societé (à développer) ...

%Bien sûr, cette analyse est eurocentrée et propose une analyse musicologique épistémologique du langage musical et de sa théorisation.

\bigskip
%pouvant s'inscrire dans une potentielle et nouvelle articulation.

Voilà pour l'essentiel. Pour illustrer notre propos, il nous faudra revenir sur la déconstruction musicale évoquée dans la troisième articulation. Cette dernière semble nous mener vers ce que nous pourrions appeler un cycle cumulatif qui se clos sur un retour à la musique de sons, dans un contexte bien différent en termes de connaissances acquises, de pratiques et de productions sonores. %Bien que nous n'ayons pas le recul nécessaire, nous pouvons observer   
C'est bien dans cette perspective holistique que nous pouvons créer une véritable rupture, en nous réappropriant le monde sonore vers un objectif commun; un nouveau paradigme de musique vivante.

%\bigskip

%    
  

\section{L'œuvre ouverte}

Il faut dire que durant le \textsc{xx}\ieme ~siècle, il y eut beaucoup d'effervescence artistique et musical, et la musique de sons prit des formes aussi variés que pu l'être l'art en général, et la déconstruction fut une véritable épopée.

Il serait vain ici de faire une énumération de ces processus de déconstruction/reconstruction %-- de surcroit exhaustive -- de ce phénomène 
tant la terminologie mise en œuvre\endnote{La terminologie est loin d'être clair. En effet, d'aucuns parleront de style, de genre ou de mouvement, mais la différence -- voire la \textit{différance} derridienne -- est d'ordre typologique. % et chronologique, si bien que l'on pourrait en établir la généalogie. %Du prototype à l'archétype jusqu'au stéréotype, la nuance est subordonnée au régime doxique et épistémique.
%Pour ma part, 
\\ \indent Dans le même ordre d'idée, il serait plus intelligible de parler de prototype pour une particularité admise au sein d'une communauté, d'archétype pour la modélisation d'un ou de plusieurs prototypes et/ou archétypes, et de stéréotype pour la `normalisation' doxique et épistémique d'un ou de plusieurs archétypes. À noter que la \textit{doxa} est devenu de plus en plus manipulable et conditionnée par les mass média et les réseaux sociaux. De plus, l'\textit{épistémè} s'opère  traditionnellement par voie institutionnelle, bien que la tendance s'amenuise grâce notamment à l'accès et au partage du savoir par l'internet.}  multiplie les dénominations et se prête volontiers à une polysémie, parfois confuse, inclusive, dont les contours restent incertains, subtils, aussi bien pour les musicologues que pour les acteurs se réclamant, ou ne se réclamant pas, de telle ou telle mouvance.

Cela dit, il est possible d'esquisser, selon l'analyse paradigmatique précédemment évoquée et selon l'\textit{instrumentarium}\endnote{L'\textit{Instrumentarium} se rapporte à un ensemble d'instruments, à une formation, et qui se caractérise par son organologie; laquelle 
 consiste essentiellement à décrire tout objet destiné à produire des sons notamment par \textit{le comment, le pourquoi, depuis quant, par qui ces sons sont-ils produits, pour jouer quelle musique, et à quelle fins}; %
% SOURCE -- Geneviève Dournon, Instruments de musique du monde : foisonnement et systématiques. In : Musiques : une encyclopédie pour le XXIe siècle - L’unité de la musique (pages 833-864), Volume 5, Actes Sud/Cité de la musique, Arles/Paris 2007. 
et par extension à répertorier ces objets pour une culture donnée, de leur singularité à leur taxonomie à l'échelle de l'humanité.}, les différentes tendances musicales de ce \textsc{xx}\ieme ~siècle.
 Sachant que la musique tonale est une interprétation hiérarchique de la musique modale, laquelle s'inscrit dans la musique de sons, l'\textit{instrumentarium} peut aussi se résumer de manière inclusive par les instruments respectivement  %-- et par extension tout objet -- 
 à hauteurs indéterminées, à hauteurs déterminées `continues', et à hauteurs prédéterminées, 
 %\endnote{La distinction entre hauteurs déterminées et hauteurs déterminées tempérées se situe dans le fait que dans le premier les hauteurs des sons sont arbitraires et donc non tempérées,  tandis que pour le deuxième, les hauteurs des sons font partie d'un système d'échelle permettant un accordage convenue selon des critères et des objectifs délibérés.},
 dont la manipulation compositionnelle\endnote{Grâce notamment au sérialisme integral, qui a ouvert une brèche sur l’aspect compositionnel concernant l’ensemble des paramètres contrôlables qui caractérisent l'objet sonore, la source sonore s'inscrit -- en plus de la hauteur et de la dynamique -- dans le mode de production et de contextualisation, tels que le traitement du signal, physique, analogique ou numérique, et l’acoustique en termes de spatialisation et de psychoacoustique.} s'effectue de manière intuitive, scripturale, analogique, ou numérique. 
 Ainsi, selon ces critères, la combinaison de ces éléments caractérisant l'objet musical résume assez bien les spécificités inouïes de la musique du \textsc{xx}\ieme ~siècle. 
 
 Ceci vaut pour le phénomène sonore en tant qu'objet musical perçu. Au delà ou en deçà, il y a  le contexte situationnel ou rituel dans lequel la structure et la forme, c'est à dire le déploiement en temps selon des règles et des objectifs circonstanciels prédéfinis, vont prendre une dimension sociétale et communautaire %-- voire identitaire -- 
tacite, implicite, convenu, pour asseoir une autorité quelconque, faire école, faire partie d'une esthétique lié à un mouvement artistique, et que sais-je encore.


Tout cela s'est construit, si l'on peut dire, sur la déconstruction\endnote{Autant certains artistes radicaux du début du XX\up{e} siècle pressentaient les dérives tyrannique de l'ère industriel, autant beaucoup d'artiste du second après-guerre, notamment durant les années 60 et 70, 
%Déjà à cette époque, beaucoup 
avait conscience de l'impasse consumériste et de l'enjeu écologique induit, et cela s'est manifesté socialement et musicalement par une sérieuse remise en question philosophique, laquelle soit dit en passant reste toujours d'actualité. Ceci expliquant cela, les premiers déconstruisent la tonalité, tandis que les seconds contestèrent la suprématie dodécaphonique et sériel d'avant-garde.} de la tonalité. En effet, celle-ci va déterminer ce que sera la musique de sons d'aujourd'hui, notamment  par la valorisation des sons dits extra-musicaux, par l'invention de systèmes dits atonales jusqu'à l'impasse que fut le sérialisme intégral, par une approche analytique de plus en plus précise allant au cœur du son, et par les tentatives de taxonomie axiomatique du phénomène sonore et musical. De fait, la `reconstruction' prendra des formes les plus inattendues et souvent les plus surprenantes, comme par exemple des concerts de `bruits', de percussions, de synthétiseurs de toutes sortes selon leur degré de technicité, des installations et des \textit{happenings} mettant à rude épreuve les habitudes du public plus ou moins mélomane, provoquant tantôt indignation et fanatisme, tantôt incrédulité et fascination. %Ces comportements extrêmes s'estompèrent progressivement, par la connaissance et l'habituation, une forme de `déjà vu' ou `entendu' les désamorçant, par l'épuisement des possibilités combinatoires inouïes\endnote{Si tant est que cela soit possible, il doit bien en rester quelques unes à découvrir ou à re-découvrir.}, même si la musique reste un puissant véhicule d'affects.  

%tout cela ne surprend plus grand monde de nos jours, même si la musique reste un puissant véhicule d'affects.  

%à ce propos il serait intéressant  d'établir une généalogie des genres et des styles
%Pour mémoire, et dans les grandes lignes, la déconstruction de l'ultrachromatisme wagnerien débouchera sur le sérialisme integral de Stockhausen, en passant par le dodécaphonisme shoenbergien, le bruit au sens de son complexe à hauteur indéterminé s'invitera de plus en plus dans le champ du possible, de Russolo et de Varèse jusqu'à Lachemann, en passant par le spectralisme  de Grisey, pour les musiques écrites.
\bigskip


Dès les années 50, de nouvelles façons de composer intègrent une part d'indétermination plus ou moins contenu, tant sur la structure que sur la forme\endnote{La structure et la forme en musique peuvent être compris de différentes manières.\\ \indent Par exemple, pour John Cage, la \textit{structure} est la divisibilité d'une œuvre, pendant que la \textit{forme} en exprime la continuité, autrement dit la \textit{morphologie de la continuité}. Cette continuité s'articule selon une \textit{méthode} qui s'applique sur le \textit{matériau} en tant qu'éléments structurants.\\ \indent Pour ce qui nous occupe, la structure est un processus immanent, c'est à dire la \textit{fractalisation} d'une caractéristique lié à la nature du matériau, tandis que la forme est un processus transcendant, au sens où le tout dépasse la somme de ses parties vers l'expérience d'une intention manifeste ou d'une limitation extrinsèque. Autrement dit, la forme est le résultat ou le phénomène émergeant d'un processus dépendant du matériau structurant et de son environnement.}, sous forme de degré de liberté accordé à l'interprète. Nous pouvons distinguer trois approches ou techniques développés depuis cette date. La première approche est une technique de composition faisant intervenir la notion d'aléatoire controlé. Cela consiste à donner une liberté d'interprétation suivant les indications du compositeur, de façon à générer une complexité visant un effet global déterminé. Cela concerne principalement les éléments structurant de l'œuvre. Une autre technique concerne les formes dites mobiles -- non sans la référence aux mobiles d'Alexander Calder --, pour lesquelles le ou les interprètes doivent combiner, selon les indications plus ou moins explicites du compositeur, des fragments musicaux prédéterminés. Puis une approche plus global sous le terme d'indétermination, notamment promu par John Cage, usant de `partitions' graphiques, textuelles, dont le but est dans un premier temps de stimuler la créativité des acteurs, pour dans un second temps évaluer l'interaction entre l'interprète et l'objet médiateur proposé par le compositeur, et ce, non pas pour exprimer quoi que se soit, mais pour exemplifier des processus générés par l'œuvre.

Ces concepts sont des modalités d'improvisation orientée (ou pas) -- qui ont pour certaines d'entre elles existé bien avant les années 50, mais de façon anecdotique et peu ou pas développés -- qui servent un dessein compositionnel, et reposent apriori sur le potentiel créatif du ou des interprètes.  Aussi, ces concepts doivent être distingués des procédés d'écriture faisant appel aux mathématiques probabilistes et statistiques ou au hasard, permettant l'organisation musicales de système plus ou moins complexe, et entièrement géré par le compositeur, et de l'improvisation qui ne nécessite aucune préparation particulière hormis la mise en pratique d'un savoir, d'une théorie, d'une pratique, soumis à l'inspiration du moment. %souvent repose sur une harmonie tonale, voire sur une liberté totale concernant le free jazz. 

\bigskip

%%À la marge de tous ces événements, 
%%L'œuvre ouverte est un pan
%À partir des années 60, quelques artistes réfléchissent sur différentes manières de composer, d'interpréter, et d'écouter de la musique, et proposent de nouveaux concepts en ce sens, tels que la \textit{moment form}, la `musique expérimentale', et l'œuvre ouverte.
%
%%envisagent le concept d'œuvre ouverte, aussi appelé `musique expérimentale'. 
%
%L'œuvre ouverte s'inscrit dans les modalités d'improvisation précédemment évoquées, avec la volonté, selon Umberto Eco, de proposer des formes complexes, voire ambiguës, en jouant avec \textit{une pluralité de signifiés qui coexistent en un seul signifiant}. Plus généralement, l'œuvre ouverte consiste à proposer une action vers l'inconnue, dans le cadre de ce qui est déterminé. À noter que la `musique expérimentale' est une spécificité outre-atlantique pour désigner une approche plutôt `cagienne'  -- comme un cas particulier d'œuvre ouverte -- telle qu'elle vise un résultat non prévisible \textit{a priori} et non déterminé pour le moins, dans la mesure où l'exécution est composée en place et lieu d'une composition exécutée.  Bien que l'œuvre ouverte soit un concept en construction, entre une visée utopique remettant en question la notion même d'œuvre d'art -- \textit{les œuvres cesseront d'être des fins en soi, pour devenir des fenêtres sur le monde} --, et l'intention de réduire, voire d'abolir, l'écart \textit{entre celui qui produit et celui qui reçoit}, elle est cette composition suggérant à l'interprète une possible action, une ouverture à un espace de jeu\endnote{Au lieu de proposer une indication précise, que l'interprète réalisera selon sa propre compréhension de l'œuvre, et parfois à l'encontre de l'intention que porte ladite indication, le compositeur ouvre un champ de possibilités, posant les jalons d'un univers que l'interprète incarnera, et dans lequel il devra lui-même s'ouvrir à l'indetermination comme une expérience, libéré de toute intentionnalité (sauf indication contraire), évitant de la sorte toute `trahison' de la part de celui-ci et se repositionnera  dans le respect des prescriptions de l'auteur.}. 

À partir des années 60, quelques artistes envisagent le concept d'œuvre ouverte. L'œuvre ouverte s'inscrit dans les modalités d'improvisation précédemment évoquées, avec la volonté de proposer des formes complexes, voire ambiguës, en jouant avec \textit{une pluralité de signifiés qui coexistent en un seul signifiant}. %, selon Umberto Eco in L'œuvre ouverte
Plus généralement, l'œuvre ouverte consiste à proposer une action vers l'inconnue, dans le cadre de ce qui est déterminé. 
Bien que l'œuvre ouverte soit un concept en construction, entre une visée utopique remettant parfois en question la notion même d'œuvre d'art -- \textit{les œuvres cesseront d'être des fins en soi, pour devenir des fenêtres sur le monde} --, et l'intention de réduire, voire d'abolir, l'écart \textit{entre celui qui produit et celui qui reçoit}, elle est cette composition suggérant à l'interprète une possible action, une ouverture à un espace de jeu\endnote{Au lieu de proposer une indication précise, que l'interprète réalisera selon sa propre compréhension de l'œuvre, et parfois à l'encontre de l'intention que porte ladite indication, le compositeur ouvre un champ de possibilités, posant les jalons d'un univers que l'interprète incarnera, et dans lequel il devra lui-même s'ouvrir à l'indetermination comme une expérience, libéré de toute intentionnalité (sauf indication contraire), évitant de la sorte toute `trahison' de la part de celui-ci et se repositionnera  dans le respect des prescriptions de l'auteur.}. 


Par définition, l'œuvre ouverte est la somme d'événement, voire la somme d'individu, d'une situation donnée, comme l'expression de cette situation par la médiation de l'art en question. C'est un processus, intégrant \textit{a priori} et \textit{in situ}, une potentielle ou effective contextualisation.

%\bigskip
%
%Bien que le concept d'œuvre ouverte est celui qui nous intéresse, il convient d'évoquer et de préciser les concepts de \textit{moment form} et de `musique expérimentale', parce qu'ils proposent des pistes de prospection sur l'acte de percevoir (de) la musique. En effet, ces derniers intègrent un paradigme d'écoute qui consiste à considéré l'objet sonore pour ce qu'il est, dans sa nature ontologique, sans lien aucun -- du moins si l'auditeur ne focalise pas son attention dans cette direction -- avec ce qui peut le suivre et ce qui l'a précédé. 
%En d'autres mots, il s'agit de considérer une sériation d'objets sonores -- ou moments pour la 
% \textit{moment form} --, lesquels répondent à la définition de 
%%\textsl{In other words, the} \textit{moment form} \textsl{is a seriation of moment defined as 
%\textit{self-contained (quasi-)independent section, set off from other sections by discontinuities}. -- idea of autonomous musical units, in order to reduce the perception toward the smallest slice of time tending toward eternity -- la forme est de type mobile, tant que les moments ne sont pas reliés entre eux ou peuvent l'être suivant une esthetique compositonel predefinit et bien sur de facon non hierarchique, et intègre de la sorte un potentiel de never-end process...
%%Quant à la `musique expérimentale', principalement définit et incarnée par John cage, le concept vise un résultat non prévisible \textit{a priori} et non déterminé pour le moins, dans la mesure où l'exécution est composée en place et lieu d'une composition exécutée. 
%Le lien qui existe avec la \textit{moment form}, la `musique expérimentale' et l'œuvre ouverte, consiste à exercer, à diversifié, à transcender nos modalités d'écoute\endnote{...}, conditionnées par notre héritage culturel, voire atavique. 
%Bien que le concept d'œuvre ouverte peut inclure ces concepts à différents niveaux, le changement de paradigme s'opère sur notre compréhension du monde sonore en musique.
%
%\bigskip

%Pour la petite histoire, ce concept s'est construit sur une idée ... aléatoire controlé ...  écriture musicale versus free jazz
%
%il faut distinguer l'oeuvre ouverte  des forme mobiles, aleatoire controle, stochastique,

%Ainsi, dans l'art sonore, certains compositeurs ont tenté avec plus ou moins de succès une reconstruction interactive avec les interprètes et parfois même avec le public.

%L'enregistrement sonore va permettre des techniques de transf
%\bigskip\bigskip
%   [à préciser -- de quelles tentatives il est question, etc.]
%\bigskip
%
%  
%Bien que des tentatives en ce sens ont déjà été réalisé dans le cadre pédagogique pour les écoles de musique, grâce notamment aux initiatives de Jean-Yves Bosseur consignées dans l'un de ses livres textes-partitions \textit{Le temps de le prendre}, ce mode de pratique musicale n'a pas pu transcender nos habitudes culturelles\endnote{Ces habitudes sont ancrés en nous selon un mode atavique systémique de soumission pour lequel il est difficile, dans un premier temps d'en avoir conscience, et dans un second temps, de s'en émanciper.} tant le lobby de l'industrie musicale sous tout ses aspects privilégie l'attitude mercantile du fait musical. 

%\bigskip
%
%Paradoxalement, ces concepts s'adressent à une classe d'avant-garde 
%

\section{\textit{Ni pour, ni contre, bien au contraire}}

Concernant ce \textsc{xx}\ieme ~siècle, il y eut deux ruptures, plus radicales et plus révolutionnaires que la déconstruction du modèle traditionnelle, %notamment par les expressionnistes, 
qui s'est traduit dans un premier temps par deux courants antagoniste remettant en cause les conventions et les contraintes idéologiques, esthétiques et politiques de leur époque, que furent les futuristes %\endnote{} 
et les dadaïstes\endnote{Le 5 février 1915, Hugo Ball et Richard Huelsenbeck rédigent le premier Manifeste dada sous forme de tract:  « \textit{Nous ne sommes pas assez naïfs pour croire dans le progrès. Nous ne nous occupons, avec amusement, que de l’aujourd’hui. Nous voulons être des mystiques du détail, des taraudeurs et des clairvoyants, des anti-conceptionnistes et des râleurs littéraires. Nous voulons supprimer le désir pour toute forme de beauté, de culture, de poésie, pour tout raffinement intellectuel, toute forme de goût, socialisme, altruisme et synonymisme.} »} autour de la première guerre mondiale. Les premiers promurent le progrès telle une démarche avant-gardistes mêlant urbanisme, machines et vitesse, tandis que les seconds contestèrent l'ordre établie d'une civilisation absurde par la dérision et la liberté absolue. 

Et dans un second temps, après la seconde guerre mondiale, certains compositeurs s'interrogent et interrogent l'auditeur et l'auditoire sur la manière de percevoir le monde sonore, dans un contexte de recherches bénéficiant de l'émergence d'une transdisciplinarité, notamment autour de l'intelligence artificielle, regroupant les sciences dites cognitives, incluant la psychoacoustique, la philosophie en termes de phénoménologie et d'ontologie, l'ethnomusicologie et les neurosciences, pour ne citer que ceux-là.

\subsection{Manifeste}

Autant, dans les années qui précédèrent la première guerre mondiale, les futuristes `surfaient' sur la vague du progrès, en tant que mouvement d'avant-garde adulant la vitesse sous toutes ses formes, refusant l'héritage du passé et tourné vers l'avenir, prêchant une réogarnisation du monde autour d'idéaux tels que la fusion de l’homme avec la machine, le fantasme d'immortalité par la technologie, avec la préfiguration de la cybernétique. Autant les bruits générés par les machines dans l'espace urbain et industriel suscitèrent l'intérêt d'un certain Luigi Russolo, qui en 1913 publia le manifeste futuriste \textit{L'Art des bruits}, %(\textit{L'Arte dei rumori}), 
dans lequel il théorise une façon de composer avec ce qu'il appelle les sons-bruits. 
Autant, durant la première guerre mondiale, les dadaïstes promurent l'anti-art sous toutes ses formes. Ces derniers rejettent alors toutes règles -- une sorte de \textit{tabula rasa} -- et ouvrent un espace de liberté totale, pourvu qu'elle transcende ou critique toutes les contraintes, toutes les lois, toutes les conventions, instaurées notamment par l'avant-garde et l'académisme. 

%\bigskip

Retenons que ces mouvements étaient fortement politisés et provocateur, flirtant avec le fascisme pour les uns, et l'anarchie pour les autres, voire le communisme pour le reste. %, ils étaient tous révolutionnaires. 
Ces mouvements se sont revendiqués comme un art de vivre, et de nombreux émules ont prolongés l'idée artistique première et l'ont pour certain réactualisés dans la seconde moitié de ce \textsc{xx}\ieme ~siècle, en tant que mouvement autonome tel que Fluxus par exemple (héritier de Marcel Duchamp et de John Cage)  -- précédé par le surréalisme\endnote{Entre poésie et arts plastiques, la musique surréaliste se définit plus comme une insurrection dadaïste, une totale liberté, qu'une exploration, une expression, de l'inconscient ou onirique.} de l'entre deux guerres -- ou en tant que démarche artistique avec notamment la musique concrète. Cent ans après, l'héritage musicale futuriste est omniprésent chez la plupart des compositeurs, bien que certains se revendiquent volontiers du dadaïsme dans l'esprit, l'un n'empêchant pas l'autre. 


%Musicalement, cela s'est traduit pour les uns, avec le manifeste futuriste de Luigi Russolo en 1913 \textit{L'art des bruits}, et pour les autres avec le manifeste littéraire de Hugo Ball et Richard Huelsenbeck en 1915 sous forme de track \textit{Dada soulève tout}\endnote{Dada = \textit{Nous ne sommes pas assez naïfs pour croire dans le progrès. Nous ne nous occupons, avec amusement, que de l’aujourd’hui. Nous voulons être des mystiques du détail, des taraudeurs et des clairvoyants, des anti-conceptionnistes et des râleurs littéraires. Nous voulons supprimer le désir pour toute forme de beauté, de culture, de poésie, pour tout raffinement intellectuel, toute forme de goût, socialisme, altruisme et synonymisme.}}
%Le début du \textsc{xx}\ieme ~siècle marque deux courants contaistataire 

%Pour mieux comprendre les tenants et les aboutissants de cette démarche, revisitons le cheminement historique à partir du début du \textsc{xx}\ieme ~siècle et tout au long de ce siècle.  

%\subsubsection{Les futuristes}
%\subsubsection{Les dadaïstes}

\subsection{Discontinuité}

%\subsubsection{\textsl{Moment form}}

Dès les années 50, tandis que des chercheurs s'intéressent et tentent de comprendre comment nous percevons la musique, grâce notamment au développement du cognitivisme et, plus tard dans les années 60, des neurosciences, certains compositeurs commencent à s'interroger sur la perception même du son, et à s'émanciper de la musique `expressive', de l'objet sémantique. En effet, pour ces compositeurs, il s'agit de diriger, de focaliser l'écoute sur le (ou un) son ou l'objet sonore (ou un objet sonore) en tant que tel et dans son atemporalité -- qui peut être défini comme la focalisation perceptive d'une durée infinitésimale, dans le but de tendre vers l'éternité du moment présent -- ou autrement dit sans mémoire. D'aucuns parleront d'immersion sonore. Dans tous les cas, cela relève d'un acte volontaire.

%notamment Karlheinz Stochausen 
%
%. Cet aspect sera développé sous différent approche telle que la \textsl{moment form} et la `musique expérimentale'. Il s'agit de diriger, de focaliser l'écoute sur le son ou l'objet sonore en tant que tel et dans sa temporalité.  
%
%L'activité cognitive induite est encore un sujet d'étude et de recherche
%
%La temporalité en musique est un phénomène lié à la manière de rompre une relative continuité

La \textsl{moment form} -- idée développée par Karlheinz Stockhausen et étroitement liée à l'idée de forme mobile  -- et la `musique expérimentale' -- notamment selon l'approche philosophique de John Cage qui considère la musique comme un phénomène naturel avant d'être l'objet d'un démiurge -- proposent ainsi des pistes de prospection sur l'acte de percevoir (de) la musique autrement. Ceux-ci intègrent un paradigme d'écoute qui consiste à considéré l'objet sonore pour ce qu'il est, dans sa nature ontologique, son essence, sans lien aucun %-- du moins si l'auditeur ne focalise pas son attention dans cette direction -- 
avec ce qui peut le suivre et ce qui l'a précédé. 
En d'autres mots, il s'agit de considérer une sériation d'objets sonores ou de moments %dont chaque section est
% \textit{moment form} --, lesquels répondent à la définition de 
%%\textsl{In other words, the} \textit{moment form} \textsl{is a seriation of moment defined as 
autonome en tant qu'unité musicale. %et se distingue des autres sections dans une logique temporelle discontinue.  
%\textsl{The idea of autonomous musical units is, in order to reduce the perception toward the smallest slice of time, to tend toward eternity}. 
L'idée d'unité musicale autonome entre dans le cadre de la théorie de la \textit{gestalt}\endnote{La théorie de la \textit{gestalt} est l'ensemble des processus de l'esprit et des sens mis en œuvre sur notre relation avec le monde, sur sa représentation. Elle tend à simplifier ou à optimiser notre perception, non pas par une accumulation d'éléments groupés ou discriminés reliant cette perception, mais par la recherche d'une forme simple ou connue liée à cette perception. Ces processus agissent en temps, et par conséquent sollicitent la mémoire sur ses différents plans simultanément, impliquant la mémoire à court terme, à long terme, déclarative, non déclarative, sensorielle, sémantique, etc. }\label{efn:gestalt}. %rend compte de la focalisation perceptive d'une durée infinitésimale, afin de tendre vers l'éternité du moment présent.
La forme est alors de type mobile, tant que les moments ne sont pas reliés entre eux ou peuvent l'être suivant une esthétique compositonelle prédéfinit et bien entendu de façon non hiérarchique, et intègre de la sorte un potentiel de \textsl{`never-ending process'}.
Pour la `musique expérimentale', le concept vise un résultat non prévisible \textit{a priori} et non déterminé pour le moins, dans la mesure où l'exécution est composée en place et lieu d'une composition exécutée. %\subsubsection{Musiques `expérimentales'}

Dans une modalité d'écoute, le silence prend une part privilégiée. 
Le silence en musique est un élément essentiel, qui dans le contexte de discontinuité s'inscrit souvent comme un marqueur séparant deux unités musicales. Aussi, il est l'alpha et l'omega d'une entité sonore ou musicale. Le silence peut alors revêtir plusieurs caractères, suivant sa durée, du micro-silence comme modalité d'articulation au silence absolu\endnote{Le silence dans sa durée et dans son principe revêt un caractère paradoxal car il suppose être, dans l'absolu, l'absence de propagation d'ondes sonores ou de vibration du tympan. Ce qui n'est qu'une vue de l'esprit. %hors d'atteinte car nous ne pouvons nous soustraire au propre son que notre corps produit)
Donc, de manière relative, le silence est l'indicateur d'une information nulle. 
Mais replacé dans son contexte, dans son usage courant, l'expérience subjective du silence prend une dimension autre et peut apparaitre comme silence d'apaisement, de détente, de répis ou silence inquiétant, éprouvant, oppressant; \textit{silence de ceux qui n'ont rien à se dire ou silence de ceux qui se comprennent sans un mot.} Dans le contexte musicale, c'est une respiration, un soupir, une pause, une suspension. Ou bien alors \textit{l'ensemble des sons non voulus}, selon le point de vue de l'inévitable John Cage pour ne pas le nommer.}, son placement en temps, et le contexte environnemental. Il serait vain céans d'en décrire tous les aspects, mais il est possible d'en distinguer au moins trois classes. Le silence de prolongation, celui qui fait résonance avec, et résonner, l'espace en termes de réverbération, comme la queue d'une comète représentant le (ou les) son(s) qui précéde(nt) cet état, et qui persiste(nt) dans notre esprit au delà de son effectivité dans l'espace partagé avec l'auditeur. % de son(leur) existence. 
Le silence de préparation, il est l'inéluctable catastrophe thomienne\endnote{Selon l'idée aristotélicienne d'une stabilité structurelle sous-jacente à toute qualité, l'avènement d'un conflit -- lutte entre au moins deux attracteurs par exemple -- provoquant une rupture, un saut qualitatif, constitue une catastrophe telle que René Thom l'a formalisé et modélisé en 1972 en termes de topologie et de géométrie différentielles, et qui peut être compris plus largement en termes phénoménologiques.}\label{efn:thom}, une tension qui doit se résoudre et qui se résout. Et enfin le silence `vide', qui se \textit{détache du silence de prolongation et ne présage aucune résolution}. C'est le silence cagien, celui qui n'a ni passé, ni futur, celui qui n'est pas vraiment le silence, mais une plongée dans l'imperceptible, le subtil. Il est le présent.
On pourrait ajouter le silence intermédiaire ou le silence élastique, c'est à dire une sorte de fondu/enchainé sans rupture -- conditionné par le contexte formel -- entre le silence de prolongation et le silence de résolution.

%Quant à la `musique expérimentale', principalement définit et incarnée par John cage, le concept vise un résultat non prévisible \textit{a priori} et non déterminé pour le moins, dans la mesure où l'exécution est composée en place et lieu d'une composition exécutée. 
Le lien qui existe avec la \textit{moment form}, la `musique expérimentale' et l'œuvre ouverte, consiste à exercer, à diversifié, à transcender nos modalités d'écoute, conditionnées par notre héritage culturel, voire atavique et nos croyances; ou plus simplement à considérer le moment présent dans l'attention du monde vibratoire, avec tous nos sens et toute notre intelligence.



%Ceci constitue une des raisons pour laquelle cela ne peut se faire qu'en dehors des institutions.

%\bigskip

%\textit{Cette approche est liée à une écoute, aussi affinée que possible, du monde sonore qui nous entoure, à ce que peuvent impliquer des échanges musicaux à l'intérieur d'une collectivité lorsque la musique n'est plus seulement propulsée comme produit à consommer, ou conçue comme recherche de laboratoire (ce qui représente, pour beaucoup, l'image de marque de la musique dite contemporaine), mais plutôt comme processus de sensibilisation et de questionnement, modelable selon les nécessités et facultés d'invention de chacun.} J.-Y. \textsc{Bosseur}, \textit{Le temps de le prendre}, Paris, Éditions Kimé, 1997, p. 202.

%\bigskip
%Malheureusement cela n'a pas fait consensus dans le temps et dans l'espace, si bien qu'aujourd'hui il n'en reste qu'une anecdote -- dans le meilleur des cas -- dans la littérature consacrée à l'histoire de la musique du XX\up{e} siècle.
%\bigskip\bigskip\bigskip
%   [à reformuler...]
%
%%  
%Pour revenir à notre propos, il va sans dire qu'il ne s'agit pas de remettre en cause des pratiques musicales s'inscrivant dans la tradition, mais plutôt de réinventez un monde à partir de nos acquis dans une perspective de recherche continuelle et sans cesse renouvelée de notre monde sonore impliquant les forces vives de chacun.
%
%Il est alors question ici de se repositionner par rapport à ces tentatives, de modéliser une pratique issue de ces expériences, qui n'ont en aucun cas était des échecs, mais plutôt une avant-garde que l'histoire nous pousse à réitérer aujourd'hui,
%\bigskip

%    
  

\section{Qu’est-ce que la musique?}

Essentiellement, avant d'être de la musique, le phénomène sonore est la perception de significatives fluctuations moléculaires dans l'air (ou dans l'eau pour les amateurs de plongée sous-marine) en termes de pression/dépression par l'appareil auditif, mais aussi ressenti physiquement, par la peau, les organes, impliquant le corps dans sa totalité, notamment par un effet de résonance sur le squelette. L'interprétation de cette perception relève de l'affect, corrélé à la cognition dans sa transversalité.
% cf le corps et l'esprit selon António Damásio dans L'Erreur de Descartes 

Pour autant que je sache, il n'existe pas de définition définitive au concept de musique\endnote{À noter que le concept de musique ou la musique telle que nous la pratiquons dans notre monde soit-disant civilisé, n'est pas aussi prétendument universel, puisque par exemple il n'existe pas de terme équivalent en langue \textsl{Inuit}, mais plutôt le terme générique \textit{nipi} pour le phénomène sonore en tant que tel. De même, pour la tribu des \textsl{Blackfoot}, seul le terme de \textit{saapup} réfère au chant, à la dance et à la cérémonie. En Afrique, les tribus \textsl{Tiv, Yoruba, Igbo, Efik, Birom, Hausa, Idoma, Eggon, Luo} et \textsl{Jarawa} n'ont tout simplement pas de mot pour musique. %\\ \indent De plus, l'universalisme tend à promouvoir l'égémonie imperialisme du système capitalisme qui tend à normaliser la notion de culture à l'encontre des plus en plus rares cultures éthniques
L'universalisme est un concept ethno-centré qui vise plus à démystifier -- voire à démythifier -- une pratique musicale, qu'à comprendre les véritables enjeux humains que la musique met en pratique. 
}$^{,}$\endnote{Cependant, une approche étymologique peut nous éclairer sur les intentions signifiées originelle. Ainsi, le terme musique vient du latin \textit{musica} et issue du grec ancien \textit{mousikè}, lesquels sont traduits laconiquement  par `ce qui concerne les muses'. Toutefois, bien que le terme \textit{mousa} signifie `muse', il peut aussi désigné la `science', le `chant' ou la `parole'. De plus, \textit{hè mousikè} ne désigne pas seulement l’art musical -- lequel se traduirait plutôt par \textit{mousikos akousma} -- mais l’ensemble des arts présidés par les muses. Pour finir, il existe une expression utilisant la même racine, \textit{mousikôs anèr} qui désigne une personne lettré et cultivé, instruit des arts que président les muses.\\ \indent Pour la petite histoire, dans la mythologie grecque, il existe deux versions concernant les muses -- pour chacune d'entre elles est associée un (ou des) art(s) ou une (ou des) connaissance(s) qui lui est (sont) propre --, déesses,  filles de Zeus et de Mnémosyne. La première version fait état de trois filles:
 \renewcommand{\labelenumi}{\arabic{enumi}.} \begin{enumerate}
\item Mélété (muse de la méditation et de l'exercice, de la pratique et du soin); 
\item Mnémé (muse la mémoire); et
\item Aoidé (muse du chant). 
\end{enumerate}
Quant à la seconde version, elles sont neuf:
 \begin{enumerate}
\item Calliope est la muse de la poésie épique et de l’éloquence [trompette];
\item Clio est la muse de l'histoire [trompette];
\item Erato est la muse de la poésie romantique %l'élégie 
[cithare, tambourin, lyre, viole];
\item Euterpe est la muse de la musique (inévitable tautologie) [piccolo double, flûte simple (aulos), hautbois double, trompette];
\item Melpomène est la muse du chant, de l’harmonie musicale et de la tragédie [corne, cor];
\item Polymnie est la muse de la poésie lyrique et de l’harmonie, de la danse, de la géométrie et de la rhétorique [orgue];
\item Thalie est la muse de la comédie [viole];
\item Terpsichore est la muse de la danse %et au chorus théâtral grec 
[lyre, viole, harpe]; et
\item Urania est la muse de l’astronomie et de l’astrologie mais aussi des mathématiques.
\end{enumerate}
À noter que la plupart d'entre elles sont souvent représentées avec un instrument de musique, entre autre attribut.}, mais plutôt un cluster polysémique nodal fluctuant\endnote{Cela mérite quelques éclaircissements. En effet, le concept doit se comprendre non pas comme une idée isolée, mais plutôt comme un réseaux de relations entre différent signifiés dans le but d'articuler une compréhension du monde qui se veut à la fois cohérente et fonctionnel dans une réalité, pour le moins, identifiée comme telle. Ce réseau, dans le meilleur des cas, peut se résumer à, ou converger en, un point nodal; le concept est clair. Mais souvent nous avons affaire à une multitude de points nodaux, que l'on nomme cluster et qui induit une  polysémie plus ou moins équivoque et naturellement corrélée  à un contexte. % -- que les linguistes nomment onomasiologie ou sémasiologie, selon le sens de la démarche analytique, respectivement du concept vers le signe ou du signe vers le concept. 
Dans certains cas, le concept se construit et évolue dans le temps passant d'un signifié à l'autre, sans jamais se fixer sur un point nodal déterminé tant la complexité du réseau impliqué l'en empêche. C'est le cas de la musique.}\label{efn:concept}, dans lequel chacun puise sa propre version, vision, souvent suivant le contexte référant, proposant un point de vue autre, parfois jusqu'à la contradiction\endnote{Une énumération, exhaustive ou non, n'est pas le propos qui nous intéresse, cependant j'invite le lecteur à consulter les différents dictionnaires de musique, de Jean-Jacques Rousseau à \textit{The New Grove Dictionary of Music} par exemple, les dictionnaires, encyclopédiques ou non, à différentes époques, et à se poser la question, et aussi à la poser autour de soi, pour se rendre compte de la diversité acceptionnelle que revêt notre compréhension de ce que pourrait, ou devrait, être la musique.}$^{,}$\endnote{Le son et la musique peuvent aussi bien être détournés à des fins nuisibles telles que la torture, le contrôle de l'individu en l'incapacitant ou en semant la confusion des sens, incluant les infrasons et les ultrasons; ou à des fins mercantiles destinées à rassurer l'utilisateur du lieu ou à stimuler le client potentiel.}.


Cependant, il y a quelques fondamentaux sur lesquels il est possible d'établir un point de vue panoramique. Ainsi, la musique est l'appréciation, l'entendement, du son, tel que tous les sons sont potentiellement de la musique ou musicaux. Aussi, la musique est un phénomène sonore incarné, soit une dimension spirituelle au sens initiation, rituel ou développement personnel, et compris dans une `mise en scène' par et pour un agent humain. 
%Elle n'est pas porteuse de sens connu ou reconnu en dehors du conditionnent culturelle en termes symboliques -- même s'il peut y avoir quelques universaux tels que: les sons graves sont telluriques et les sons aigus aériens --, notamment par le cinéma. Au plus, la musique parle d'elle-même autour d'archétype ou de stéréotype compris dans une culture donnée.
Autrement dit, la musique est une construction de l'esprit\endnote{\textsl{However, besides the art music, some musicologists consider the secular and religious nature of music evoked earlier, the arousal state induced by it like trance, and also the formal aspect also evoked in this writing as ritual -- more or less common for all known societies, according to an ethnographic corpus, like for instance the} \textit{ostinato} \textsl{promising the trance, attractor(s) in harmony (see modality and tonality) to gratify ear or to seduce audience  --, and this interpreted such as lullaby, love, dance, or healing music. But these considerations induce and remain a mind construction clearly in the frame of human ethology on societal manner based, and as asymptotic concepts from a western ethnological point of view.}}.

%à ce que serait la musique en tant que concept universel. Pour John Cage every thing is music et pour Victor Hugo la musique, c'est du bruit qui pense.
\bigskip

Partant de ces postulats, la musique ne peut être qu'un objet complexe dans son approche analytique, et qu'une approximation dans son approche synthétique, d'autant plus qu'à cela s'ajoute l'intention -- quelle qu'elle soit, délibérée ou non, singulière ou plurielle -- de l'émetteur et du récepteur. 

Vous l'aurez compris, la musique ne peut se résumer au système tonal et tempéré, à une mélodie que l'on siffle ou que l'on chante sous la douche, ou à une pulsation `cardio-émotive'. Encore moins à un object cernable tant notre compréhension du monde est partielle et partiale. 

%Qu’est-ce qu'un bruit?  \textit{Tout est bruit pour qui a peur.}
%  
%\bigskip
%
\section{Pensez la musique autrement}

Il y eut quelque tentatives de reconstruction créatives dans les années soixante, où la musique est expérimentée en tant que processus collectif et collaboratif, intégrant l'improvisation, tant sur la nature des sources sonores\endnote{La nature des sources sonores concerne aussi bien les instruments de musiques que des objets qui n'ont à priori rien à voir avec la musique, pouvant ou devant être détournés de leur fonction première, par exemple verser de l'eau dans un trombone, casser des vitres, le gargarisme, etc. De plus, la radio, l'enregistrement, la sonorisation et les sons de synthèse viennent s'ajouter à cet arsenal acoustique.} que sur la manière de construire, d'explorer un monde sonore avec d'autres. Dans une ambiance de lutte de classes, notamment contre le caractère élitiste et hiérarchique de la musique d'avant-garde, Alvin Curran -- co-fondateur de \textit{Musica Elettronica Viva} fondé en 1966 à Rome -- témoigne: « \textit{It was like discovering music for the first time... Each of us reveled in discovery of his own inner music -- the source itself. And `harmonizing' with the others became like a drug experience, demanding ever more fluency and intensity. From the beginning of these sessions, there was an awareness that we were dealing with something very serious – very fundamental.} »
%SOURCE -- http://www.alvincurran.com/writings/12%20years%20music%20rome.html
%CURRAN, Alvin, "A guided tour through 12 years of American Music in Rome",
%published in Soundings No. 10, Soundings Press, Santa Fe, 1976.

C'est ce `quelque chose' qu'il nous faut saisir, cette magie de l'instant, de l'inouï, de l'émotion partagée. Penser la musique autrement, c'est participer activement, pour soi et pour les autres, et aussi de partager ces moments d'oubli et de mémoire qui nous habitent, entre l'immanence et la transcendance de notre être.

%\bigskip
%
%scratch orchestra ...
%

\subsection{Concept} %--> object sonore}

Déjà évoqué en tant que possible `cluster polysémique nodal fluctuant'%\footref{efn:concept}
, le concept -- du latin \textit{conceptus} qui renvoie à l'action de contenir -- est la representation de l'esprit par un signifiant, articulé notamment par inférence selon la position qu’il occupe dans une chaine de signifiants, d'un signifié `gestaltique'%\footref{efn:gestalt}
, appliqué -- par extension à notre propre perception, et par abstraction -- à notre propre compréhension du monde. 
%La relation qu'entretiennent les notions de signifiant et de signifié, introduit en linguistique par Ferdinand de Saussure, et reprit en psychanalyse par Jacques Lacan, serait pour le premier de nature fixe, et apparait pour le second comme instable, et reconnait qu'un même signifiant peut se référer à plusieurs signifiés ou concepts différents selon le contexte.  \textit{Le signifiant ne signifie rien en soi, mais acquiert une signification en fonction de la position qu’il occupe dans une chaîne de signifiants.}
% SOURCE -- Serafino Malaguarnera in https://www.youtube.com/watch?v=CboOVK_cgNc&ab_channel=SerafinoMalaguarnera

En art, le concept est le caractère multiple ou reproductive d'une idée. Ainsi, l'idée prime sur sa représentation. \textit{Ce qui importe n’est ni l’objet proposé en lui-même, ni l’acte de proposition en lui-même, mais l’idée de cet acte}. Autrement dit, le concept est une représentation de l’esprit d’un objet, au sens philosophique du terme, par ce qui le caractérise.
% SOURCE -- Sophie Stévance, Les opérations musicales mentales de Duchamp. De la « musique en creux », 2009, page 5, §11. ---> http://imagesrevues.revues.org/375
Ainsi, l'art conceptuel s'approprie un champ d'exploration, en termes de formalisation et de modélisation, jusque là réservé aux philosophes et aux scientifiques, au delà de l'œuvre d'art en tant qu'objet unique, et au delà du savoir-faire de l'artiste, pour interroger les sens de celui qui en fait l'expérience.
Dès 1913, Marcel Duchamp, influencé et inspiré par le mouvement dada, remit en cause la notion même d'œuvre art, en détournant des objets industriels, se contentant de les titrer, de les signer, et de les exposer dans des lieux habituellement réservés à l'art. Ce fut la naissance du \textit{ready-made}, véritable art conceptuel avant l'heure. C'est la décontextualisation, puis la recontextualisation -- ou anachroniquement et plus précisément la déterritorialisation puis la reterritorialisation -- du signifiant pour un autre signifié, explicité ou laissé à l'appréciation -- voire l'imaginaire -- de l'observateur. 

Puis vint l'age d'or de l'art conceptuel. On pourrait en dater le début avec \textit{One and Three Chairs} de Joseph Kosuth en 1965 qui, partant du concept de la chaise, propose trois représentations distinctes, dans ce cas sans répétition formelle: \textit{l’ensemble dépasse ainsi l’objet pour approcher le concept} lui-même. 
% SOURCE -- https://blogacabdx.ac-bordeaux.fr/culture64/wp-content/uploads/sites/74/2019/12/EAC35.pdf
Puis la fin -- non de l'art conceptuel puisque désormais \textit{tout art après Duchamp est conceptuel}, mais de son age d'or -- avec \textit{An Oak Tree} de Michael Craig Martin en 1974, qui expose hors de portée du public un verre d'eau, en le désignant comme étant un chêne, détournant ainsi le concept par transsubstantiation. Le texte qui l'accompagne -- qui d'ailleurs fait partie intégrante de l'œuvre, quoique curieusement de manière anachronique -- est particulièrement éloquent à cet égard.  

Évidemment, les œuvres et les artistes précédemment cités ne seraient en aucun cas représentés à eux seuls l'art conceptuel de cette époque, mais le sont non seulement à titre d'exemples, d'illustrations, mais aussi, sont particulièrement significatives et significatifs dans leur radicalité et leur chronologie, pour y être effectivement mentionnés.


Maintenant, la musique, ou plus spécifiquement l'objet musical, en tant que concept, n'échappe pas à la règle. Au même titre que la représentation mentale du concept par l'objet ou par le signe, la musique conceptuelle est l'image sonore, une forme d'écoute, que met en place notre esprit par la pensée d'une idée -- à propos de la musique -- évoquée ou exprimée notamment par l'œuvre. %En ce sens, l’idée de l’œuvre d’art unique est éludée au profit 
En ce sens, nous pouvons parlé de musiques imaginaires ou imaginées par l'observateur et l'artiste. Comme nous scénographions notre lecture d'un livre, outre « lire entre les \hbox{lignes »}, nous inventons, nous décodons le médiateur musical, selon une sémantique du signe -- voire une sémiologie musicale\endnote{Pour autant, il n'y a pas de message dans l'art. Le signifié, au plus, suscite des impressions et des émotions, il dépend de la culture, de la sensibilité, du contexte spatial et temporel de l'expérience, de chacun.  De plus, le processus créateur du compositeur consiste principalement à communiquer une idée musicale -- qui peut être votre.
Certes, il peut y avoir quelques universaux -- notamment exploités par le cinéma -- tels que: les sons graves sont telluriques et les sons aigus aériens, mais cela ne nous dit rien sur le signifié lui-même.
%\textit{Mais l'art, ne délivre pas de message. « Signifier » en musique ne veut pas dire faire comprendre : l'œuvre musicale veut simplement communiquer une sensibilité personnelle présentée à nos sens mais accessible bien au-delà. Cette communication prend d'ailleurs chez certains compositeurs l’allure d’une intégration de plus en plus complexe de matériaux disparates.}
\\ \indent Pour mémoire, la sémiologie est l'étude des pratiques signifiantes ou d'un système signifiant. Cela concerne les symboles ou les signes destinés à évaluer, à transmettre ou à partager une information, un concept, telle que de la musique en ce qui nous intéresse. Dans ce cas, il nous faut inclure l'intention du compositeur selon des stéréotypes ou des archétypes, généralement culturels, et la compréhension que peut en faire le potentiel auditeur. La sémiologie consiste alors à comprendre un élément d'information. Autrement dit, ce qu'un signe, ou un ensemble de signes, provoque, ou peut provoquer dans notre esprit.}
 -- qui nous ai proposé. L'œuvre est donc ce qui provoque notre imaginaire, souvent dans le(s) domaine(s) qui lui(leur) est(sont) attribué(s). 

Présentement, ce n’est plus l’œuvre d’art qui crée le public, c’est le public qui crée l’œuvre d’art. Désormais, nous pouvons créer l'œuvre, non pour un public, mais pour nous élever par l'acte créatif que le concept nous permet d'expérimenter.

%se libérer de l’intentionnalité dans l’acte créateur, de la notion de goût par l’indifférence esthétique, ainsi que de l’idée de l’œuvre d’art unique qu’il soumet à la loi du « définitivement inachevé »
%L'idée musicale est le processus mis en place par le compositeur 

\bigskip

Penser la musique autrement, c'est prendre conscience que tout peut être musique\endnote{Le phénomène peut effectivement être considéré au delà de nos propre sens, tel que les infrasons et les ultrasons, la transposition de `chants' d'insectes, la sonification du fond diffus cosmologique (la sonification consiste à transposer, à adapter, à transformer, des mesures ou des données afin de les rendre perceptible dans le champ du sonore), etc; ou bien autonome tel que le \textit{soundscape} -- \textit{i.e.} les bruits de la nature, les bruits urbains, ... -- les musiques algorithmiques, dites aussi génératives ou procédurales, reposant sur une grammaire, une combinatoire, une modélisation, voire une fractalisation, et j'en oublie.}, dès lors que l'on se laisse porter par notre capacité à expérimenter, à interpréter, et interpeler nos sens. Elle est concept, et à ce titre nous pouvons en parler en termes de  sémasiologie et d'onomasiologie\endnote{La sémasiologie et l'onomasiologie sont deux termes plutôt rare que j'ai découvert au cours de mes lectures en linguistiques concernant la lexicologie, et qui pourraient s'appliquer lorsque l'on parle de concept en art. Ainsi, le premier rend compte de la signification d'un mot, voire d'une œuvre, en partant du(des) signe(s) vers le concept, tandis que le second, en partant d'un concept, étudie ses désignations, non seulement par le signe ou par le mot, voire par une œuvre, mais aussi par la grammaire et le contexte.}, avec l'intention d'aller au delà de l'expérience même, et de composer, d'inventer, et de transmettre.  
%    

%  \bigskip
% 
%\bigskip
%
%art classique (figuratif) -- art moderne (technique) -- art contemporain (récit) 
%
%modalité/tonalité -- dodécaphonisme/serialisme/minimaliste/spectral ... -- œuvre ouverte ...

\section{Objet sonore}

Bien que le concept d'objet sonore reste une tentative de formalisation\endnote{Difficile de ne pas cité l'incontournable \textit{Traité des objets musicaux} de Pierre Schaeffer publié en 1966, quant on parle d'objet sonore. Cependant, celui-ci se restreint à dessein qu'à la musique dite acousmatique, et malgré le caractère érudit de l'ouvrage, il subsiste quelques écueils, pour le moins problématique lorsque l'on parle de musique en général. Par exemple sur la notion d'objet `convenable' dont la définition reste pour le moins abstraite et ouverte. Aussi, la notion d'écoute réduite, qu'il qualifie de condition \textit{sine qua non} à tout objet sonore, consisterait à faire abstraction du message causal du son et de sa contextualisation, fruit de l'\textit{épochè} (sic) -- qu'il distingue de l'objet musical qui fait intervenir une référence culturelle -- pour en faire son analyse typologique. Il choisit la méthode par répétition -- reconnaissant la difficulté d'une telle démarche qui contribuerait à nous éloigner de toutes causalités -- induisant l'enregistrement de l'objet d'étude, confinant ce dernier dans la musique concrète.\\ \indent Rappelons que Pierre Schaeffer forgea ce terme de musique concrète en 1948, pour disparaitre dans les années soixante, remplacer par le terme plus générique de musique électroacoustique. La musique concrète consiste à faire de la musique empiriquement, \textit{i.e.} concrètement, à partir de sons enregistrés, sur vinyl puis sur bandes magnétiques, pour être manipulés, transformés, mixés, en une œuvre pour le moins inouï.}  -- voire de modélisation -- d'un phénomène sonore existant ou imaginaire, irréductible ou complexe, structurant et structuré, je me propose d'en identifier, d'en énoncer quelques axiomes ou éléments irréductibles comme point de départ vers une nomenclature, une exploration, qui se doit d'être propre à chaque formation. 
Ainsi, le principe qui régira toute approche performative sera basé de près ou de loin sur l'objet sonore en tant qu'unité, \textit{gestalt}, forme. 

Le concept d'objet sonore n'est pas clairement posé tant il dépend de son objet téléologique. Ici, il s'agit plus de construire cet objet téléologique à partir d'objets sonores que le contraire.   
Aussi, il me parait plus simple et plus pertinent d'aborder l'objet sonore par une analogie toute relative emprunt à la linguistique et plus spécifiquement à la phonologie -- voire la phonètique. Ainsi, il est possible de distinguer l'unité irréductible en tant que phonème, que l'on peut rebaptiser `sonème'. Puis, le morphème qui est composé de sonèmes pour former une unité `supérieur', d'ordre culturel, symbolique, musical. Enfin, la syntaxe, qui satisfait un ensemble de règles combinant, associant, articulant les sonèmes et les morphèmes, visant un effet, une image, un concept, entre différents sonèmes ou morphème, en `harmonie' (superposition de sonèmes/morphèmes) ou en temps (enchainement de sonèmes/morphèmes). À partir de là, les motifs et les phrases peuvent constituer un ordre supérieur d'objets sonores. De la micro-forme à la macro-forme, cela pourrait se résumer ainsi:
sonème $\rightarrow$ morphème $\rightarrow$ motif $\rightarrow$ phrase $\rightarrow$ section $\rightarrow$ œuvre. Ces termes pouvant bien entendu se recouvrir jusqu'à ne faire qu'un.


Ceci constituera notre définition de l'object sonore.
\bigskip

%il est possible d'identifier quelques axiome ou elements irreductibles ...

%\noindent \textbf
{Nous entendons tous, mais nous n'écoutons pas les même choses.}
%\smallskip
%\textit{La recherche d’une typologie « absolue » est illusoire.} Cependant, 
En tout état de cause, il est possible de décrire un objet sonore selon des caractéristiques utiles que l'on souhaite formaliser ou formuler dans le but de disposer d'un langage, et d'un objectif, commun au groupe impliqué. Loin de prétendre à une description absolue et exhaustive, je me contenterai d'énoncer quelques pistes, usant de la terminologie d'un compositeur, sans que cela soit gravé dans le marbre. 
\bigskip

De l'objet sonore:

\begin{longtable}{lrcccl}
\texttt{[1]} & durée courte & $\leftarrow$ & -- -- -- -- -- -- & $\rightarrow$ & durée longue \\
\texttt{[1]} & continu & $\leftarrow$ & -- -- -- -- -- -- & $\rightarrow$ & discontinu \\
\texttt{[2]} & \textit{quantum} & $\leftarrow$ & grain & $\rightarrow$ & lisse \\
\texttt{[2]} & régulier & $\leftarrow$ & -- -- -- -- -- -- & $\rightarrow$ & chaotique \\
\texttt{[2]} & son pur & $\leftarrow$ & son cannelé & $\rightarrow$ & bruit blanc \\
\texttt{[2]} & monophonie & $\leftarrow$ & -- -- -- -- -- -- & $\rightarrow$ & polyphonie \\
\texttt{[2]} & hauteurs définis & $\leftarrow$ & masse fixe & $\rightarrow$ & masse variable \\
%grave & $\leftarrow$ & -- -- -- -- -- -- & $\rightarrow$ & aigu \\
\texttt{[3]} & \textit{rallentendo} & $\leftarrow$ & \textit{a tempo} & $\rightarrow$ & \textit{accelerando} \\
\texttt{[4]} & hyper-localisation & $\leftarrow$ & -- -- -- -- -- -- & $\rightarrow$ & ubiquité \\
\texttt{[4]} & près & $\leftarrow$ & -- -- -- -- -- -- & $\rightarrow$ & loin \\
\texttt{[4]} & petit espace clos & $\leftarrow$ & -- -- -- -- -- -- & $\rightarrow$ & grand espace clos \\
\end{longtable}

\noindent {\small \texttt{[1]} unité, segment temporel, \texttt{[2]} qualité spectrale, densité de masse, \texttt{[3]} agogique, \texttt{[4]} spatialisation.}

%Les qualités intermédiaires sont des alternatives et non des passages obligés.

\bigskip

%\begin{description}
%\item[\texttt{[]}] \textbf{Bruit blanc}
%\begin{itemize}
%\item respiration, souffle, ...
%\item vent, pluie, ...
%\item moulin à café, papier de verre, ...
%\end{itemize}
%\item[\texttt{[]}] \textbf{Agogique macroscopique}
%\begin{itemize}
%\item \textit{accelerando}
%\item \textit{rallentendo}
%\end{itemize}
%
%\end{description}

De l'articulation:

\smallskip

\noindent Étroitement lié à la notion de dynamisme, 
de l'événement sonore irréductible à l'ensemble de l'œuvre,
la dynamisation du fait musical
s'inscrit dans un contexte expressif, discursif, voire
 contemplatif. 
 Les éléments dynamiques sont entendus selon leurs relations dans le temps, avec un avant et un après.
Ainsi, le découpage formel du flux musical est reconnu en temps par le mode d’articulation.
L'articulation se définit alors en termes d'accentuation, de transition, de ponctuation, et de caractère.
%voici quelques linéaments concernant la dynamisation du fait musical, en partant de l'événement sonore irréductible à l'ensemble de l'œuvre,
%Cela se traduit par la description des différents procédés dynamiques afin de les
%inscrit dans un contexte expressif, contemplatif, voire discursif. 
%Dans cet ordre d’idée, les éléments dynamiques peuvent aussi, le cas échéant, être classifiés selon leurs relations dans le temps, avec un avant et un après. Ainsi, le découpage du flux musical est reconnu en temps par le mode d’articulation.

%Cela se traduit par une taxinomie des modes d’accentuations, lesquels vont s’inscrire par extension à la phrase, au `bloc xénakien', autrement dit à l'objet sonore d'ordre supérieur,
 %par une structuration en temps -- `mélodique' -- des énoncés en groupes rythmiques 
 %et à la morphologie holistique de l’œuvre par la perception de la succession plus ou moins régulière des accents démarcatifs en terme de mouvement.

%Aussi, les blocs thématiques, pouvant être définit en mouvements identifiés par leur caractère et délimités par une catastrophe au sens thomien du terme, 
%peuvent s’inscrire dans ou être articuler selon une terminologie narrative telle que l’orientation, le climax, et la coda.

\begin{enumerate}
\item (a) \myuline{Accent} 

associé à un objet sonore en terme d’entité (sonème ou morphème).

\begin{itemize}
\item Par le mode d’attaque lié au mode d’excitation ou au mode de jeu.
\item Par l'allongement de l'objet sonore en termes de durée.
\item Par l’énergie déployée, exprimée le plus souvent en terme de nuance. 
\item Par le timbre résultant qui combine les points précédents en y incluant l'aspect vertical par la superposition de timbres.
\end{itemize}
Il va sans dire que tous ces points ne sont pas exclusif, et au contraire sont corrélés entre eux.
 \end{enumerate}
Les trois points suivants s’inscrivent dans un rapport inter-relationnel d’ordre supérieur impliquant au moins deux objets sonores en temps.
\begin{enumerate}%[resume]
\item (b) \myuline{Accent}
\begin{itemize}
\item En jouant sur la durée de l'objet sonore en termes d'anticipation ou de retard.
\item Par le registre, qui d’un point de vue structurel peut être lié à la tonalité ou à la modalité (au sens d’effet plus ou moins attractif d’un ensemble de sons relatif à un son unique) ou plus généralement lié à la tessiture de l'instrument, autrement dit à l’intonation.
\item Par un micro-silence, induit par les accents précédemment énumérés, ou délibéré.
\end{itemize}

En bref, les accents possibles dépendent de l'\textit{instrumentarium} et de leur rôle en tand que tel, c’est à dire si l’on a affaire à un accent démarcatif (marquant une `frontière' entre deux objets), à un accent d’insistance (mise en relief d'un élément constituant l'objet) ou un accent contrastif (mise en évidence d'une unité par rapport à une autre).

\item \myuline{Transition}

La transition est un objet sonore -- qui en présente par conséquent les mêmes caractéristiques --  dont la fonction est d'articuler, de relier, deux objets sonores en temps dans une logique discursive.

 \end{enumerate}
Quant aux deux derniers points, l'articulation se joue -- à l'instar d'un langage usant par définition de formes symboliques inventoriés et  d'une syntaxe de composition répétée à plusieurs niveaux et d'une complexité plus ou moins élaboré d'ordre culturelle -- sur une structuration d'ordre supérieur et concernent moins la musique elle-même que la manière de l'interpréter, ou plutôt devrais-je dire de l'articuler, selon une effective transmission orale ou écrite, voire suggérée ou laissée à l'appréciation des acteurs. 

\begin{enumerate}[resume]

\item \myuline{Ponctuation}

Étroitement lié au caractère,  emprunt d'un concept linguistiques, la ponctuation en musique concerne l'articulation prosodique en termes de respiration, d'intonation, et de cadence. Cela s'applique généralement au niveau du motif ou de la phrase au sein d'une section.


\item \myuline{Caractère}

Le caractère contextualise l'état d'esprit dans lequel l'œuvre musicale doit être jouée ou vécue. Au delà du caractère proprement dit -- qui peut être par exemple triste, gai, enjoué, agité, léger, etc. -- et en plus des différents modes d'articulation, la pulsation, le tempo, voire le mètre, participent aussi à la manifestation dudit caractère.

\end{enumerate}

%
%2 - ARTICULATION (gestion des attaques et des silences) et PONCTUATION (articulation prosodique) [2].
%
%2.1 - Modulation / Segmentation
%
%2.2 - Cadence / Inflexion
%
%2.3 - Respiration / Silence
%
%\bigskip
%

%\begin{itemize}
%\item attaque, accent;
%\item punctuation, cadence;
%\item transition;
%\item caractère, tempo.
%\end{itemize}

\bigskip

De la syntaxe:

\smallskip

 \noindent{Le principe d'organisation de la musique est fondé et repose sur la dialectique répétition/variation, comprise entre le silence et le contraste.}

%%\noindent 
% silence $\leftarrow\rightarrow$
%répétition 
%%$\leftarrow\rightarrow$ micro-variation 
%$\leftarrow\rightarrow$ variation 
%%$\leftarrow\rightarrow$ zone d'indétermination 
%$\leftarrow\rightarrow$ contraste
%
%%\bigskip 
%
%\noindent \textbf{Chut! Écoutez, il y a quelques choses à entendre.}
%
%\smallskip

\subsection{Le phénomène émergent}
 %L'émergence est un terme relativement nouveau, et comme pour le concept, souvent mal compris, il est sujet à des interprétations plus ou moins fantaisistes, au gré de l'air du temps. Par conséquent, une définition s'impose, afin d'intégrer ce phénomène dans notre expérience sensorielle, tant dans une possible prédictibilité selon notre acquis, que dans son effectivité.
 
 L'émergence est une caractéristique d'un système en tant qu'unité, induit par l'interaction de ses éléments le constituant et avec son environnement. Par définition, cette caractéristique ne peut être déduite à partir de ses constituants, soit parce que le système a atteint un degré de complexité ne permettant pas d'en prédire le comportement dans son ensemble autrement que par une évaluation empirique, soit parce que cette caractéristique est une propriété structurelle qui dépend de la façon dont le système est agencé. L'un étant corrélé à l'autre.
%  Vaste sujet que l'émergence, dont la musique est le phénomène émergent du monde sonore.
%  
%def ---> Le phénomène émergent est une propriété, une caractéristique d'un système complex, 
%...
%
%La musique en tant que système complex, répond aux principes d'émergence. 
%
%phénomène d’emergence dans la superposition d’objet sonore
%
%fusion/fission

%interprétation/composition -- phénomène d'incarnation (empreinte d'un agent humain au sens écoutez, il y a quelques choses à entendre)

La phénoménologie de l’émergence en musique résulte de deux systèmes pour le moins complexes, tels que la sensibilité auditive, soit `l'encodage perceptif'\endnote{Dans les grandes lignes, `l'encodage' perceptif est le médiateur entre les ondes sonores en tant que variation dépressionnaire du milieu provoqué par une significative perturbation mécanique et le traitement de l'information encodée dans le cerveau. Ce médiateur est -- vous l'aurez deviné -- l'oreille. Celle-ci présente la particularité -- notamment dû au canal auditif externe -- d'être plus sensible autour de la fréquence de 3\,500 Hertz. Cette sensibilité décroit dans les graves et les aigus selon les courbes isosoniques (niveau sonore perçu en fonction de la fréquence) qui rendent compte de manière empirique de la sensibilité du système auditif humain, car cette sensibilité n'est pas linéaire et varie d'un individu à l'autre, notamment en fonction de l'age et de l'exposition à risque endurée. De plus, il existe une bande de fréquence, appelé bande critique, dans laquelle une fréquence peut perdre de son intensité jusqu'à devenir inaudible lorsqu'une fréquence voisine dépasse un certain seuil d'intensité -- communément appelé effet de masque -- (à noter que la largeur de bande augmente avec la fréquence `masqueuse' et avec son intensité), ou bien perçue au dessus de ce seuil en termes de battement ou de rugosité (ce qui constitue en soi un phénomène émergent) -- sachant que la rugosité concerne la modulation de fréquence ou d'amplitude avec un effet maximal aux alentours de 70 Hertz, et avec un effet de battements pour les fréquences basses. Ainsi, le signal est décomposé dans la cochlée pour y être converti en un signal électrique transmit dans le cerveau par le nerf cochléaire composé d'environ 30\,000 axones.}, et la psychologie du sujet; ou autrement dit, le son comme porteur d'information potentiellement utile au regard de la cognition du sujet pour le traitement de cette information. 

Concernant les processus mis en œuvre par le cerveau pour traiter l'information auditive, ceux-ci sont déterminés par deux modes d'apprentissage que l'on peut distinguer par l'acquis et par l'inné -- ou respectivement supervisé et non-supervisé selon la terminologie des modèles informatiques en intelligence artificielle. Le premier est un apprentissage qui conditionne une téléologie sociale et culturelle, tandis que le second se fait de manière implicite, sans que l'on en ai conscience, selon des récurrences structurelles de notre expérience avec l'environnement tel qui nous ai rapporté par nos perceptions sensorielles, et selon notre architecture cognitive. Cela se traduit pour l'un, de perpétuer et d'assimiler une connaissance, une pratique; et pour l'autre, de s'intégrer et de s'adapter dans un environnement qui nous ai donné d'expérimenter selon nos propres capacités en tant qu'être humain.

Pour ce qu'il est de l'apprentissage implicite, 
%$bien que cela reste un sujet d'étude en psychoacoustique et en cognition, 

%---------
%
%Outre le fait que l'activité musicale sollicite les ressources du cerveau dans son entier, en corrélant notamment le son à l'image pour le plus évident, mais aussi à tous nos sens, ainsi que nos connaissances acquises (notamment abstraites et symboliques),  il existe un schéma fonctionnel et structurel qui nous permet de comprendre ces deux modes d'apprentissages. 
%
%...
%
%Au niveau de la perception, un phenomene emergent consiste à percevoir un son dans un contexte sonore...
%
% D'abord, il s'agit d'identifier et de circonscrire l'object sonore en tant qu'unité irréductible selon des indices acoustiques tels que la régularité spectrale, le destin commun, et la contextualisation. La régularité spectrale est l'empreinte auditive ou plus simplement le timbre de la voix, d'un instrument, ou d'un son connu ou à connaitre. Le plus souvent, l'attaque ou le mordant, c'est à dire la courte période permettant la mise en vibration de l'air par l'instrument jusqu'à cette régularité spectrale, et la brillance, caractérisée principalement par le centre de gravité spectral, participent à la discrimination et à la reconnaissance de la source et de ses caractéristiques, permettant des groupements distincts et le traitement de structures temporelles. Le destin commun ...
%
%%..., l'audition musicale est un processus d'apprentissage implicite, c'est à dire que l'acquisition de connaissance de notre environnement via nos sens
%
%Pour ma part, je distingue deux modalités émergentes; à savoir la \textsl{fission}, qui est la capacité à distinguer un son spécifique de l'ensemble, et la \textsl{fusion}, comme le son résultant d'un complexe sonore en termes de texture ou de variation périodique ou continue.
%
%\bigskip
%
%%en psychoacoustique il y a trois phénomène induit...
%%La compréhension et la prévisibilité du phénomène émergent requièrent quelques connaissances psychoacoustique telles que `l'encodage' perceptif. Celui-ci présente deux caractéristiques que sont le `filtrage' qui amplifie certaines fréquences et qui en réduit d'autres, et ...
%
%
%timbre
%
%hauteur musicale
%
%hauteur virtuelle ---> périodicité lié à la régularité spectrale 
%
%hauteur spectrale ---> brillance estimé par le centroîd
%
%ambiguité voire illusion auditives entre hauteur virtuelle et hauteur spectrale
%
%indice acoustique
%-- regularité spectrale (timbre de la voie ou de l'instrument ou d'un son connu)
%-- destin commun --> probable source commune
%mecanisme cognitifs interagissant
%contextualisation (effet de contexte -- semantique, symbolique, ... -- completion d'information partielle par inference bayesienne dans le principe), sollicitation multi-sensorielle notamment visuelle
%
%décryptage par la probabilité d'une séquence, autrement dit de ce qui peut ou va suivre plus indice visuel (mouvement de la bouche pour le chant ou la parole, instrumentiste ou le faiseur de son en action)
%
%---> apprentissage et expérience
%
%+ Effet McGurk 
%
%Du reste, de manière générale l’oreille n’aime pas la monotonie et se focalise sur tout ce qui est
%non-stationnaire : vibrato, effet choral, transitoires, variations du spectre dans le temps.
%
%la brillance, caractérisée principalement par le centre de gravité spectral
%-- le mordant, caractérisée par la durée de l’attaque : de l’ordre de 1 à 2 ms pour le piano, 50 ms
%pour le violon
%Brillance et mordant rendent compte de 85\% de l’information caractéristique des timbres.

\subsection{\textsl{The score}}

%Il convient de distinguer deux approches. L'une, spécifique aux compositeurs européens, s'inspire et s'exprime selon des recherches essentiellement littéraires, tandis que l'autre, concerne les compositeurs américains qui sont surtout animés par des recherches picturales.
Lorsque la magie opère, il est naturel de vouloir la reproduire, pour nous mêmes, pour nos contemporains, voire pour les générations futures. La solution la plus évidente qui nous vient à l'esprit est bien entendu l'enregistrement. Quoique ce dernier requiert une logistique matériel pour le moins significative, avec une pérennité du support pour le moins discutable. Cependant, cela reste un instantané, qui dans le meilleur des cas, devient l'œuvre, tandis qu'il reste un ersatz de l'expérience vécue dans les autres cas. La partition -- \textsl{the score} -- s'avère alors nécessaire, voire indispensable, pour reproduire un effet voulue. Cet effet est le concept.

Bien que le partage et la transmission peuvent se faire de manière orale, induisant une proximité en temps et lieu des acteurs, la partition ouvre des perspectives conceptuelles et d'élaboration, insoupçonnées autrement. D'ailleurs, en tant que concept, la partition est bien souvent l'œuvre.  


Le concept d'œuvre ouverte est spécifique aux compositeurs européens, qui sont inspirés par des recherches essentiellement littéraires (alors que les compositeurs américains sont surtout influencés par des recherches picturales).

Autant il y eut une volonté de standartisation initié en Europe durant la renaissance jusqu'au romantisme, encore en vigueur dans l'enseignement musicale, autant aujourd'hui, la disparité notationnelle tend vers une  specialisation par type, par compositeur, voire par œuvre.
D'une extreme à l'autre, il est difficile d'imaginer un universalisme applicable dans ce domaine, et il est tout autant difficile de deffricher, de decrypter, d'envisager une notation plus ou moins esoterique ou pragmatique, repondant a une situation historique, social ou politique sans entrer dans un savoir pluridisciplinaire qui dépasse innaccecssible a l'echelle humaine. L'idée consiste à trouver un compromis entre l'individu et la masse. Compromis largement compris et pratiquer par les societe tribal, pour lesquel il serait bon de s'inspirer que d'interpreter. il s'agit moins de comprendre que d'etablir une relation digne, respectueuse, et harmonieuse au sens etymologique du terme, c'est a dire du grec (armozo), qui veut dire joindre, faire coincider, adapter, emboiter. (Du point de vue philosophique, en particulier dans la Grèce antique, on peut considérer l'harmonie comme le fait pour tous les éléments d'un tout d'être à la place qui leur est destinée, de telle sorte que le tout est meilleur que la somme des parties. L'harmonie est ainsi une propriété structurelle de ce tout. wikipedia)

 Si la science et la philosophie ont la vérité comme objet visé, la grammatologie
voit en revanche dans l’écriture un phénomène sans vérité. L’écriture est cela même qui
déstabilise la notion de vérité depuis son origine.

Représentation symbolique d'une situation musicale, la partition est à la fois ...

%     Synchroniquement, quelques artistes conceptualisent la musique en termes d’idées formalisables dont la résultante musicale est imaginée ou interprétée hic et nunc, impliquant la propre expérience du sujet.

%Parmi ces artistes, il convient de citer :
%\begin{itemize}
%\item 1913 – L’arte dei Rumori – Luigi Russolo
%\item 1916 – Ready-made – Marcel Duchamp
%\item 1939 – Experimental Music – John Cage
%\end{itemize}

« Dans l'œuvre de Cage, l'acte de notation ne se referme pas sur la constitution d'un objet musical, mais agit plutôt comme le déclencheur d'une chaîne d'actions et de réactions sonores dont le compositeur ne cherche pas nécessairement à prévoir toutes les conséquences. » Jean-Yves Bosseur, \textit{Du son au signe. Histoire de la notation musicale}, Paris, Éditions Alternatives, 2005, p. 121.

Aussi, cette période se caractérise par la multiplication des systèmes d'écritures qui peut se résumer par:
\begin{itemize}
\item 1914 – Sports et divertissements – Erik Satie → ajout de texte;
\item 1961 – Available forms I – Earle Brown → forme mobile avec un paramétrage fixe;
\item 1961/62 – Volumina – György Ligeti → representation visuelle.
\end{itemize}

« Noter, ce n'est plus nécessairement indiquer une hauteur de son, un rythme ... noter, c'est aussi inventer une écriture. » op. cit. p. 133.
 
\section{\textsl{So what}}

un des concept le + important consiste à construire ou à reconstruire la musique que l'on souhaite faire en intégrant les différentes étapes comme autant de niveaux de pratique nécessaire à la réalisation de celle-ci. Les avantages sont que chaque personnes quelque soit le niveau peut intégrer l'œuvre et évoluer avec elle de façon à atteindre l'objectif `virtuose' ou musical avec le soutien et le concours de l'ensemble. Cela créé un dynamisme naturel et une réalisation de soi aussi plus naturelle et spontanée.

-----------------------------------------------------------------

Avant toutes choses, je voudrais revenir sur quelques notions telles que le concept et la culture afin d'apporter un nouvel éclairage sur une même vision vue par l'anthropologue Clifford Geertz. En effet, selon ses propres mots: 

 « \textit{Croyant, comme Max Weber, que l’homme est un animal pris dans les réseaux de signifiance qu’il a lui-même tissés, je considère la culture comme assimilable à une toile d’araignée, et par suite son analyse comme relevant non d’une science expérimentale en quête de loi mais d’une science interprétative en quête de sens.} »

Aussi, la culture est « \textit{un modèle de significations incarnées dans des symboles qui sont transmis à travers l'histoire, un système de conceptions héritées qui s'expriment symboliquement, et au moyen desquelles les hommes communiquent, perpétuent et développent leur connaissance de la vie et leurs attitudes devant elle} ». Quant le symbole représente « \textit{tout objet, acte, événement, propriété ou relation qui sert de véhicule à un concept} ».
% SOURCE L’anthropologie interprétative Clifford Geertz (1973) Bali, interprétation d’une culture, Gallimard. https://dun.unistra.fr/ipm/uoh/anthropologie/fichiers/doc_geertz_int_01_01_fr.pdf

\bigskip

... faire une passerelle vers le spirituel ...

\bigskip


%Ce qui ressort de cette aperçu historique est l'incroyable diversité de concevoir, de composer, d'écouter et de vivre le phénomène sonore que l'on appelle dans certaines circonstances musique. Évidemment, loin d'être exhaustif, ce résumé a pour but de donner des pistes d'investigations afin de découvrir, de redécouvrir ou d'approfondir certaine démarche et d'avoir une connaissance ou un savoir commun, si nous voulons ensemble construire ou vivre notre propre relation avec le monde vibratoire.
la concurrence et la competition est contre productive, car elle consiste la plupart du temps à disqualifier ou à décrédibiliser le ou les adversaires, si bien que toute hégémonie -- monopole -- dans ce contexte tend inévitablement vers le totalitarisme.

Toutes confrontations avec un système gérant des masses et non des individu\endnote{Bien que les société de masse peuvent dans certains cas privilégié l'individualité, c'est uniquement dans le but de diviser pour mieux régner, et en aucun cas de permettre ou de promouvoir l'émergence de mouvements alternatifs dans un but d'émancipation, ou la réalisation de soi.} restent stérile car antinomique.

fluxus derive totalitaire et commerciale qui a tue le mvt....

Cela dit, cette démarche était encadrée dans un context institutionnel permettant de promouvoir une pratique musicale \textit{hic}  et \textit{nunc} et une approche compositionnelle autre pour enfin créer des liens avec l'enseignement traditionnel. Bien que cela n'est pas fait consensus dans le temps et dans l'espace, Il en reste des traces de nos jours dans les ateliers dit d'éveil musical pratiqués dans certaines classes de musique au collège ou dans les écoles de musique. 



et de reduire l'acte artistique a une seule personne pour des raison commerciale, et pour legitimer une hierarchie omnipresente, devalorisant toutes activite subalterne, reduit a des executant que l'on appele professionnel pour verouiller le systeme.


\subsection{\textit{Praxis}} 

les repetitions ne sont pas une fin mais un moyen d'atteindre un objectif personnel ou collectif.
l'obligation de repetitions est une imposture au sens ou il s'agit plus de confiance en soi. Certes on peut acquerir cette confiance en soi par un forcing avec repetition, en forcant la nature et par consequent contre nature -- a moins que l'on considere la vie comme une conquete du plus fort, du plus performant, une competition favorisant les favorises, qui est malheureusement le credo de notre civilisation, et enteriner par le darwinisme primaire (au sens doxique du terme qui consiste a generaliser une idee formulee par Charles Darwin en 1859 dans son ouvrage a caractere scientifique l'Origine des espèces dont l'idee premiere relate d'une selection naturelle et graduelle induite par une concurrence dite vitale dans le processus de l'evolution des especes. Ce qui peut en constituer une possible cause -- quoique le terme d'evolution me parait plutot excessif quant il s'agit seulement d'une adaptation (bien qu'il existe des organismes qui se reproduisent mecaniquement sans dessein (\textit{i.e.} gene egoiste) ou qui ne possedent pas de lien genealogique (\textit{i.e.} gene orphelin), et evoluer vers quoi? certainement pas en terme de complexité organique, a moins que l'on considere l'espece humaine comme l'aboutissement de l'evolution, ce qui implique un dessein teleologique plus ideologique et anthropocentré que scientifique ou humain -- mais en aucun cas la seule. Par exemple l'idee ou le concept de niche, qui induit le phenomene d'harmonie avec l'environnement selon les limites ou les capacites du sujet en tant qu'entité biologique, semble plus "conforme" a la realite, notament archeologique (\textit{i.e.} la discontinuite du registre fossile), bien que ces differentes approches ne soient en aucun cas exclusives.) -- alors qu'il suffit de pratiquer avec nos proches, et la confiance, notre evolution, notre developpement, tant en pratique que spirituel, ira de soi, porter par la dynamique bienveillante du groupe, ce qui peut aussi etre vrai dans les relations inter-groupe.

L’obligation de produire aliène la passion de créer. Le travail productif relève des procédés de maintien de l’ordre.
Raoul Vaneigem

C’est autant de pistes qui nous appartient de suivre selon la situation, de nous en inspirer selon notre humeur, et d’explorer au delà des sentier battu ce que le monde sonore peut nous offrir pour notre plus grand plaisir. 

\bigskip

Maintenant, il est possible d'appréhender la musique, dans une continuité dite de tradition -- perpétuation d'une codification --, ou bien dans une logique de rupture -- recherche ou émergence de nouveaux paradigmes. Bien entendu, la rupture se comprend selon la catastrophe thomienne%\footref{efn:thom}%\endnote{Selon l'idée aristotélicienne d'une stabilité structurelle sous-jacente à toute qualité, l'avènement d'un conflit -- lutte entre au moins deux attracteurs par exemple -- provoquant une rupture, un saut qualitatif, constitue une catastrophe telle que René Thom l'a formalisé et modélisé en 1972 en termes de topologie et de géométrie différentielles, et qui peut être compris plus largement en termes de phénoménologie.}
, dans une sorte de \textit{continuum}, en tant que de structure sous-jacente, pour le moins en temps. 

certes, tout le monde n’est pas compositeur ou musicien tel que nous le concevons, cependant, la participation active, dans le processus même et/ou dans l’interprétation de l’objet sonore doit être accessible à tous 



%On peut aussi l'appréhender respectivement comme acte d'expression ou de créativité, lesquels ne sont évidemment pas exclusifs.
\bigskip

À cela, j'ajouterai comme horizon le développement personnel et collectif, ce qui constitue, au delà de l'art pour l'art, ou au delà d'une posture artistique consciente, un pas vers l'émancipation du sujet vers sa réalisation, faisant partie d'un tout. Il faut apprendre à désapprendre. 
Au delà du fait que cela pourrait être l'une des voie vers la sagesse -- supposée de l'\textit{homo sapiens}, et plus spécifiquement de l'\textit{homo historicus}, qui s'est construit sur les certitudes des autres --, il s'agit ensuite de reconstruire \textit{ex nihilo} une pratique créative, constructive et collective de la musique \textit{hic et nunc}. 

... empirisme manifeste, avec la connaissance de la déconstruction, usant de l'induction, afin d'explorer d'autres voies, d'autres possibilités de concevoir un fait musical, en conscience ...

... partage, communiquer, transmettre --- intra-groupe, extra-groupe.
%Probablement l'une des meilleures directions à suivre si l'on cherche un chemin vers la sagesse -- supposée de l'\textit{homo sapiens}, et plus spécifiquement l'\textit{homo historicus}, qui s'est construit sur les certitudes des autres. 

appropriation de l’idée dans le respect avec la source (ponctuelle tant la “véritable source” se perd par l’appropriation de l’idée sans le respect de la source — quoique l’idée peut se trouver dans “l’air” auquel cas il est possible de parler de source d’inspiration).

%on peut se laisser porter par la musique ...

\bigskip


[ … ] dans un contexte sociétal qui se doit d'évoluer; pour lors, ce nouveau paradigme a pour vocation d'initier un mouvement culturel en tant que tel et social ethnologiquement -- voire éthologiquement -- parlant au regard du contexte historique.%\footref{efn:pr}. 

%The next endnote\endnote{This endnote gets a label.}\label{en:test} has  the number~\ref{en:test}. Let's now test endnotemark\endnotemark[\ref{en:test}].

%\bigskip
Le but est alors de penser la musique comme concept en termes de formalisation et de modélisation. Ainsi le concept s’inscrira dans une praxis%\footref{efn:pr} 
musicale délibérée (paysage sonore, synthèse sonore, ensemble musical, etc.).

Autrement dit, cet atelier est un lieux de discussion philosophique, voire scientifique, sur le comment retranscrire une idée musicale. C'est alors un lieu d'échange de compétences expérientielles et techniques, tant dans la pratique compositionnelle qu'instrumentale dont le dessein et bien sûr de stimuler notre propre créativité. Il s'agit aussi de créer une dynamique de recherche en marge des institutions.


Aussi, cet atelier s’intéresse aux travaux de « polyphonie spatiale » et plus particulièrement dans un espace quadriphonique usant de l'algorithmie.
La syntaxe spatiale est ainsi partie prenante de la « mise en scène » du son.

La composition ouverte est une approche conceptuelle qui se présente sous la forme d’une partition qui peut être graphique, textuelle et/ou traditionnelle en termes de directives formelles et/ou structurelles. La subjectivité et l’expérience de l’interpréte font de la partition un objet « sur mesure ». Autant dans le fond que dans la forme, ce type de composition s’inspire largement des travaux compositionnels de John Cage et de Jean-Yves Bosseur.


l’œuvre ouverte pour ce qui concerne mon approche peut être vu comme un processus de cristallisation …
La cristallisation peut se faire au niveau de la structure et/ou au niveau de l'object sonore lui-même.

de l’œuvre ouverte, une possible et parfois souhaitable cristallisation peut émerger spontanément.

La pertinence de cette approche repose sur la prédictibilité ou la constance du ou des concept(s) émergent(s) s’inscrivant dans le contexte systémique de la partition. Autrement dit, les relations entres les propositions compositionnelles du compositeur, l’interpréte et l’intention artitisque sont intimement liés et délibérés.

Comment ressentons-nous l’intention musicale ? L’affect est bien souvent le facteur décisif. L’affect concerne notre propre subjectivité, notre propre culture et notre propre expérience du phénomène sonore et ce, en terme circonstanciel. Aussi, c’est par le biais de la perception, phénomène physique et cognitif, qu’il est possible d’appréhender le phénomène musical en termes de formalisation et modélisation.

Cela reste dans tout les cas une démarche purement philosophique pour laquelle il convient de rester humble. Penser la musique autrement est prospecter au delà du pays fertile, en quête d’un savoir insaisissable avec lequel je vous invite à composer.

Bien que le concept d'œuvre ouverte peut inclure tous les concepts musicaux précédemment évoqués à différents niveaux, le changement de paradigme s'opère sur notre compréhension du monde sonore en musique, par notre expérience de vie en conscience.

\bigskip

C’est autant de pistes qui nous appartient de suivre selon la situation, de nous en inspirer selon notre humeur, et d’explorer au delà des sentier battu ce que le monde sonore peut nous offrir pour notre plus grand plaisir. 

\newpage
  
\nolinenumbers
\setenotez{list-name = {Notes}}
\printendnotes

\newpage

\begin{titlepage} 
\clearpage
\thispagestyle{empty}
\setstretch{.8}
\pagecolor{yellow!10}

\vspace*{\baselineskip} 	
	
	\vspace*{0.167\textheight} 
	
\centerline{The truth is out there.$^*$}
  \vfill
  {\scriptsize \noindent $^*$ La vérité est un concept définit par ce que l'on pourrait appeler un \textit{cluster polysémique nodal fluctuant}, et est par conséquent loin d'être universelle et absolue; n'en déplaise aux gens de pouvoir pétris de certitudes, souvent au service de dogmes ou d'oligarques dont ils n'en comprennent que rarement les conséquences destructrices induites.}
  
 % {\footnotesize  $^*$ La vérité est un \textit{cluster polysémique nodal fluctuant}, et est loin par conséquent d´être universelle et absolue; n'en déplaise aux gens de pouvoir pétrient de certitudes, souvent au service de dogmes dont ils n'en comprennent que rarement les conséquences destructrices induites. }
%\vspace{-0.2cm} 
%\centerline{  {\footnotesize  ... }}


\end{titlepage}


\end{document}