\documentclass{article}
%\documentclass[a5paper]{article}
%\usepackage[fontsize=8pt]{scrextend}
%% count number of words:
% $ pdftotext essai.pdf - | wc -w
% $ detex essai.tex  | wc -w

\usepackage[french]{babel}
\usepackage[T1]{fontenc}
\usepackage[utf8]{inputenc}
\usepackage{setspace}
\usepackage{longtable}
\usepackage{xcolor}
\usepackage[font=small,margin=1mm]{caption}
\usepackage[toc,page]{appendix}
\usepackage{tocloft}
\addtocontents{toc}{\cftpagenumbersoff{part}}
\addtocontents{toc}{\cftpagenumbersoff{subsection}}
\renewcommand\cftsecfont{\bfseries}
\renewcommand{\cftaftertoctitle}{\vspace{5pt} \hrule}
\usepackage[nottoc,notindex]{tocbibind}
\usepackage[pdftex,backref=false,pdfborder={0 0 0}]{hyperref}
\usepackage{textcomp}
\makeatletter
\renewcommand{\@biblabel}[1]{}
\makeatother
\pagecolor{yellow!10}
\title{%\textsf{\Huge \textbf{Open Composition}}
}
\author{\textsf{\textbf{Yann Ics}}}
\date{}
%\setlength{\oddsidemargin}{10pt} % Marge gauche sur pages impaires
%\setlength{\evensidemargin}{10pt} % Marge gauche sur pages paires
%\setlength{\marginparwidth}{84pt} % Largeur de note dans la marge
%\setlength{\textwidth}{381pt} % Largeur de la zone de texte (17cm)
\setcounter{secnumdepth}{0} 
\usepackage[modulo]{lineno}
%\linenumbers

%%******************* added <<<<<<<<<<<<<<<
\makeatletter
\renewcommand*\l@section{\@dottedtocline{1}{1.0ex}{3.5ex}} % add dots to TOC
\def\addcontentsline#1#2#3{%
    \addtocounter{linenumber}{-1} %  the line number of the section name
    \addtocontents{#1}{\protect\contentsline{#2}{#3}{\thelinenumber}{}%
   % \protected@file@percent
    }
    \stepcounter{linenumber} % back to right linenumber
}
\makeatother

\renewcommand{\contentsname}{Contents~\hfill\textbf{Line}}
\usepackage{enotez}
% https://tex.stackexchange.com/questions/202805/refer-to-an-endnote-with-the-enotez-package-points-to-the-wrong-note
\renewcommand{\footnote}{\endnote}
\newcommand{\footref}[1]{%
  \enotezwritemark{\enmarkstyle\ref{#1}}%
}
%\let\endnote=\endnote
\newcommand{\eo}{\textsl{etou}}
\definecolor{DarkRed}{RGB}{139,0,0}
\pagenumbering{gobble} 
\usepackage{enumitem}
\usepackage{contour}
\usepackage{ulem}
\AtBeginDocument{\addtocontents{toc}{\protect\thispagestyle{empty}}} 

\renewcommand{\ULdepth}{1.8pt}
\contourlength{0.8pt}

\newcommand{\myuline}[1]{%
  \uline{\phantom{#1}}%
  \llap{\contour{white}{#1}}%
}

\newcommand{\myulineyellow}[1]{%
  \uline{\phantom{#1}}%
  \llap{\contour{yellow!10}{#1}}%
}
\usepackage{gensymb}

%%%%%%%%%%%%%%%%%%%%%%%%%%%%%%%%%%%%
%%%%%%%%%%%%%%%%%%%%%%%%%%%%%%%%%%%%
%%%%%%%%%%%%%%%%%%%%%%%%%%%%%%%%%%%%
%%%%%%%%%%%%%%%%%%%%%%%%%%%%%%%%%%%%
%%%%%%%%%%%%%%%%%%%%%%%%%%%%%%%%%%%%
%%%%%%%%%%%%%%%%%%%%%%%%%%%%%%%%%%%%
%%%%%%%%%%%%%%%%%%%%%%%%%%%%%%%%%%%%

\begin{document}

%\begin{minipage}{.033\textwidth}\includegraphics[width=\linewidth]{1123}\end{minipage}

\makeatletter
  \begin{titlepage}
  \centering
  {\Large \@author} \\
  \rule{\linewidth}{.5pt}
  \vspace{4cm}
  
  {\LARGE \textbf{\@title}} \\
   \vspace{6 mm}
   {\color{brown} {\Large \textbf{\texttt{\, ... à la recherche d'un nouveau}}}}\\
   \vspace{2mm}
   {\color{brown}{\Large \textbf{\texttt{\qquad paradigme de musique vivante.}}}}\\
%    \vspace{12mm}
%   {\color{black}{\normalsize \qquad \textsl{\texttt{The truth is 
%   \textup{\myulineyellow{until proven otherwise}}
%  %\textup{\myulineyellow{fortunately always}} 
%   out there}}$^*$}}\\
  \vfill
  \rule{\linewidth}{.5pt}
%  in search of a new paradigm of living music.
     {\Large \textsf{\textsl{}}}\\ 
          {\Large \textsf{\textsl{}}}\\ 

      {\Large \textsf{Essai}}\\

 \end{titlepage}
\makeatother
\setcounter{tocdepth}{1}
\renewcommand{\contentsname}{Sommaire\\}

{\color{DarkRed}\tableofcontents
%\addtocontents{toc}{~\hfill\textbf{Ligne}\par}
\noindent\hrulefill}

\pagecolor{white}

%% blank page
%\newpage
%\thispagestyle{empty}
%\mbox{}
%\newpage

%\vspace*{,5cm}

\renewcommand\thelinenumber{\color{purple}\arabic{linenumber}}
\linenumbers
  
\section{Avertissement}

Le but de cet essai est d'exposer et de contextualiser une pensée sur notre propre existence -- à travers le prisme du monde vibratoire et en tant que musicien -- qui m'apparaissait pourtant simple, mais dont la mise à l'écrit s'est révélé d'une complexité déroutante, tant une question en appelle une autre, et tant la réponse peut soulever des concepts inattendus, pluridisciplinaires, reliant parfois des savoirs disparates. Aussi, 
je revendique un style d'écriture qui se veut concis et aussi précis que possible -- à la limite de l'aphoristique --, à l'encontre d'une quelconque vulgarisation, et qui requiert parfois l'utilisation d'une terminologie spécifique. En effet, il existe des mots, qui synthétisent, subsument, condensent et cristallisent des concepts, 
et d'autres, plus diffus, parfois en cours de conceptualisation, paradoxalement éloquent, 
et qui révèlent un monde aussi complexe que fascinant, pour lesquels j'estime opportun d'en illustrer la pertinence.
%, et selon ma propre sensibilité. 
J'avoue prendre le risque de ne pas être lu ou compris, mais à l'heure où l'accessibilité du savoir est à la portée de tous -- ou presque --, j'invite le lecteur, le cas échéant, à rechercher, disséquer, au delà de mon propre propos, le sens de certains concepts, mots, peu ou pas développés à dessein, et de ne pas hésiter à déconstruire mon modeste édifice, usant pour les plus aventureux de la grammatologie déridienne\endnote{Jacques Derrida écrit en 1967 \textit{De la grammatologie}, %
% SOURCE [MHRA style] -- Contributeurs à Wikipedia, 'De la grammatologie',  Wikipédia, l'encyclopédie libre, 14 juillet 2018, 10:54 UTC, <https://fr.wikipedia.org/wiki/De_la_grammatologie?oldformat=true#Port%C3%A9e_de_la_grammatologie&oldid=150370940> [Page consultée le 19 septembre 2021]
ouvrage dans lequel il vise à décrire, à comprendre, les relations entre signifiant et signifié d'un texte, \textit{à savoir une science qui ne cesse de revenir sur ses conditions de possibilité, une science qui n'avance qu'en régressant vers le doute qui est son fondement}. Autrement dit, la grammatologie deridienne est un concept qui consiste à déconstruire le language à la manière d'un horloger afin d'en comprendre les mécanismes épistémiques\footref{efn:ep}. \textit{Si la science et la philosophie ont la vérité comme objet visé, la grammatologie voit en revanche dans l’écriture un phénomène sans vérité. L’écriture est cela même qui déstabilise la notion de vérité depuis son origine.}}. 
% SOURCE ---> https://hal.archives-ouvertes.fr/hal-01504272/document -- PAR-DELÀ LE SENS, L’ECRITURE ET LE TEXTE. JACQUES DERRIDA Par Joseph-Igor MOULENDA
%\bigskip

\section{…}
Suite à une longue reflexion au sujet d'accéder, de pratiquer, de créer et de partager le phénomène sonore en tant qu'objet musical, et à une expérience de vie dans laquelle je suis toujours en quête par ailleurs d'un ailleurs, 
% et à cause d'une tendance misanthropique  
 je me propose d'exposer sous la forme d'un essai ma vision et mes espérances sur notre propre existence et notre propre ontologie, et ce à travers le prisme de l'expérience acoustique et musical. En effet, si il est devenu évident que la musique soit un produit de consommation courante, avec ceux qui la font et ceux qui la consomment, au regard de ceux qui la vivent au sein d'une recherche introspective ou d'un rituel,
 %d'une communion propre à un environnement propice
il m'est possible de ressentir un vide entre le fait de ne pas souscrire à la logique consumériste et de ne pas avoir d'ascendant sur l'experience musicale en tant que pratique collective.
% -- certes, excepté les quelques élus à la gloire de l'excellence pour lesquels on se doit de s'incliner. 
Serait-ce l'horizon d'un nouveau paradigme, \textit{à la limite du pays fertile}. %
% SOURCE -- --> Paul Klee, --> Pierre Boulez
% les tenants et les aboutissants
Sans aucun doute, et l'objet de cet essai est d'en identifier le substrat et d'en décrire les modalités praxéologiques\endnote{ De \textit{Praxis}, qui désigne la pratique d'une activité visant au développement du sujet.}\label{efn:pr}. 

\bigskip
Ceci est purement théorique, car toutes mes tentatives en ce sens ont été des échecs, et se sont limitées à une espèce de galimatias plus ou moins musicale ressenti comme incontrôlable et incontrôlé\endnote{Cela dit, en une ou deux sessions, il est difficile d'obtenir ne serait-ce que les prémisses d'un effet satisfaisant compris par l'ensemble des participants, bien qu'objectivement il y eut matière à considérer.} qui n'a pas enthousiasmé les participants. Probablement dû à une éloquence défaillante de ma part et aussi à une absence de linéaments -- quoique cela aurait du être définit par le groupe --, en bref une préparation inadéquate\endnote{L'idée première était d'initier un processus par la prise de décision par consensus, lequel devait être introduit conceptuellement, demandant un `effort intellectuel' que l'audience n'était pas prêt, ni préparé, à exercer.\\ \indent La prise de décision par consensus est un processus impliquant tous les participants afin de résoudre une problématique pour le bénéfice de tous. Cela implique l'analyse du projet pour en comprendre les tenants et les aboutissants, afin d’être discuter jusqu'à ce que chaque personne soit d'accord. Si une personne désapprouve la résolution, le groupe se doit de proposer une alternative jusqu'à ce que tout le monde soit d'accord. Bien sûr, cela peut impliquer certaines règles en cas de désaccord afin de maintenir la cohésion du groupe. \\ \indent À noter que ces règles sont aussi proposées par consensus et peuvent par conséquent évoluées, être adaptées ou changées, soulevant la problématique d'initier ou de proposer les prémisses opératoires selon la propre sensibilité du groupe, et ce sans être sous l'influence d'un modèle quelconque, tant que faire se peut.}$^{,}$\endnote{Ceci constitue le pourquoi de cet essai, dont le dessein est justement, dans un premier temps, de préparer les potentiels protagonistes à une expérimentation \textit{in vivo}, afin d'en percevoir les effets sur l'accomplissement personnel et collectif dans un second temps.}. Cependant, ce `nouveau paradigme' doit se construire selon un mouvement consensuel des participants qui ne souffre ni leader, ni  maitre, et doit émerger d'une édification systémique. %(1) du moins de façon permanente. De plus, comme toutes \textit{praxis}, celle-ci doit s'inscrire dans le temps 
Bien sûr, cela doit s'inscrire dans le temps -- \textsc{je dirais pour ma part} (par expérience, aussi modeste soit-elle) \textsc{au moins une session par semaine sur une période d'au moins 6 mois pour en apprécier les perspectives}, avec une `performance' au sens rituel marquant la fin ou le début de chaque mois -- et demande une assiduité et un volontarisme (bref une véritable \textit{praxis}%\footref{efn:pr}
), qui dans le cadre proposé -- c'est à dire libre\endnote{Un des points important dans le déroulement des activités était la possibilité d’un accès libre, afin de créer son propre calendrier, en fonction bien sûr des ressources et des disponibilités du lieux et des personnes en termes de possibilités pratiques telles que l’espace, les horaires et la nature du processus en cours. Dans tout les cas de figures, c’est le partage d’expériences et de compétences sans apriori et le respect d'un calendrier convenu qui devait constituer le dessein de cet `atelier'.\\ \indent Le manque de temps est aussi revendiqué, compromettant la liberté individuelle de faire ce qui pourrait sembler juste ou opportun -- à moins que ce ne soit une façon d'éluder notre mal-être face à notre propre capacité à comprendre et à changer le monde.} -- n'a tout simplement pas trouvé, ne serait-ce que par curiosité, son `public'. Les musiciens sortent difficilement du cadre tonal ou de leur piédestal elitiste, de leurs habitudes plus ou moins professionnelles, tandis que les autres se déclarent ineptes à toutes pratiques musicales. En bref, les uns produisent, les autres consomment, point. 

%This is purely theoretical, because all my attempts in this direction have been unsuccessful, and have been limited to a kind of free jazz jamming session, which did not thrill the participants. Probably due to a lack of pedagogy on my part and also to a real lack of charisma. In addition, this musical practice must take place over time -- I would say at least over a period of 6 months with one session per week to appreciate the perspectives -- and requires diligence and voluntarism, which in the proposed framework -- that is to say free -- has simply not found, if only out of curiosity, its `audience'. The musicians have difficulty getting out of the tonal framework and the others declare themselves inept at all musical practices. In short, some produce, others consume, period.

Il faut sans doute être fou pour vouloir changer cela. Peut être ne le suis-je pas assez moi-même, et que cet écrit est finalement destiné à en inspirer certains afin de permettre à d'autres -- y compris moi-même -- d'en être les acteurs et les témoins ...
%You probably have to be crazy to want to change that. Maybe I am not enough myself, and that this writing is ultimately intended to inspire some ...

%\section{Propos}
\section{\textsl{Purpose}}
\textsl{It appears that our globalist civilisation has failed, and finally  be only a consumerist purpose to serve the} \textit{Brave New World}\endnote{Le \textit{Meilleur des mondes} est une projection à terme -- d’aucuns parleront de société parfaite -- basé sur le conditionnement des castes et la compartimentation du savoir et des compétences.%\\   \indent L'aspect politique -- en tant qu'état d'une possible transition -- ne sera pas trop développé au cours de cet essai, mais il est parfois nécessaire de préciser ce qui se cache derrière ce besoin de changement, ce qu'il faut changer et peut-être comment et pourquoi. Ces digressions se feront au besoin ici ou là en note.% -- preceded by the symbol  
%\\ \indent Présentement, il importe de souligner la dérive totalitaire de la spécialisation et de la professionnalisation dans nos sociétés. Le premier réduit les connaissances à une ou quelques thématiques contre toute observation holistique, et par induction le second pervertit les relations humaines car le professionnel agit de manière routinière, standardisée, en tant que spécialiste. Cela signifie une dépendance excessive du système et de ses institutions, notamment par le cloisonnement des responsabilités.
\\ \indent \textsl{The political aspect -- as a possible transitional state -- won't be developed too much during this writing, but it needs sometimes to be precise about what is behind the need for change and what to change and maybe how and why. These digressions will be done here and there, when required as a note.}% -- preceded by the symbol \anar
\\ \indent \textsl{Presently, it is essential to emphasise the totalitarian drift of specialisation and professionalisation in our societies. The first one reduces the knowledge to one or few topics against any holistic
observations. By induction, the second one perverts human relationships because the professional acts a routine as a specialist, independently of any other considerations. This means an over-reliance on the system and its institutions, notably by partitioning responsibilities. Moreover, for most of them,  this is just a race for profits, nothing more, nothing less.}
}\label{efn:bnw} \textsl{described by Aldous Huxley in 1932, under the `approving gaze' of} \textit{Big Brother}\endnote{\textit{Big Brother} -- en référence au roman de George Orwell \textsl{1984} publié en 1949 -- est un système de surveillance et de propagande omniprésente -- incarnant le `parti' et qui encourage la délation jusqu’au sein de la famille --, et par extension tout dispositif, notamment par le biais de l'intelligence artificielle connectée, qui nuit à la liberté individuelle et à la vie privé, généralement sous le prétexte fallacieux de sécurité nationale.}. 
\textsl{There is definitely no future for this world}\endnote{\textsl{The inevitable collapse of our civilisation is due to the hegemony or even the supremacy of the monetary system. Indeed, Money -- \textit{i.e.} the haves, the super-rich -- rules the world against Nature, against common sense, promoting notably competitiveness instead of sharing resources.}
%\\ \indent \textit{L’obligation de produire aliène la passion de créer.}
}. \textsl{We have to build from scratch a new world. There is no miracle recipe, but some alternatives exist}\endnote{\textsl{Some of them require -- in the context of political transition -- to be mentioned such as: 
\begin{itemize}
\item A Mondial negative growth is obviously needed -- at least according to the actual standard, which means positive growth on different markers.
\item An Unconditional Universal Income, a notion imagined and developed by the french sociologist and economist Bernard Friot.
\end{itemize}}} \textsl{and can be the daylight of our hopes. 
For instance, at our modest level, the restructuration of tribes, clans or any small communities should do consensus for a New World based on respect for life and by extension Nature, and the development of the inner culture of each group}\endnote{\textsl{The national level for instance is definitely too wide for cultural identity, and with the globalisation, this level tends to be the same for every country -- at least for connected people. The idea here is to deconstruct the concepts of nation, state and country.
\\ \indent Obviously, this approach goes against the established order, but this is one way among others if we want to evolve with awareness.}}. 

\textsl{Obviously, music can play a determining role inside a group of persons and between groups as embodied beings.
Moreover, music promotes interrelationship and create a living communion with each other.
The `ultimate' aim is to raise awareness of another way to think and to practice music, and by this way stimulate our own creativity and fantasy.
We need to consider the didactic approach which has to be designed to fit into a larger context such as a community or societal renewal, where each individual must participate in one way or another in the musical workshop. 
I believe that everybody can be part of musical practice in a playful way, for the benefit of  both the individual and the community level.
%  
%En effet, la musique permet un développement cognitif significatif créant un lien social fort et permet aussi pour chacun de bénéficier de ses nombreuses vertus thérapeutiques.
Indeed, music allows significant cognitive development creating a strong social bond and also allows everyone to benefit from its many
therapeutic virtues on the body and mind}\endnote{La pratique de la musique est un vecteur de plasticité anatomique et fonctionnelle en termes de registres perceptif, cognitif (mémoire et langage), émotionnel et moteur, impliquant le réseau neuronal dans son entièreté et ses répercussions sur l'ensemble du corps. En d'autres termes, la musique (1) optimise ou améliore globalement les aptitudes du cerveau, telles que la concentration, la mémorisation, et (2) au delà du divertissement supposé — que je qualifierais pour le moins de digression, par extension au sens étymologique premier de `divertissement’ (\textit{i.e.} du latin \textit{divertere} qui signifiait le fait d'amener [quelqu'un] vers d'autres idées et sans gaieté particulière) — dispense un nécessaire et bénéfique exutoire (ou canalisateur d'énergie) dans de nombreux cas; (3) aide (peut être le corollaire des deux précédents points) -- en améliorant de façon significative -- à soigner ou à soulager des sujets dans le cadre de la thérapeutique, ou tout simplement à mieux vivre, pour soi et avec les autres.}.

% - [ ] revoir cognition et bienfait … améliore la concentration — la musique ne rend pas plus intelligent mais optimise globalement les aptitude du cerveau, telle que la concentration, la memorisation, la relaxation, …  et au delà du divertissement supposé — que je qualifierais pour le moins de digression par le sens étymologique premier de `divertissement’ — un nécessaire et bénéfique exutoire dans de nombreux cas 
\bigskip

\textsl{Beforehand, we mean to redefine music, historically and for this purpose as a collective work within a never-ending rhizomatic learning}\endnote{Le \textit{rhizomatic learning} est un modèle où la recherche d'un résultat est effectuée par les participants de manière dynamique en fonction du contexte. Il met l'accent sur l'interconnexion des compétences de chacun par consensus.\\ \indent Le terme rhizome est attribué à Gilles Deleuze et Félix Guattari évoqué dans \textit{Mille plateaux }  -- édité en 1980 -- dans lequel ils dévelopent l'idée d'une `structure organique' définie par les principes suivants : \begin{itemize}%[label=\textbullet]
\item « \textit{principe de connexion et d'hétérogénéité} [ … ] \textit{n'importe quel point d'un rhizome peut être connecté à un autre, et doit l'être} »
\item « \textit{principe de multiplicité} [ … ] \textit{l'organisation propre du multiple en tant que tel, qui n'a nullement besoin de l'unité pour former un système} »
\item « \textit{principe de rupture assignifiante}  [ … ]  \textit{un rhizome peut être rompu, brisé en un endroit quelconque} »
\item « \textit{principe de cartographie et de décalcomanie} » (la carte est l'exploration de la structure, c'est à dire \textit{une expérimentation en prise sur le réel}, tandis que les calques sont les éléments structuraux et identifiés comme tels.)
\end{itemize}
%\\ -- « \textit{principe de connexion et d'hétérogénéité} [...] \textit{n'importe quel point d'un rhizome peut être connecté à un autre, et doit l'être} »\\ -- « \textit{principe de multiplicité} [...] \textit{l'organisation propre du multiple en tant que tel, qui n'a nullement besoin de l'unité pour former un système} »\\ -- « \textit{principe de rupture assignifiante}  [...]  \textit{un rhizome peut être rompu, brisé en un endroit quelconque} »\\ -- « \textit{principe de cartographie et de décalcomanie} » (la carte est l'exploration de la structure, c'est à dire \textit{une expérimentation en prise sur le réel}, tandis que les calques sont les éléments structuraux et identifiés comme tels.)
}. \textsl{This means} `pensez la musique autrement', \textsl{or in other words, get out of our mercantile logic and rid of fanatic postures.
There is no need to retrace our steps or copying tribal lifestyles, but rather to make music our own, based on its} \textit{épistémè}\endnote{Contrairement à l'épistémologie qui relate de la connaissance diachroniquement suivant une logique généalogique, l'\textit{épistémè} est la contextualisation synchrone de la connaissance relatant de la complexité d'un système dans l'espace et le temps, en d'autre termes, le contexte historique et social dans lequel la connaissance est acquise.
%\textit{\textbf{Épistémè}}: unlike epistemology which relates knowledge diachronically in a genealogical tree, \textit{épistémè} is the synchronous contextualisation of knowledge relating complexity of a system of thought in space and time, in other words, the societal and historical context in which knowledge is spread.
}\label{efn:ep}
\textsl{within the group.}
  
\section{Contexte historique}

D'un point de vue historique et formel, le fait musical fut décrit -- et l'est encore -- la plupart du temps dans un rapport  d'opposition paradigmatique en termes de discrimination sociale tel que la musique profane versus la musique sacré, la musique traditionnelle versus la musique savante, et la musique populaire versus la musique élitiste.
%- musique profane versus musique sacré -- introduisant les notions de monophonie et de polyphonie;
%- musique traditionnelle versus musique savante -- par extension danses et développement de la forme;
%- musique populaire versus musique bruitiste -- comprenant l'industrie musicale et la conceptualisation de l'objet sonore.
On pourrait ajouter, à partir du \textsc{xx}\ieme ~siècle, en termes de discrimination sociétale\endnote{\textsl{The difference between social and societal is that `social' has to do with human relationships and groups of people within a society often class-based while `societal' has to do with society as a whole, as a system, by its organisation.}}, la musique `\textit{underground}' et de recherche versus la musique `\textit{mainstream}' dite aussi de masse. %diffusée en boucle dans les mass média et soutenue par les institutions en place.
En fait d’opposition et d’un point de vue purement musicale, la disjonction
de ces dualités paradigmatiques est la plupart des cas beaucoup plus diffuse que ce que les
termes en eux-même pourraient laisser entendre\endnote{Dans ce contexte nominalisme, la discrimination et la typologie, nécessaire à l'intelligibilité d'une phénoménologie,  permettent d’accéder par l'esprit à une réalité souvent complexe, à en saisir la problématique, mais créent dans le même temps des tensions, des confusions, voire des impossibilités sur les inévitables recouvrements.}. Cela dit, il s’agit bien du rôle
que la musique tient dans la vie sociale qui en détermine véritablement sa nature,
indépendamment de l’objet musical lui-même. Autrement dit, l’interprétation
musicale en ces termes s’inscrit dans une démarche herméneutique\endnote{L'herméneutique ici est l'\textit{ensemble des connaissances et des techniques qui permettent de faire parler les signes et de découvrir leur sens} -- selon la définition de Michel Foucault formulée dans \textit{Les Mots et les choses} (édité en 1966).}.

%Ces deux paradigmes s’inscrivent alors et aussi dans le temps en termes de synchronie et de diachronie. 
\bigskip

Dans un même ordre d'idée et d'un point de vue structurel, nous pouvons identifier trois paradigmes musicaux respectivement inclusifs tels que la musique de sons\endnote{La musique de sons se réfère présentement à l'objet sonore en tant que tel et plus spécifiquement à une ontologie du phénomène vibratoire compris comme un fait musical.}, la musique modale\endnote{La modalité est un système de construction scalaire prédéterminé permettant une variété de mélodie et de configurations motiviques dans son contexte culturel et compositionnel.  Autrement dit, la modalité est le résultat d’une selection de sons -- corréler pour un instrument donné à sa facture ou à son intonation --  dans un contexte culturel donné, et qui constitue les `phonèmes' d’une phraséologie musicale codifiée.%    La période comprend la musique de la grèce antique – à partir de l'épitaphe de Seikilos – et les chants grégorien transcrit principalement en neumes. ...
} et la musique tonale\endnote{La tonalité est un cas particulier de la modalité, héritée de la gamme pythagoricienne, dont la hiérarchisation des degrés crée des pôles d'attraction, %notamment de la sensible vers la tonique, selon les deux modes diatoniques %heptatoniques 
%mineur et majeur -- le premier étant le relatif du second. Ainsi, la tonalité repose
reposant sur la dualité tension/détente, dissonance/consonance, définit respectivement par la cadence parfaite du cinquième au premier degré du ton considéré.\\ \indent La tonalité est un compromis entre une volonté d’unifier les échelles modales diatoniques selon les lois de l’acoustique -- \textit{i.e.} relatif aux harmoniques du monocorde pythagoricien -- et d’universalisme, en passant par le temperament égal, et qui constitue encore de nos jours le standard tant en pratique qu’en théorie.% La majeur partie des musiques dites commerciales sont de nature tonale.%    reposant sur la dualité tension/détente ... tonique/dominante sensible etc ...
}, lesquelles s'articulent autour de trois périodes historiques. %La première articule la modalité vers la tonalité, par 

Ce discernement doit se comprendre selon la culture dite `occidentale'. Il va de soi que toute musique confondue est une combinaison de sons issues d'échelles plus ou moins complexes, intégrant aussi bien les bruits (sonailles, maracas, ...) que les sons pures (flûte, piano, ...). Les modalités diatoniques du plain-chant et la tonalité sont des échelles particulières développées uniquement en Europe. %La tonalité connaitra son apogée durant le romantisme du  \textsc{xix}\ieme ~siècle.

La première articulation pourrait commencer par la découverte d'instruments préhistorique dont le caractère modal est avéré tel que les flûtes datant du paléolithique supérieur -- soit il y a environ 35\,000 ans\endnote{Il s'agit %pour les plus anciens artefacts -- datées au carbone 14 -- 
de flûtes en os et en ivoire, découvertes en 2008 dans la grotte de Hohle Fels, située dans le Jura Souabe à Bade-Wurtemberg en Allemagne.} --, ou bien aux premières traces de musiques écrites sous forme de textes accompagnés et/ou chantés, il y a 3\,400 ans pour les plus anciennes\endnote{Il s'agit de chants Hourrites accompagnés d'instructions pour instrument à neuf cordes, gravés sur tablette d'argile en écriture cunéiforme, découverts en 1950 dans l'ancienne cité d'Ougarit en Syrie.}.%, ou bien encore ... monocorde pythagore, epitaph de seikilos ...

La deuxième articulation concerne, à partir de la Renaissance et plus particulièrement en Europe, l'émergence de la tonalité en tant que système musical,  
%des tonalités majeure et mineure 
incluant l'établissement du tempérament égal\endnote{Le tempérament égal divise l'octave en douze demi-tons égaux, permettant de moduler vers n'importe quelle tonalité; c'est un compromis à l'encontre de l'intonation juste. Le tempérament égal constitue depuis le socle normatif de la musique en général.} à la période Baroque. Le fait marquant et déterminant  fut certainement le perfectionnement de l'imprimerie\endnote{En imprimerie, deux techniques font leur apparition, l'une utilisant une technique d'impression dite de typographie à caractères mobiles. L'amélioration concerne le mode de fabrication de ces caractères mobiles en alliage de plomb, la presse à vis et l'encre grasse à base d'huile de lin, mise au point notamment par Gutenberg vers 1450. L'autre est la technique dite de taille-douce, découverte par hasard par le graveur florentin Maso Finiguerra en 1452, qui consiste à graver en creux une plaque de métal, souvent en cuivre, puis rempli de noir de fumée avant d'y appliquer et presser une feuille de papier humide.} et le début d'une industrialisation papetière grâce notamment à l'énergie hydraulique. Cela permit l'édition et la diffusion de partitions et d'ouvrages théorique sur la musique, qui contribua largement à l'affirmation de la tonalité en tant que système émergent que connaitra l'ère classique, puis le romantisme du \textsc{xix}\ieme ~siècle, tout en maintenant la modalité par la standardisation typographique induite de la transcription musicale des œuvres du passé.

Et enfin la troisième articulation couvrant largement le \textsc{xx}\ieme ~siècle va déconstruire\endnote{La déconstruction concerna l'art dans son ensemble depuis les origines, de l’art pariétal paléolithique au monochrome de Kasimir Malevich, du \textit{musilangage} au silence de John Cage, des Venus préhistoriques à l’urinoir de Marcel Duchamp; autrement dit de l’œuvre sans signature à la signature `sans œuvre' (du moins il reste le concept dans tous les cas, pour lequel l'œuvre devient une possible représentation) -- voire du transcendant à l'immanent, de l'art à l'artiste, de l'acteur/spectateur au spectateur/acteur, du signifiant au signifié.\\ \indent Noter que le \textit{musilangage} est une hypothèse de recherche transdisciplinaire qui défend l'idée d'une origine commune entre la musique et le langage; \textit{la  musique  aurait  ainsi  une  origine  vocale,  et  le  langage  une origine  musicale}.} %
% SOURCE -- Caroline Boe. Le musilangage : une hypothèse d’origine commune entre la proto-musique et le proto- langage. 2019.  hal-02004782v2 
la tonalité, puis la modalité pour `re-découvrir', `ré-investir', la musicalité du phénomène sonore, élément constitutif de la musique de sons. 

Le point de départ se situe durant la révolution industrielle avec l'invention de la téléphonie et de l'enregistrement sonore vers la fin du \textsc{xix}\ieme ~siècle, qui va littéralement `déterritorialiser' %\textit{déterritorialisé}\endnote{\textit{\textbf{Déterritorialisation}}:  concept créé par Gilles Deleuze et Félix Guattari en 1972 ...}
 l'objet sonore dans l'espace et dans le temps. Ce phénomène va amplifier -- dans tous les sens du terme -- et %démocratiser
 diffuser la tonalité comme standard musical, avec le tempérament égal, ainsi que la modalité sous toutes ses formes avec l’ouverture aux musiques du monde, %, initiée par les compositeurs du \textsc{xx}\ieme ~siècle, depuis 
 à partir de l’exposition universelle de Paris de 1889, et entériner par la diffusion discographique et radiophonique durant les années 60, ce qui a permis d’élargir l’horizon musical en termes de pratique et d’écoute au delà de nos propres cultures.%musicales. 

Dans le même temps, de nouveaux instruments dits analogiques\endnote{Par définition, les instruments `dits' analogiques se caractérisent par le ou les traitements en fonction continu du temps d'un signal électrique.} ouvrent un potentiel inouï tant au niveau du timbre que de la technique compositionnelle.

À cela, s'ajoute la révolution informatique  -- dit aussi numérique --  à la fin du \textsc{xx}\ieme ~siècle,  par la démocratisation, grâce à l'internet, des ressources numériques -- \textit{i.e.} documentations, informations et applications -- permettant à tout à chacun de produire et de partager sa propre musique avec le monde entier, avec de nouvelles pratiques compositionnelles, des techniques de diffusions inouïes et des qui restent à inventer -- du moins pour ce qui concerne le monde `connecté'.

%de nouvelles techniques de diffusions inédites et de nouveaux horizons qui restent à inventer. %Le côté obscure de l'internet est la tendance à aplanir les différences culturelles 
%Il pourrait s'agir d'une quatrième articulation, mais nous n'avons pas encore le recul nécessaire pour le dire et nous ne savons pas encore vers quel paradigme celui-ci  ...

\bigskip
%pouvant s'inscrire dans une potentielle et nouvelle articulation.

Voilà pour l'essentiel. Pour illustrer notre propos, il nous faudra revenir sur la déconstruction musicale évoquée dans la troisième articulation. Cette dernière semble nous mener vers ce qui fait la musicalité du phénomène sonore. Comme un réenchantement, dans une sorte de cycle hélicoïdal sur l'axe du temps, dont le contexte est bien éloigné de celui nos lointains ancêtres en termes de connaissances acquises, de pratiques et de productions sonores. C'est bien dans cette perspective holistique que nous pouvons créer une véritable rupture, en nous réappropriant le monde sonore vers un objectif commun; un nouveau paradigme de musique vivante.

\section{L'œuvre ouverte}

Il faut dire que durant le \textsc{xx}\ieme ~siècle, il y eut beaucoup d'effervescence artistique et musical, et la musique de sons prit des formes aussi variés que pu l'être l'art en général, et la déconstruction fut une véritable épopée.

Il serait vain ici de faire une énumération de ces processus de déconstruction/reconstruction %-- de surcroit exhaustive -- de ce phénomène 
tant la terminologie mise en œuvre\endnote{La terminologie est loin d'être clair. En effet, d'aucuns parleront de style, de genre ou de mouvement, mais la différence -- voire la \textit{différance} derridienne -- est d'ordre typologique. % et chronologique, si bien que l'on pourrait en établir la généalogie. %Du prototype à l'archétype jusqu'au stéréotype, la nuance est subordonnée au régime doxique et épistémique.
%Pour ma part, 
\\ \indent Dans le même ordre d'idée, il serait plus intelligible de parler de prototype pour une particularité admise au sein d'une communauté, d'archétype pour la modélisation reposant sur l'invariance de plusieurs prototypes et/ou archétypes, et de stéréotype pour la `normalisation' doxique et épistémique d'un ou de plusieurs archétypes. À noter que la \textit{doxa} émerge d'un conditionnement qui est devenu de plus en plus manipulable et manipulé par les mass média et les réseaux sociaux. De plus, bien que l'\textit{épistémè} s'opère traditionnellement par voie institutionnelle, la tendance s'amenuise -- certes encore à la marge -- grâce notamment à l'accès et au partage du savoir par l'internet. Et ce à l'encontre d'une hégémonie mondialiste qui cherche à imposer un modèle de pensée unique
% -- comprenant bien entendu l'art dit `officiel' --
 qui est évidemment pas souhaitable.}  multiplie les dénominations et se prête volontiers à une polysémie, parfois confuse, inclusive, dont les contours restent incertains, subtils, aussi bien pour les musicologues que pour les acteurs se réclamant, ou ne se réclamant pas, de telle ou telle mouvance.

Cela dit, il est possible d'esquisser, selon l'analyse paradigmatique précédemment évoquée et selon l'\textit{instrumentarium}\endnote{L'\textit{Instrumentarium} se rapporte à un ensemble d'instruments, à une formation, et qui se caractérise par son organologie; laquelle 
 consiste essentiellement à décrire tout objet destiné à produire des sons notamment par \textit{le comment, le pourquoi, depuis quant, par qui ces sons sont-ils produits, pour jouer quelle musique, et à quelle fins} --  %
% SOURCE -- Geneviève Dournon, Instruments de musique du monde : foisonnement et systématiques. In : Musiques : une encyclopédie pour le XXIe siècle - L’unité de la musique (pages 833-864), Volume 5, Actes Sud/Cité de la musique, Arles/Paris 2007. 
et par extension à répertorier ces objets pour une culture donnée, de leur singularité à leur taxonomie à l'échelle de l'humanité.}, les différentes tendances musicales de ce \textsc{xx}\ieme ~siècle.
 Sachant que la musique tonale est le résultat d'une interprétation hiérarchique de la musique modale, laquelle s'inscrit dans la musique de sons, l'\textit{instrumentarium} peut aussi se résumer de manière inclusive par les instruments respectivement  %-- et par extension tout objet -- 
 à hauteurs indéterminées, à hauteurs déterminées `continues' \endnote{Par exemple la voix, les instruments à cordes sans frette, tels que le violon, le violoncelle, pour ne citer que les plus évidents, ou les instruments à vent tels que le trombone à coulisse, permettant d'interpréter n'importe quelle échelle musicale.}, et à hauteurs prédéterminées\endnote{Dans ce cas, les hauteurs des sons produits par l'instrument font partie d'un système d'échelle permettant un accordage convenue selon des critères et des objectifs délibérés, le plus souvent culturels}, 
 dont la manipulation compositionnelle s'effectue de manière intuitive, scripturale, analogique, ou numérique\endnote{Grâce notamment au sérialisme integral -- qui a ouvert une brèche sur l’aspect compositionnel concernant l’ensemble des paramètres contrôlables qui caractérisent l'objet sonore. La source sonore s'inscrit alors -- en plus de la hauteur et de la dynamique -- dans un mode de production et de contextualisation, tels que le traitement du signal, physique, analogique ou numérique, et l’acoustique en termes de spatialisation et de psychoacoustique.}. 
 Ainsi, selon ces critères, la combinaison de ces éléments caractérisant l'objet musical résume assez bien les spécificités inouïes de la musique du \textsc{xx}\ieme ~siècle. 
 
 Ceci vaut pour le phénomène sonore en tant qu'objet musical perçu. Au delà ou en deçà, il y a  le contexte situationnel ou rituel dans lequel la structure et la forme, c'est à dire le déploiement en temps selon des règles et des objectifs circonstanciels prédéfinis, vont prendre une dimension sociétale et communautaire %-- voire identitaire -- 
tacite, implicite, convenu, pour asseoir une autorité quelconque, faire école, faire partie d'une esthétique lié à un mouvement artistique, et que sais-je encore.

Tout cela s'est construit, si l'on peut dire, sur la déconstruction\endnote{Autant certains artistes radicaux du début du XX\up{e} siècle pressentaient les dérives tyrannique de l'ère industriel, autant beaucoup d'artistes du second après-guerre, notamment durant les années 60 et 70, 
%Déjà à cette époque, beaucoup 
avaient conscience de l'impasse consumériste et de l'enjeu écologique induit, et cela s'est manifesté socialement et musicalement par une sérieuse remise en question philosophique, laquelle soit dit en passant reste toujours d'actualité. Ceci expliquant cela, les premiers déconstruisent la tonalité, tandis que les seconds contestèrent la suprématie dodécaphonique et sériel d'avant-garde.} de la tonalité. En effet, celle-ci va déterminer ce que sera la musique de sons d'aujourd'hui, notamment  par la valorisation des sons dits extra-musicaux, par l'invention de systèmes dits atonales jusqu'à l'impasse que fut le sérialisme intégral, par une approche analytique de plus en plus précise allant au cœur du son, et par les tentatives de taxonomie axiomatique du phénomène sonore et musical. De fait, la `reconstruction' prendra des formes les plus inattendues et souvent les plus surprenantes, comme par exemple des concerts de `bruits', de percussions, de synthétiseurs de toutes sortes selon leur degré de technicité, des installations et des \textit{happenings} mettant à rude épreuve les habitudes du public plus ou moins mélomane, provoquant tantôt indignation et fanatisme, tantôt incrédulité et fascination\endnote{Ces comportements extrêmes s'estompèrent progressivement par la connaissance et l'habituation, une forme de `déjà vu' ou `entendu', voire de `on-ne-veut-pas-ou-plus-en-entendre-parler' les désamorçant, par l'épuisement des possibilités combinatoires inouïes (si tant est que cela soit possible). Il doit cependant bien en rester quelques unes à découvrir ou à re-découvrir. Tout cela ne surprend plus grand monde de nos jours, notamment par conformisme, même si la musique reste un puissant véhicule d'affects.}. 
%voir le dedain et l'hilarite

%à ce propos il serait intéressant  d'établir une généalogie des genres et des styles
%Pour mémoire, et dans les grandes lignes, la déconstruction de l'ultrachromatisme wagnerien débouchera sur le sérialisme integral de Stockhausen, en passant par le dodécaphonisme shoenbergien, le bruit au sens de son complexe à hauteur indéterminé s'invitera de plus en plus dans le champ du possible, de Russolo et de Varèse jusqu'à Lachemann, en passant par le spectralisme  de Grisey, pour les musiques écrites.
\bigskip

Dès les années 50, de nouvelles façons de composer intègrent une part d'indétermination plus ou moins contenu, tant sur la structure que sur la forme\endnote{La structure et la forme en musique peuvent être compris de différentes manières.\\ \indent Par exemple, pour John Cage, la \textit{structure} est la divisibilité d'une œuvre, pendant que la \textit{forme} en exprime la continuité, autrement dit la \textit{morphologie de la continuité}. Cette continuité s'articule selon une \textit{méthode} qui s'applique sur le \textit{matériau} en tant qu'éléments structurants.\\ \indent Pour ce qui nous occupe, la structure est un processus immanent, c'est à dire la \textit{fractalisation} d'une caractéristique liée à la nature du matériau, tandis que la forme est un processus transcendant, au sens où le tout dépasse la somme de ses parties vers l'expérience d'une intention manifeste ou d'une limitation extrinsèque. Autrement dit, la forme est le résultat ou le phénomène émergeant d'un processus dépendant du matériau structurant et de son environnement.}, sous forme de degré de liberté accordé à l'interprète. Nous pouvons distinguer trois approches ou techniques développés depuis cette date. La première approche est une technique de composition faisant intervenir la notion d'aléatoire contrôlée. Cela consiste à donner une liberté d'interprétation suivant les indications du compositeur, de façon à générer une complexité visant un effet global déterminé. Cela concerne principalement les éléments structurant de l'œuvre. Une autre technique concerne les formes dites mobiles -- non sans la référence aux mobiles d'Alexander Calder --, pour lesquelles le ou les interprètes doivent combiner, selon les indications plus ou moins explicites du compositeur, des fragments musicaux prédéterminés. Puis une approche plus global sous le terme d'indétermination, notamment promu par John Cage, usant de `partitions' graphiques, textuelles, dont le but est dans un premier temps de stimuler la créativité des acteurs, pour dans un second temps évaluer l'interaction entre l'interprète et l'objet médiateur proposé par le compositeur, et ce, non pas pour exprimer quoi que se soit, mais pour exemplifier des processus générés par l'œuvre.

Ces concepts sont des modalités d'improvisation orientée (ou pas) -- qui ont pour certaines d'entre elles existé bien avant les années 50, mais de façon anecdotique et peu ou pas développées -- qui servent un dessein compositionnel, et reposent apriori sur le potentiel créatif du ou des interprètes.  Aussi, ces concepts doivent être distingués (1) des procédés d'écriture faisant appel aux mathématiques probabilistes et statistiques ou au hasard, permettant l'organisation musicales de système plus ou moins complexe, et entièrement géré par le compositeur, et (2) de l'improvisation qui ne nécessite aucune préparation particulière hormis la mise en pratique d'un savoir, d'une théorie, d'une pratique, soumis à l'inspiration du moment. %souvent repose sur une harmonie tonale, voire sur une liberté totale concernant le free jazz. 

\bigskip

%%À la marge de tous ces événements, 
%%L'œuvre ouverte est un pan
%À partir des années 60, quelques artistes réfléchissent sur différentes manières de composer, d'interpréter, et d'écouter de la musique, et proposent de nouveaux concepts en ce sens, tels que la \textit{moment form}, la `musique expérimentale', et l'œuvre ouverte.
%
%%envisagent le concept d'œuvre ouverte, aussi appelé `musique expérimentale'. 

À partir des années 60, quelques artistes envisagent le concept d'œuvre ouverte. L'œuvre ouverte s'inscrit dans les modalités d'improvisation précédemment évoquées\endnote{Petite précision sur les modalités d’improvisation versus l’improvisation. Bien que la différence entre l’improvisation et la composition n’est pas toujours clair, l’idée ici est de discerner l’improvisation en tant qu’œuvre ou attitude, et les modalités d’improvisation en tant que procédé compositionnel ponctuel et structuré. Ainsi, lorsque l’on parle d’improvisation en général, il est question d'improvisation telle que la pratique les jazzmen, telles les cadences virtuoses, les \textit{Jamming sessions}, voire le \textit{soundpainting}, et j’en oublie. Ces pratiques sont bien entendu convenues, souvent soutenues, reposant sur des codes préexistant bien définit mais refermés sur eux-mêmes en tant que concept. Cela dit, ces pratiques peuvent très bien s'inscrire, s’ajouter, en termes de modalités d’improvisation dans un processus d’écriture, et plus largement d'initiation, de développement, ou d'exploration, voire de construction d'un objet sonore dans le cadre qui nous intéresse.}, avec la volonté selon %Umberto Eco
de proposer des formes complexes, voire ambiguës, en jouant avec \textit{une pluralité de signifiés qui coexistent en un seul signifiant}. %, selon Umberto Eco in L'œuvre ouverte
Plus généralement, l'œuvre ouverte consiste à proposer une action vers l'inconnue, dans le cadre de ce qui est déterminé. 
Bien que l'œuvre ouverte soit un concept en construction, entre une visée utopique remettant parfois en question la notion même d'œuvre d'art -- \textit{les œuvres cesseront d'être des fins en soi, pour devenir des fenêtres sur le monde} --, et l'intention de réduire, voire d'abolir, l'écart \textit{entre celui qui produit et celui qui reçoit}, elle est cette composition suggérant à l'interprète une possible action, une ouverture à un espace de jeu\endnote{Au lieu de proposer une indication précise, que l'interprète réalisera selon sa propre compréhension de l'œuvre, et parfois à l'encontre de l'intention que porte ladite indication, le compositeur ouvre un champ de possibilités, posant les jalons d'un univers que l'interprète incarnera, et dans lequel il devra lui-même s'ouvrir à l'indetermination comme une expérience, libéré de toute intentionnalité (sauf indication contraire), évitant de la sorte toute `trahison' de la part de celui-ci et se repositionnera  dans le respect des prescriptions de l'auteur, et ce en toute liberté.}. 

Par définition, l'œuvre ouverte est la somme d'événement, voire la somme d'individu, d'une situation donnée, comme l'expression de cette situation par la médiation de l'art en question. C'est un processus, intégrant \textit{a priori} et \textit{in situ}, une potentielle ou effective contextualisation.

%Pour la petite histoire, ce concept s'est construit sur une idée ... aléatoire controlé ...  écriture musicale versus free jazz

%il faut distinguer l'oeuvre ouverte  des forme mobiles, aleatoire controle, stochastique,

%Ainsi, dans l'art sonore, certains compositeurs ont tenté avec plus ou moins de succès une reconstruction interactive avec les interprètes et parfois même avec le public.
 %\bigskip
%Paradoxalement, ces concepts s'adressent à une classe d'avant-garde 

\section{\textit{Ni pour, ni contre, bien au contraire}}

Concernant ce \textsc{xx}\ieme ~siècle, il y eut deux ruptures, plus radicales et plus révolutionnaires que la déconstruction du modèle traditionnelle, %notamment par les expressionnistes, 
qui s'est traduit dans un premier temps par deux courants antagonistes remettant en cause les conventions et les contraintes idéologiques, esthétiques et politiques de leur époque, que furent les futuristes %\endnote{} 
et les dadaïstes\endnote{Le 5 février 1915, Hugo Ball et Richard Huelsenbeck rédigent le premier Manifeste dada sous forme de tract:  « \textit{Nous ne sommes pas assez naïfs pour croire dans le progrès. Nous ne nous occupons, avec amusement, que de l’aujourd’hui. Nous voulons être des mystiques du détail, des taraudeurs et des clairvoyants, des anti-conceptionnistes et des râleurs littéraires. Nous voulons supprimer le désir pour toute forme de beauté, de culture, de poésie, pour tout raffinement intellectuel, toute forme de goût, socialisme, altruisme et synonymisme.} »} autour de la première guerre mondiale. Les premiers promurent le progrès telle une démarche avant-gardiste mêlant urbanisme, machines et vitesse, tandis que les seconds contestèrent l'ordre établi d'une civilisation absurde par la dérision et la liberté absolue. 

Et dans un second temps, après la seconde guerre mondiale, certains compositeurs s'interrogent et interrogent l'auditeur et l'auditoire sur la manière de percevoir le monde sonore, dans un contexte de recherches bénéficiant de l'émergence d'une transdisciplinarité, notamment autour de l'intelligence artificielle, regroupant les sciences dites cognitives, incluant la psychoacoustique, la philosophie en termes de phénoménologie et d'ontologie, l'ethnomusicologie et les neurosciences, pour ne citer que ceux-là.

\subsection{Manifeste}

Autant, dans les années qui précédèrent la première guerre mondiale, les futuristes `surfaient' sur la vague du progrès, en tant que mouvement d'avant-garde adulant la vitesse sous toutes ses formes, refusant l'héritage du passé et tourné vers l'avenir, prêchant une réogarnisation du monde autour d'idéaux tels que la fusion de l’homme avec la machine, le fantasme d'immortalité par la technologie, avec la préfiguration de la cybernétique. Autant les bruits générés par les machines dans l'espace urbain et industriel suscitèrent l'intérêt d'un certain Luigi Russolo, qui en 1913 publia le manifeste futuriste \textit{L'Art des bruits}, %(\textit{L'Arte dei rumori}), 
dans lequel il théorise une façon de composer avec ce qu'il appelle les sons-bruits. 
Autant, durant la première guerre mondiale, les dadaïstes promurent l'anti-art sous toutes ses formes. Ces derniers rejettent alors toutes règles -- une sorte de \textit{tabula rasa} -- et ouvrent un espace de liberté totale, pourvu qu'elle transcende ou critique toutes les contraintes, toutes les lois, toutes les conventions, instaurées notamment par l'avant-garde et l'académisme. 

Retenons que ces mouvements étaient fortement politisés et provocateur, flirtant avec le fascisme pour les uns, et l'anarchie pour les autres, voire le communisme pour le reste. %, ils étaient tous révolutionnaires. 
Ces mouvements se sont revendiqués comme un art de vivre, et de nombreux émules ont prolongés l'idée artistique première et l'ont pour certains réactualisés dans la seconde moitié de ce \textsc{xx}\ieme ~siècle, en tant que mouvement autonome tel que Fluxus\endnote{Fluxus fut un mouvement --  s'il en est, d'arrière-garde des années soixante et soixante-dix (héritier de Marcel Duchamp et de John Cage) -- particulièrement intéressant dans le cadre qui nous occupe et qui illustre une tentative de changement de paradigme, qui avec le recul,  nous donne l'opportunité d'en comprendre les difficultés et les enjeux. D'abord, Fluxus fut une appellation par laquelle des artistes pouvaient expérimenter un art holistique  d'influence dadaïste, tant dans la forme (fusion des arts) que dans le fond (rejoignant la vie ordinaire).
Pour Fluxus, l'art -- sous toutes ses formes -- ne doit plus être un produit évalué, encadré, convenu et classifié, dans un jeu de rituels étriqués et arbitraires.  Il s'agit plutôt d'explorer au delà de ces frontières et sans frontière. 
Et par extension, à l'encontre du produit fini (le contenu), l'art devient action (le contenant) en impliquant directement le public.
\textit{Fluxus nous enseignent que l'art est peut-être autre chose que les formules héritées du passé, que l'art peut engendrer la vie, et la vie devenir de l'art.}
Bref, une contre-culture où tout est art, pour finalement prendre une attitude de non-art, d'anti-art.\\ \indent Il n'y eut aucun consensus à son devenir, et les directives de George Maciunas (initiateur du projet) étaient loin de faire l'unanimité. Les divergences politiques, relativement forte à l'époque, que véhiculent certains membres de Fluxus par l'action,
éloignent le projet de son but premier, pour devenir l'instrument d'une lutte contre `l'impérialisme culturelle'. Condition \textit{sine qua non} revendiquée par ceux-là. \textit{Il s'agit surtout de lutter contre l'académisme, contre le système du vedettariat, contre l'art élitiste}, pour les autres.\\ \indent Cependant, Fluxus n'était pas une communauté ou un groupe vivant ensemble, mais une bannière sous laquelle quelques artistes ont essayé d'insuffler une autre manière de considérer l'art, d'ouvrir une autre conscience de ce peut être l'art, et qui s'est limité à un public restreint, initié et de fait élitiste (ironie du sort); et s'est dissipé dans l'indifférence totale, comme le fut le dadaïsme en son temps.}\label{efn:fl} par exemple  -- précédé par le surréalisme\endnote{Entre poésie et arts plastiques, la musique surréaliste se définit plus comme une insurrection dadaïste, une totale liberté, qu'une exploration, une expression, de l'inconscient ou onirique.} de l'entre deux guerres -- ou en tant que démarche artistique avec notamment la musique concrète. Cent ans après, l'héritage musicale futuriste est omniprésent chez la plupart des compositeurs, bien que certains se revendiquent volontiers du dadaïsme dans l'esprit, l'un n'empêchant pas l'autre. 

%Musicalement, cela s'est traduit pour les uns, avec le manifeste futuriste de Luigi Russolo en 1913 \textit{L'art des bruits}, et pour les autres avec le manifeste littéraire de Hugo Ball et Richard Huelsenbeck en 1915 sous forme de track \textit{Dada soulève tout}\endnote{Dada = \textit{Nous ne sommes pas assez naïfs pour croire dans le progrès. Nous ne nous occupons, avec amusement, que de l’aujourd’hui. Nous voulons être des mystiques du détail, des taraudeurs et des clairvoyants, des anti-conceptionnistes et des râleurs littéraires. Nous voulons supprimer le désir pour toute forme de beauté, de culture, de poésie, pour tout raffinement intellectuel, toute forme de goût, socialisme, altruisme et synonymisme.}}
%Le début du \textsc{xx}\ieme ~siècle marque deux courants contaistataire 

%\subsubsection{Les futuristes}
%\subsubsection{Les dadaïstes}

\subsection{Discontinuité}

%\subsubsection{\textsl{Moment form}}

C'est encore à partir des années 50, tandis que des chercheurs s'intéressent et tentent de comprendre comment nous percevons la musique, grâce notamment au développement du cognitivisme et, plus tard dans les années 60, des neurosciences, que certains compositeurs commencent à s'interroger sur la perception même du son, et à s'émanciper de la musique `expressive', de l'objet sémantique\endnote{La sémantique musicale -- à l'instar du langage -- est la relation que l'on établie entre le phénomène ou l'objet en tant que tel et le symbole auquel celui-ci est rattaché.}. En effet, pour ces compositeurs, il s'agit de diriger, de focaliser l'écoute sur le (ou un) son ou l'objet sonore (ou un objet sonore) en tant que tel et dans son atemporalité -- qui peut être défini comme la focalisation perceptive d'une durée infinitésimale, dans le but de tendre vers l'éternité du moment présent -- ou autrement dit sans mémoire. D'aucuns parleront d'immersion sonore. Dans tous les cas, cela relève d'un acte volontaire.

%notamment Karlheinz Stochausen 
%
%. Cet aspect sera développé sous différent approche telle que la \textsl{moment form} et la `musique expérimentale'. Il s'agit de diriger, de focaliser l'écoute sur le son ou l'objet sonore en tant que tel et dans sa temporalité.  
%
%L'activité cognitive induite est encore un sujet d'étude et de recherche
%
%La temporalité en musique est un phénomène lié à la manière de rompre une relative continuité

La \textsl{moment form} -- idée formalisée par Karlheinz Stockhausen et étroitement liée à l'idée de forme mobile  -- et la `musique expérimentale' -- notamment selon l'approche philosophique de John Cage qui considère la musique comme un phénomène naturel avant d'être l'objet d'un démiurge -- proposent ainsi des pistes de prospection sur l'acte de percevoir (de) la musique autrement. Ceux-ci intègrent un paradigme d'écoute qui consiste à considéré l'objet sonore pour ce qu'il est, dans sa nature ontologique, son essence, sans lien aucun %-- du moins si l'auditeur ne focalise pas son attention dans cette direction -- 
avec ce qui peut le suivre et ce qui l'a précédé. 
En d'autres mots, il s'agit de considérer une sériation d'objets sonores comme une succession de moments %dont chaque section est
% \textit{moment form} --, lesquels répondent à la définition de 
%%\textsl{In other words, the} \textit{moment form} \textsl{is a seriation of moment defined as 
 en tant qu'unité musicale autonome. %et se distingue des autres sections dans une logique temporelle discontinue.  
%\textsl{The idea of autonomous musical units is, in order to reduce the perception toward the smallest slice of time, to tend toward eternity}. 
L'idée d'unité musicale autonome entre dans le cadre de la théorie de la \textit{gestalt}\endnote{La théorie de la \textit{gestalt} est l'ensemble des processus de l'esprit et des sens mis en œuvre sur notre relation avec le monde, sur sa représentation. Elle tend à simplifier ou à optimiser notre perception, non pas par une accumulation d'éléments groupés ou discriminés reliant cette perception, mais par la recherche d'une forme simple ou connue liée à cette perception. Ces processus agissent en temps, et par conséquent sollicitent la mémoire sur ses différents plans simultanément, impliquant la mémoire à court terme, à long terme, déclarative, non déclarative, sensorielle, sémantique, etc. }\label{efn:gestalt}. %rend compte de la focalisation perceptive d'une durée infinitésimale, afin de tendre vers l'éternité du moment présent.
La forme est alors de type mobile, tant que les moments ne sont pas reliés entre eux ou peuvent l'être suivant une esthétique compositonelle prédéfinit et bien entendu de façon non hiérarchique, et intègre de la sorte un potentiel de \textsl{`never-ending process'}.
Pour la `musique expérimentale', le concept vise un résultat non prévisible \textit{a priori} et non déterminé pour le moins, dans la mesure où l'exécution est composée en place et lieu d'une composition exécutée. %\subsubsection{Musiques `expérimentales'}

Dans une modalité d'écoute, le silence prend une part privilégiée. 
Le silence en musique est un élément essentiel, qui dans le contexte de discontinuité s'inscrit souvent comme un marqueur séparant deux unités musicales. Aussi, il est l'alpha et l'omega d'une entité sonore ou musicale. Le silence peut alors revêtir plusieurs caractères, suivant sa durée, du micro-silence comme modalité d'articulation au silence absolu\endnote{Le silence dans sa durée et dans son principe revêt un caractère paradoxal car il suppose être, dans l'absolu, l'absence de propagation d'ondes sonores ou de vibration du tympan. Ce qui n'est qu'une vue de l'esprit. %hors d'atteinte car nous ne pouvons nous soustraire au propre son que notre corps produit)
Donc, de manière relative, le silence est l'indicateur d'une information nulle. 
Mais replacé dans son contexte, dans son usage courant, l'expérience subjective du silence prend une dimension autre et peut apparaitre comme silence d'apaisement, de détente, de répis ou silence inquiétant, éprouvant, oppressant; \textit{silence de ceux qui n'ont rien à se dire ou silence de ceux qui se comprennent sans un mot.} Dans le contexte musicale, c'est une respiration, un soupir, une pause, une suspension, une attente, une expectation. %Ou bien alors \textit{l'ensemble des sons non voulus}, selon le point de vue de l'inévitable John Cage pour ne pas le nommer.
}, son placement en temps, et le contexte environnemental. Il serait par trop disert d'en décrire tous les aspects, mais il est possible d'en distinguer au moins trois classes. Le silence de prolongation, celui qui fait résonance avec, et résonner, l'espace en termes de réverbération, comme la queue d'une comète représentant le (ou les) son(s) qui précéde(nt) cet état, et qui persiste(nt) dans notre esprit au delà de son effectivité dans l'espace partagé avec l'auditeur. % de son(leur) existence. 
Le silence de préparation, il est l'inéluctable catastrophe thomienne\endnote{Selon l'idée aristotélicienne d'une stabilité structurelle sous-jacente à toute qualité, l'avènement d'un conflit -- lutte entre au moins deux attracteurs par exemple -- provoquant une rupture, un saut qualitatif, constitue une catastrophe telle que René Thom l'a formalisé et modélisé en 1972 en termes de topologie et de géométrie différentielles, et qui peut être compris plus largement en termes phénoménologiques.}\label{efn:thom}, une tension qui doit se résoudre et qui se résout. Et enfin le silence `vide', qui se \textit{détache du silence de prolongation et ne présage aucune résolution}. C'est le silence cagien, celui qui n'a ni passé, ni futur, celui qui n'est pas vraiment le silence, mais une plongée dans l'imperceptible, le subtil. Il est le présent.
On pourrait ajouter le silence intermédiaire ou le silence élastique, c'est à dire une sorte de fondu/enchainé sans rupture -- conditionné par le contexte formel -- entre le silence de prolongation et le silence de résolution.

%Quant à la `musique expérimentale', principalement définit et incarnée par John cage, le concept vise un résultat non prévisible \textit{a priori} et non déterminé pour le moins, dans la mesure où l'exécution est composée en place et lieu d'une composition exécutée. 
La \textit{moment form}, la `musique expérimentale' et l'œuvre ouverte, consiste alors à exercer, à diversifié, à transcender nos modalités d'écoute, conditionnées par notre héritage culturel, voire atavique et nos croyances; ou plus simplement à considérer le moment présent dans l'attention du monde vibratoire, avec tous nos sens et toute notre intelligence.

\section{Qu’est-ce que la musique?}

Essentiellement, avant d'être de la musique, le phénomène sonore est la perception de significatives fluctuations moléculaires dans l'air (ou dans l'eau pour les amateurs de plongée sous-marine) en termes de pression/dépression par l'appareil auditif, mais aussi ressenti physiquement, par la peau, les organes, impliquant le corps dans sa totalité, notamment par un effet de résonance sur le squelette. 
Cette perception agit comme un attracteur. Selon un seuil contextuel, ce phénomène nous fait tendre l'oreille dans un premier temps, pour en determiner sa nature ou sa source dans un second temps.
L'interprétation de cette perception relève de l'affect, corrélé à la cognition dans sa transversalité.
% cf le corps et l'esprit selon António Damásio dans L'Erreur de Descartes 

Pour autant que je sache, il n'existe pas de définition définitive au concept de musique\endnote{À noter que le concept de musique ou la musique telle que nous la pratiquons dans notre monde soit-disant civilisé, n'est pas aussi prétendument universel, puisque par exemple il n'existe pas de terme équivalent en langue \textsl{Inuit}, mais plutôt le terme générique \textit{nipi} pour le phénomène sonore en tant que tel. De même, pour la tribu des \textsl{Blackfoot}, seul le terme de \textit{saapup} réfère au chant, à la dance et à la cérémonie. En Afrique, les tribus \textsl{Tiv, Yoruba, Igbo, Efik, Birom, Hausa, Idoma, Eggon, Luo} et \textsl{Jarawa} n'ont tout simplement pas de mot pour musique. %\\ \indent De plus, l'universalisme tend à promouvoir l'égémonie imperialisme du système capitalisme qui tend à normaliser la notion de culture à l'encontre des plus en plus rares cultures éthniques
L'universalisme est un concept ethno-centré qui vise plus à démystifier -- voire à démythifier -- une pratique musicale, qu'à comprendre les véritables enjeux humains que la musique met en pratique. 
}$^{,}$\endnote{La musique se présente souvent comme un langage universelle, mais cette `évidence' ne l’est pas. Pourquoi l’art devrait exprimer quelque chose au lieu de n’exister que pour lui-même? Pourquoi recherche à communiquer une émotion, quant celle-ci peut se manifester à tout moment, et de manière impromptu, ou bien quant celle-ci n’est pas le but de l’œuvre, mais plutôt dans une forme de perception autre que celle du message -- si il y a lieu -- de son auteur? Autant de questions en suspend, pour lesquelles aucune philosophie n'apporte véritablement de réponse au delà de son propre champ d'étude.}$^{,}$\endnote{Cependant, une approche étymologique peut nous éclairer sur les intentions signifiées originelle. Ainsi, le terme musique vient du grec ancien \textit{mousikè}, entendu  par `ce qui concerne les muses' ou plus précisément `l'art ou technique des muses'; et dérivé de \textit{Mousa}, littéralement les Muses. Lesquelles \textit{sous la conduite d'Apollon intervenaient comme musiciennes dans les banquets des dieux et charmaient Zeus}.\\%Toutefois, bien que le terme \textit{mousa} signifie `muse', il peut aussi désigné la `science', le `chant' ou la `parole'. 
%De plus, \textit{hè mousikè} ne désigne pas seulement l’art musical -- lequel se traduirait plutôt par \textit{mousikos akousma} -- mais l’ensemble des arts présidés par les muses. Pour finir, il existe une expression utilisant la même racine, \textit{mousikôs anèr} qui désigne une personne lettré et cultivé, instruit des arts que président les muses.
\indent Pour la petite histoire, dans la mythologie grecque, il existe deux versions concernant les muses -- pour chacune d'entre elles est associée un (ou des) art(s) ou une (ou des) connaissance(s) qui lui est (sont) propre --, déesses,  filles de Zeus et de Mnémosyne. La première version fait état de trois filles:
 \renewcommand{\labelenumi}{\arabic{enumi}.} \begin{enumerate}
\item Mélété (muse de la méditation et de l'exercice, de la pratique et du soin); 
\item Mnémé (muse la mémoire); et
\item Aoidé (muse du chant). 
\end{enumerate}
Quant à la seconde version, elles sont neuf:
 \begin{enumerate}
\item Calliope est la muse de la poésie épique et de l’éloquence [trompette];
\item Clio est la muse de l'histoire [trompette];
\item Erato est la muse de la poésie romantique %l'élégie 
[cithare, tambourin, lyre, viole];
\item Euterpe est la muse de la musique (inévitable tautologie) [piccolo double, flûte simple (aulos), hautbois double, trompette];
\item Melpomène est la muse du chant, de l’harmonie musicale et de la tragédie [corne, cor];
\item Polymnie est la muse de la poésie lyrique et de l’harmonie, de la danse, de la géométrie et de la rhétorique [orgue];
\item Thalie est la muse de la comédie [viole];
\item Terpsichore est la muse de la danse %et au chorus théâtral grec 
[lyre, viole, harpe]; et
\item Urania est la muse de l’astronomie et de l’astrologie mais aussi des mathématiques.
\end{enumerate}
À noter que la plupart d'entre elles sont souvent représentées avec un instrument de musique, entre autre attribut.\\ \indent Certaines pistes Proto-Indo-Européennes suggèrent pour \textit{Mousa} la racine \textit{*men-} signifiant selon les sources et les contextes:
\begin{itemize}
\item  \textsl{to think, mind, or spiritual activity;}
\item \textsl{to stay, remain;}
\item \textsl{to stand out, to tower.}
\end{itemize}
}, mais plutôt un cluster polysémique nodal fluctuant\endnote{Cela mérite quelques éclaircissements. En effet, le concept doit se comprendre non pas comme une idée isolée, mais plutôt comme un réseaux de relations entre différent signifiés dans le but d'articuler une compréhension du monde qui se veut à la fois cohérente et fonctionnel dans une certaine réalité, pour le moins, identifiée comme telle. Ce réseau, dans le meilleur des cas, peut se résumer à, ou converger en, un point nodal: le concept est clair. Mais souvent nous avons affaire à une multitude de points nodaux, que l'on nomme cluster et qui induit une  polysémie plus ou moins équivoque et naturellement corrélée  à un contexte. % -- que les linguistes nomment onomasiologie ou sémasiologie, selon le sens de la démarche analytique, respectivement du concept vers le signe ou du signe vers le concept. 
Dans certains cas, le concept se construit et évolue dans le temps passant d'un signifié à l'autre, sans jamais se fixer sur un point nodal déterminé tant la complexité du réseau impliqué l'en empêche. C'est le cas de la musique.}\label{efn:concept}, dans lequel chacun puise sa propre version, vision, souvent suivant le contexte référant, proposant un point de vue autre, parfois jusqu'à la contradiction\endnote{Une énumération, exhaustive ou non, n'est pas le propos qui nous intéresse, cependant j'invite le lecteur à consulter les différents dictionnaires de musique, de Jean-Jacques Rousseau à \textit{The New Grove Dictionary of Music} par exemple, les dictionnaires, encyclopédiques ou non, à différentes époques, et à se poser la question, et aussi à la poser autour de soi, pour se rendre compte de la diversité acceptionnelle que revêt notre compréhension de ce que pourrait, ou devrait, être la musique.}$^{,}$\endnote{Le son et la musique peuvent aussi bien être détournés à des fins redoutables pour ne pas dire nuisibles. Telles que la torture, le contrôle de l'individu, en incapacitant le sujet et en semant la confusion des sens, incluant les infrasons et les ultrasons; ou comme outil destiné au contrôle et à la manipulation des masses, pour augmenter la production dans les usines, ou à des fins mercantiles destinées à rassurer l'utilisateur du lieu ou à stimuler le client potentiel.}.

Cependant, il y a quelques fondamentaux sur lesquels il est possible d'établir un point de vue panoramique. Ainsi, la musique est l'appréciation, l'entendement, du son, tel que tous les sons sont potentiellement de la musique ou musicaux. Aussi, la musique est un phénomène sonore incarné dans l'intention qu'elle porte, soit une dimension spirituelle, au sens initiation, rituel ou développement personnel, et compris dans une `mise en scène' par et pour un agent humain. 
%Elle n'est pas porteuse de sens connu ou reconnu en dehors du conditionnent culturelle en termes symboliques -- même s'il peut y avoir quelques universaux tels que: les sons graves sont telluriques et les sons aigus aériens --, notamment par le cinéma. Au plus, la musique parle d'elle-même autour d'archétype ou de stéréotype compris dans une culture donnée.
Autrement dit, la musique est une construction de l'esprit\endnote{\textsl{However, besides the art music, some musicologists consider the music notably through the songs -- more or less common for all known `small' societies, according to a significative ethnographic corpus -- by the religious,  the formality (evoked in this writing as ritual) and the arousal.  For instance, concerning arousal, the} \textit{ostinato} \textsl{can lead to some kind of trance. More broadly the `attractor(s)' phenomenon within a harmony context (see modality and tonality) gratifies the ear and seduces the audience according to the satisfaction provided by the recognition more or less agreed -- and this can be interpreted in the context such as a lullaby, love, dance, or healing music. But these considerations induce and remain an attempted formulation of a mind construction clearly in the frame of human ethology as societal study, and as asymptotic concepts understood according to the western ethnological point of view.}\\ \indent Ajoutons l'hétérogénéité musicale à l'intérieur d'une même catégorie identifié dans son contexte. Aussi, les similitudes catégorielles relèvent plus d'un biais statistique que d'une réalité objective. Certes, une musique rythmée est plus enclin à la danse, une musique lente à la méditation, mais cela n'est pas systématique. %Il est possible de danser sur une musique lente et/ou non rythmée. Il est tout autant possible de méditer sur des \textit{ostinatos} soutenu rythmiquement. D'ailleurs, l
Par exemple les musiques rythmées invitent souvent un certain public à boire jusqu'à l'ivresse, et d'autres encore comme un stimulant au travail. 
Cela détermine seulement une possible fonction, laquelle n'est pas déterminée par la musique elle-même, mais par la connaissance (acquise par l'expérience) que l'on a d'elle au sein d'un groupe partageant la même culture ou le même `état d'esprit'.}.

Aussi, d'aucuns parleront de musique selon une codification, une complicité, le plus souvent implicite, voire intuitive, suscitant une expectation procurant souvent une grande satisfaction lorsqu'elle est résolue, parfois une grande surprise ou interrogation, avec lesquelles le compositeur ou l'interprète joue.  

%à ce que serait la musique en tant que concept universel. Pour John Cage every thing is music et pour Victor Hugo la musique, c'est du bruit qui pense.
\bigskip

Partant de ces postulats, la musique ne peut être qu'un objet complexe dans son approche analytique, et qu'une approximation dans son approche synthétique, d'autant plus qu'à cela s'ajoute l'intention -- quelle qu'elle soit, délibérée ou non, singulière ou plurielle -- de l'émetteur et du récepteur. 

Vous l'aurez compris, la musique ne peut se résumer au système tonal et tempéré, à une mélodie que l'on siffle ou que l'on chante sous la douche, ou à une pulsation `cardio-émotive'. Encore moins à un object cernable tant notre compréhension du monde est partielle et partiale. 

%Qu’est-ce qu'un bruit?  \textit{Tout est bruit pour qui a peur.}
%  
%\bigskip
%
\section{Pensez la musique autrement}

Comment comprenons-nous l’intention musicale ? L’affect est bien entendu le facteur décisif. L’affect concerne notre propre subjectivité, notre propre culture et notre propre expérience du phénomène sonore et ce, en termes circonstanciel. Ainsi, du \textit{percept} à l'\textit{affect}, phénomènes physiques puis cognitifs, il nous est possible d’appréhender le phénomène musical en termes de formalisation et modélisation; c'est à dire appréhender la structure dans sa directivité et anticiper, prédire son potentiel.  

Bien qu'il y eut pléthore tentatives de reconstruction inouïes dans les années soixante, peu de musique fut expérimentée en tant que processus collectif et collaboratif, intégrant l'improvisation, tant sur la nature des sources sonores\endnote{La nature des sources sonores concerne aussi bien les instruments de musiques que des objets qui n'ont à priori rien à voir avec la musique, pouvant ou devant être détournés de leur fonction première, par exemple verser de l'eau dans un trombone, casser des vitres, faire des bruits avec la bouche tel que le gargarisme, etc. De plus, la radio, l'enregistrement, la sonorisation et les sons de synthèse viennent s'ajouter à cet arsenal acoustique.} que sur la manière de construire, d'explorer un monde sonore avec d'autres. Cependant peu n'est pas rien; et dans une ambiance de lutte de classes, notamment contre le caractère élitiste et hiérarchique de la musique d'avant-garde, 
quelques expérience en ce sens ont été réalisé -- avec plus ou moins de succès en leur temps -- mais perçu peu ou prou en ces termes:
%Alvin Curran -- co-fondateur de \textit{Musica Elettronica Viva} fondé en 1966 à Rome --  témoigne:
 « \textsl{It was like discovering music for the first time... Each of us reveled in discovery of his own inner music -- the source itself. And `harmonizing' with the others became like a drug experience, demanding ever more fluency and intensity. From the beginning of these sessions, there was an awareness that we were dealing with something very serious – very fundamental.} »\endnote{Témoignage d'Alvin Curran, co-fondateur de \textit{Musica Elettronica Viva} fondé en 1966 à Rome.}
%SOURCE -- http://www.alvincurran.com/writings/12%20years%20music%20rome.html
%CURRAN, Alvin, "A guided tour through 12 years of American Music in Rome",
%published in Soundings No. 10, Soundings Press, Santa Fe, 1976.

C'est ce `quelque chose' qu'il nous faut saisir, cette magie de l'instant, de l'inouï, de l'émotion partagée. Penser la musique autrement, c'est participer activement, pour soi et pour les autres, et aussi de partager ces moments d'oubli et de mémoire qui nous habitent, entre l'immanence et la transcendance de notre être.

%Cela reste dans tout les cas une démarche purement philosophique pour laquelle il convient de rester humble. Penser la musique autrement est prospecter au delà du pays fertile, en quête d’un savoir insaisissable avec lequel je vous invite à composer. 

%\bigskip
%scratch orchestra ...

\subsection{Concept} %--> object sonore}

Déjà évoqué en tant que possible `cluster polysémique nodal fluctuant'%\footref{efn:concept}
, le concept -- du latin \textit{conceptus} qui renvoie à l'action de contenir -- est la representation de l'esprit par un signifiant, articulé notamment par inférence selon la position qu’il occupe dans une chaine de signifiants, d'un signifié `gestaltique'%\footref{efn:gestalt}
, appliqué -- par extension à notre propre perception, et par abstraction -- à notre propre compréhension du monde. 
%La relation qu'entretiennent les notions de signifiant et de signifié, introduit en linguistique par Ferdinand de Saussure, et reprit en psychanalyse par Jacques Lacan, serait pour le premier de nature fixe, et apparait pour le second comme instable, et reconnait qu'un même signifiant peut se référer à plusieurs signifiés ou concepts différents selon le contexte.  \textit{Le signifiant ne signifie rien en soi, mais acquiert une signification en fonction de la position qu’il occupe dans une chaîne de signifiants.}
% SOURCE -- Serafino Malaguarnera in https://www.youtube.com/watch?v=CboOVK_cgNc&ab_channel=SerafinoMalaguarnera

En art, le concept est le caractère multiple ou reproductive d'une idée. Ainsi, l'idée prime sur sa représentation. \textit{Ce qui importe n’est ni l’objet proposé en lui-même, ni l’acte de proposition en lui-même, mais l’idée de cet acte}. Autrement dit, le concept est une représentation de l’esprit d’un objet, au sens philosophique du terme, par ce qui le caractérise.
% SOURCE -- Sophie Stévance, Les opérations musicales mentales de Duchamp. De la « musique en creux », 2009, page 5, §11. ---> http://imagesrevues.revues.org/375
Ainsi, l'art conceptuel s'approprie un champ d'exploration, en termes de formalisation et de modélisation, jusque là réservé aux philosophes et aux scientifiques, au delà de l'œuvre d'art en tant qu'objet unique, et au delà du savoir-faire de l'artiste, pour interroger les sens de celui qui en fait l'expérience.
Dès 1913, Marcel Duchamp, influencé et inspiré par le mouvement dada, remit en cause la notion même d'œuvre art, en détournant des objets industriels, se contentant de les titrer, de les signer, et de les exposer dans des lieux habituellement réservés à l'art. Ce fut la naissance du \textit{ready-made}, véritable art conceptuel avant l'heure. C'est la décontextualisation, puis la recontextualisation -- ou anachroniquement et plus précisément la déterritorialisation puis la reterritorialisation -- du signifiant pour un autre signifié, explicité ou laissé à l'appréciation -- voire l'imaginaire -- de l'observateur. 

Puis, après deux guerres mondiales\endnote{Véritable marqueur de notre faillite civilisationnelle.}, vint l'age d'or de l'art conceptuel. On pourrait en dater le début avec \textit{One and Three Chairs} de Joseph Kosuth en 1965 qui, partant du concept de la chaise, propose trois représentations distinctes, dans ce cas sans répétition formelle: \textit{l’ensemble dépasse ainsi l’objet pour approcher le concept} lui-même. 
% SOURCE -- https://blogacabdx.ac-bordeaux.fr/culture64/wp-content/uploads/sites/74/2019/12/EAC35.pdf
Puis la fin -- non de l'art conceptuel puisque désormais \textit{tout art après Duchamp est conceptuel}, mais de son age d'or -- avec \textit{An Oak Tree} de Michael Craig Martin en 1974, qui expose hors de portée du public un verre d'eau, en le désignant comme étant un chêne, détournant ainsi le concept par transsubstantiation. Le texte qui l'accompagne -- qui d'ailleurs fait partie intégrante de l'œuvre, quoique curieusement de manière anachronique -- est particulièrement éloquent à cet égard.  

Évidemment, les œuvres et les artistes cités ici ne seraient en aucun cas représentés à eux seuls l'art conceptuel de ces époques, mais le sont non seulement à titre d'exemples, d'illustrations, mais aussi, sont particulièrement significatives et significatifs dans leur radicalité et leur chronologie, pour y être effectivement mentionnés.

Maintenant, la musique, ou plus spécifiquement l'objet musical, en tant que concept, n'échappe pas à la règle. Au même titre que la représentation mentale du concept par l'objet ou par le signe, la musique conceptuelle est l'image sonore, une forme d'écoute, que met en place notre esprit par la pensée d'une idée -- à propos de la musique -- évoquée ou exprimée notamment par l'œuvre. %En ce sens, l’idée de l’œuvre d’art unique est éludée au profit 
En ce sens, nous pouvons parlé de musiques imaginaires ou imaginées par l'observateur et l'artiste. Comme nous scénographions notre lecture d'un livre, outre « lire entre les \hbox{lignes »}, nous inventons, nous décodons le médiateur musical, selon une sémantique du signe -- voire une sémiologie musicale\endnote{Pour autant, il n'y a pas de véritable message dans l'art. Le signifié, au plus, suscite des impressions et des émotions, il dépend de la culture, de la sensibilité, du contexte spatial et temporel, de l'expérience, de chacun.  De plus, le processus créateur du compositeur consiste principalement à communiquer une idée musicale -- qui peut être votre.
Certes, il peut y avoir quelques universaux -- notamment exploités et entérinés par analogie (tel que le \textit{mickeymousing}) par le cinéma -- tels que: les sons graves sont telluriques et puissants,  tandis que les sons aigus sont aériens voire magiques, mais cela ne nous dit rien sur le signifié lui-même.
%\textit{Mais l'art, ne délivre pas de message. « Signifier » en musique ne veut pas dire faire comprendre : l'œuvre musicale veut simplement communiquer une sensibilité personnelle présentée à nos sens mais accessible bien au-delà. Cette communication prend d'ailleurs chez certains compositeurs l’allure d’une intégration de plus en plus complexe de matériaux disparates.}
\\ \indent Pour mémoire, la sémiologie est l'étude des pratiques signifiantes ou d'un système signifiant. Cela concerne les symboles ou les signes destinés à évaluer, à transmettre ou à partager une information, un concept, telle que de la musique en ce qui nous intéresse. Dans ce cas, il nous faut inclure l'intention du compositeur selon des stéréotypes ou des archétypes, généralement culturels, et la compréhension que peut en faire le potentiel auditeur. La sémiologie consiste alors à comprendre un élément d'information. Autrement dit, ce qu'un signe, ou un ensemble de signes, provoque, ou peut provoquer dans notre esprit.}
 -- qui nous ai proposé. L'œuvre est donc ce qui provoque, stimule, notre imaginaire, souvent dans le(s) domaine(s) qui lui(leur) est(sont) attribué(s). 

Présentement, ce n’est plus l’œuvre d’art qui crée le public, c’est le public qui crée l’œuvre d’art. Désormais, nous pouvons créer l'œuvre, non pour un public, mais pour nous élever par l'acte créatif que le concept nous permet d'expérimenter.

%se libérer de l’intentionnalité dans l’acte créateur, de la notion de goût par l’indifférence esthétique, ainsi que de l’idée de l’œuvre d’art unique qu’il soumet à la loi du « définitivement inachevé »
%L'idée musicale est le processus mis en place par le compositeur 

\bigskip

Penser la musique autrement, c'est prendre conscience que tout peut être musique\endnote{Le phénomène peut effectivement être considéré au delà de nos propre sens, tel que les infrasons et les ultrasons, la transposition de `chants' d'insectes, la sonification du fond diffus cosmologique (la sonification consiste à transposer, à adapter, à transformer, des mesures ou des données afin de les rendre perceptible dans le champ du sonore), etc; ou bien autonome tel que le \textit{soundscape} -- \textit{i.e.} les bruits de la nature, les bruits urbains, ... -- les musiques algorithmiques, dites aussi génératives ou procédurales, reposant sur une grammaire, une combinatoire, une modélisation, voire une fractalisation, et j'en oublie.}, dès lors que l'on se laisse porter par notre capacité à expérimenter, à interpréter, et interpeler nos sens. Elle est concept, et à ce titre nous pouvons en envisager les ressorts didactiques selon par exemple un point de vue sémasiologique ou onomasiologique\endnote{La sémasiologie et l'onomasiologie sont deux termes plutôt rare que j'ai découvert au cours de mes lectures en linguistiques concernant la lexicologie, et qui pourraient s'appliquer lorsque l'on parle de concept en art. Ainsi, le premier rend compte de la signification d'un mot, voire d'une œuvre, en partant du(des) signe(s) vers le concept, tandis que le second, en partant d'un concept, étudie ses désignations, non seulement par le signe ou par le mot, voire par une œuvre, mais aussi par la grammaire et le contexte.}, avec l'intention d'aller au delà de l'expérience même, et de composer, d'inventer, et de transmettre; bref, faire de l’œuvre d’art une aventure de l’esprit.

%\bigskip 
%\bigskip
%art classique (figuratif) -- art moderne (technique) -- art contemporain (récit) 
%modalité/tonalité -- dodécaphonisme/serialisme/minimaliste/spectral ... -- œuvre ouverte ...

\section{Objet sonore}

Bien que le concept d'objet sonore reste une tentative de formalisation\endnote{Difficile de ne pas cité l'incontournable \textit{Traité des objets musicaux} de Pierre Schaeffer publié en 1966, quant on parle d'objet sonore. Cependant, celui-ci se restreint à dessein qu'à la musique dite acousmatique, et malgré le caractère érudit de l'ouvrage, il subsiste quelques écueils, pour le moins problématique lorsque l'on parle de musique en général. Par exemple sur la notion d'objet `convenable' dont la définition reste pour le moins abstraite et ouverte. Aussi, la notion d'écoute réduite, qu'il qualifie de condition \textit{sine qua non} à tout objet sonore, consisterait à faire abstraction du message causal du son et de sa contextualisation, fruit de l'\textit{épochè} (sic) -- qu'il distingue de l'objet musical qui fait intervenir une référence culturelle -- pour en faire son analyse typologique. Il choisit la méthode par répétition -- reconnaissant la difficulté d'une telle démarche qui contribuerait à nous éloigner de toutes causalités -- induisant l'enregistrement de l'objet d'étude, confinant ce dernier dans la musique concrète.\\ \indent Rappelons que Pierre Schaeffer forgea ce terme de musique concrète en 1948, pour disparaitre dans les années soixante, remplacer par le terme plus générique de musique électroacoustique. La musique concrète consiste à faire de la musique empiriquement, \textit{i.e.} concrètement, à partir de sons enregistrés, sur vinyl puis sur bandes magnétiques, pour être manipulés, transformés, mixés, en une œuvre pour le moins inouï.}  -- voire de modélisation -- d'un phénomène sonore existant ou imaginaire, irréductible ou complexe, structurant et structuré, je me propose d'en identifier, d'en énoncer quelques axiomes ou éléments irréductibles comme point de départ vers une nomenclature, une exploration, qui se doit d'être propre à chaque formation. 
Ainsi, le principe qui régira toute approche performative sera basé de près ou de loin sur l'objet sonore en tant qu'unité, \textit{gestalt}, forme. 

Le concept d'objet sonore n'est pas clairement posé tant il dépend de son objet téléologique. Ici, il s'agit plus de construire cet objet téléologique à partir d'objets sonores que le contraire.   
Aussi, il me parait plus simple et plus pertinent d'aborder l'objet sonore par une analogie toute relative emprunt à la linguistique et plus spécifiquement à la phonologie -- voire la phonètique. Ainsi, il est possible de distinguer l'unité irréductible en tant que phonème, que l'on peut rebaptiser `sonème'. Puis, le morphème qui est composé de sonèmes pour former une unité `supérieur', d'ordre culturel, symbolique, musical. Enfin, la syntaxe, qui satisfait un ensemble de règles combinant, associant, articulant les sonèmes et les morphèmes, visant un effet, une image, un concept, entre différents sonèmes ou morphème, en `harmonie' (superposition de sonèmes/morphèmes) ou en temps (enchainement de sonèmes/morphèmes). À partir de là, les motifs et les phrases peuvent constituer un ordre supérieur d'objets sonores. De la micro-forme à la macro-forme, cela pourrait se résumer ainsi:
sonème $\rightarrow$ morphème $\rightarrow$ motif $\rightarrow$ phrase $\rightarrow$ section $\rightarrow$ œuvre. Ces termes pouvant bien entendu se recouvrir jusqu'à ne faire qu'un.

Ceci constituera notre définition de l'object sonore.
\bigskip

%il est possible d'identifier quelques axiome ou elements irreductibles ...
%\noindent 
\textit
{Nous entendons tous, mais nous n'écoutons pas les même choses.}
%\smallskip
%\textit{La recherche d’une typologie « absolue » est illusoire.} Cependant, 
En tout état de cause, il est possible de décrire un objet sonore selon des caractéristiques utiles que l'on souhaite formaliser ou formuler dans le but de disposer d'un langage, et d'un objectif, commun au groupe impliqué. Loin de prétendre à une description absolue et exhaustive, je me contenterai d'énoncer quelques pistes, usant de la terminologie d'un compositeur, sans que cela soit gravé dans le marbre. 
\bigskip

De l'objet sonore:

\begin{longtable}{lrcccl}
\texttt{[1]} & durée courte & $\leftarrow$ & -- -- -- -- -- -- -- & $\rightarrow$ & durée longue \\
\texttt{[1]} & continu & $\leftarrow$ & -- -- -- -- -- -- -- & $\rightarrow$ & discontinu \\
\texttt{[2]} & \textit{quantum} & $\leftarrow$ & grain & $\rightarrow$ & lisse \\
\texttt{[2]} & régulier & $\leftarrow$ & -- -- -- -- -- -- -- & $\rightarrow$ & chaotique \\
\texttt{[2]} & son pur & $\leftarrow$ & son `cannelé' & $\rightarrow$ & bruit blanc \\
\texttt{[2]} & monophonie & $\leftarrow$ & -- -- -- -- -- -- -- & $\rightarrow$ & polyphonie \\
\texttt{[2]} & hauteurs définis & $\leftarrow$ & masse fixe & $\rightarrow$ & masse variable \\
%grave & $\leftarrow$ & -- -- -- -- -- -- -- & $\rightarrow$ & aigu \\
\texttt{[3]} & \textit{rallentendo} & $\leftarrow$ & \textit{a tempo} & $\rightarrow$ & \textit{accelerando} \\
\texttt{[4]} & hyper-localisation & $\leftarrow$ & -- -- -- -- -- -- -- & $\rightarrow$ & ubiquité \\
\texttt{[4]} & près & $\leftarrow$ & -- -- -- -- -- -- -- & $\rightarrow$ & loin \\
\texttt{[4]} & petit espace clos & $\leftarrow$ & -- -- -- -- -- -- -- & $\rightarrow$ & grand espace clos \\
\end{longtable}

\noindent {\small \texttt{[1]} unité, segment temporel, \texttt{[2]} qualité spectrale, densité de masse, \texttt{[3]} agogique, \texttt{[4]} spatialisation.}

%Les qualités intermédiaires sont des alternatives et non des passages obligés.

\bigskip

De l'articulation:

\smallskip

\noindent Étroitement lié à la notion de dynamisme, 
de l'événement sonore irréductible à l'ensemble de l'œuvre,
la dynamisation du fait musical
s'inscrit dans un contexte expressif, discursif, voire
 contemplatif. 
 Les éléments dynamiques sont entendus selon leurs relations dans le temps, avec un avant et un après.
Ainsi, le découpage formel du flux musical est reconnu en temps par le mode d’articulation.
L'articulation se définit alors en termes d'accentuation, de transition, de ponctuation, et de caractère.
%voici quelques linéaments concernant la dynamisation du fait musical, en partant de l'événement sonore irréductible à l'ensemble de l'œuvre,
%Cela se traduit par la description des différents procédés dynamiques afin de les
%inscrit dans un contexte expressif, contemplatif, voire discursif. 
%Dans cet ordre d’idée, les éléments dynamiques peuvent aussi, le cas échéant, être classifiés selon leurs relations dans le temps, avec un avant et un après. Ainsi, le découpage du flux musical est reconnu en temps par le mode d’articulation.

%Cela se traduit par une taxinomie des modes d’accentuations, lesquels vont s’inscrire par extension à la phrase, au `bloc xénakien', autrement dit à l'objet sonore d'ordre supérieur,
 %par une structuration en temps -- `mélodique' -- des énoncés en groupes rythmiques 
 %et à la morphologie holistique de l’œuvre par la perception de la succession plus ou moins régulière des accents démarcatifs en terme de mouvement.

%Aussi, les blocs thématiques, pouvant être définit en mouvements identifiés par leur caractère et délimités par une catastrophe au sens thomien du terme, 
%peuvent s’inscrire dans ou être articuler selon une terminologie narrative telle que l’orientation, le climax, et la coda.

\begin{enumerate}
\item (a) \myuline{Accent} 

associé à un objet sonore en termes d’entité (sonème ou morphème).

\begin{itemize}
\item Par le mode d’attaque lié au mode d’excitation ou au mode de jeu.
\item Par l'allongement de l'objet sonore en termes de durée.
\item Par l’énergie déployée, exprimée le plus souvent en termes de nuance. 
\item Par le timbre résultant qui combine les points précédents en y incluant l'aspect vertical par la superposition de timbres.
\end{itemize}
Il va sans dire que tous ces points ne sont pas exclusif, et au contraire sont corrélés entre eux.
 \end{enumerate}
Les trois points suivants s’inscrivent dans un rapport inter-relationnel d’ordre supérieur impliquant au moins deux objets sonores en temps.
\begin{enumerate}%[resume]
\item (b) \myuline{Accent}
\begin{itemize}
\item En jouant sur la durée de l'objet sonore en termes d'anticipation ou de retard.
\item Par le registre, qui d’un point de vue structurel peut être lié à la tonalité ou à la modalité (au sens d’effet plus ou moins attractif d’un ensemble de sons relatif à un son unique) ou plus généralement lié à la tessiture de l'instrument, autrement dit à l’intonation.
\item Par un micro-silence, induit par les accents précédemment énumérés, ou délibéré.
\end{itemize}

En bref, les accents possibles dépendent de l'\textit{instrumentarium} et de leur rôle en tand que tel, c’est à dire si l’on a affaire à un accent démarcatif (marquant une `frontière' entre deux objets), à un accent d’insistance (mise en relief d'un élément constituant l'objet) ou un accent contrastif (mise en évidence d'une unité par rapport à une autre).

\item \myuline{Transition}

La transition est un objet sonore -- qui en présente par conséquent les mêmes caractéristiques --  dont la fonction est d'articuler, de relier, deux objets sonores en temps dans une logique discursive.

 \end{enumerate}
Quant aux deux derniers points, l'articulation se joue -- à l'instar d'un langage usant par définition de formes symboliques inventoriés et  d'une syntaxe de composition répétée à plusieurs niveaux et d'une complexité plus ou moins élaboré d'ordre culturelle -- sur une structuration d'ordre supérieur et concerne moins la musique elle-même que la manière de l'interpréter, ou plutôt devrais-je dire de l'articuler, selon une effective transmission orale ou écrite, voire suggérée ou laissée à l'appréciation des acteurs. 

\begin{enumerate}[resume]

\item \myuline{Ponctuation}

Étroitement lié au caractère,  emprunt d'un concept linguistiques, la ponctuation en musique concerne l'articulation prosodique en termes de respiration, d'intonation, et de cadence. Cela s'applique généralement au niveau du motif ou de la phrase au sein d'une section.

\item \myuline{Caractère}

Le caractère contextualise l'état d'esprit dans lequel l'œuvre musicale doit être jouée ou vécue. Au delà du caractère proprement dit -- qui peut être par exemple triste, gai, enjoué, agité, léger, etc. -- et en plus des différents modes d'articulation, la pulsation, le tempo, voire le mètre, participent aussi à la manifestation dudit caractère.

\end{enumerate}

\bigskip

De la syntaxe:

\smallskip

 \noindent{Le principe d'organisation de la musique est fondé et repose sur la dialectique répétition/variation, comprise entre le silence et le contraste. Ainsi chaque œuvre suit ses propres règles.}

%%\noindent 
% silence $\leftarrow\rightarrow$
%répétition 
%%$\leftarrow\rightarrow$ micro-variation 
%$\leftarrow\rightarrow$ variation 
%%$\leftarrow\rightarrow$ zone d'indétermination 
%$\leftarrow\rightarrow$ contraste
%
%%\bigskip 
%
%\noindent \textbf{Chut! Écoutez, il y a quelques choses à entendre.}

\bigskip

Du statisme:

\smallskip 

\noindent Notion plutôt contemporaine émergeant vers la fin des années 50, désignant les musiques à caractère continu ou invariant, qui s'étirent dans le temps de telle façon que le materiel semble ne pas ou ne plus évoluer.  Aussi, le statisme est une modalité d'écoute. 
Bien sûr, dans le domaine musical, le statisme est une vue de l'esprit, du moins dans l'absolue, et toute modalité d'écoute sous-entende un dynamisme plus ou moins subtil, quelquefois réduit à une intention, tant dans l'articulation du discours musicale (quant celui-ci à lieu), que dans la perception psycho-acoustique (à laquelle nous n'échappons pas).

En effet, pour certains auditeurs, une œuvre sérielle peut paraitre statique car il n'y a aucune directivité tonale, excluant tous repères discursif en ce sens. À contrario, la musique dite `drone'  -- statique par définition, à la croisée des courants minimaliste, électronique, bruitiste et spectral -- peut révéler une énergie telle, provoquant chez l'auditeur une perception dynamique, encore une fois plus ou moins subtil ou prononcée et, selon la fréquence de focalisation de l'écoute. En d'autres termes il s'agit d'une attention dynamique avec ses expectations.

%Toutefois, il reste au compositeur de diriger l'ecoute sur le son lui-même, sur son devenir et/ou sur le contexte narratif de l´œuvre

Le statisme peut se décliner suivant différentes modalités, qui peuvent bien entendu se combiner. Il y a les musiques dites répétitives ou minimalistes, lesquelles consistent à répéter plus ou moins fidèlement une `pattern' usant de procédés canoniques (déphasage par décalage, augmentation ou diminution, permutation, rétro-gradiation, interpolation, ...). L'attention est portée sur la structure de surface (motifs résultants). \textit{La  forme, l'organisation interne et le processus ne font qu'un}. Selon la densité, l'on parle de texture ou de masse -- qui peut être de nature stochastique plus ou moins homogène. Viennent ensuite les sons et les silences qui s'étirent, sorte d'effet de \textit{time-stretching} ou de musiques spectrales aux transitions longues et continues révélant la structure interne, intrinsèque du son. Voire aussi les effets de battements émergeant ou de surface, par modulation d'amplitude ou de fréquence. Rappelons que l'expérience du son se fait par le corps et l'esprit. Et enfin la succession de motifs -- telle que la \textit{moment-form} -- et qui s'inscrivent dans une organisation formelle macroscopique, et peu perceptible en tant que telle.

%minimaliste:

%musiques répétitives --> structure de surface -- pattern, boucle, dephasage par decalage, augmentation ou diminution, permutation, retrogradiation... interpolation (forme, organisation interne et processus ne font qu'un)

%les sons et les silences qui s'étirent (drone/bourdon, bruit) --> structure intrinsèque (animation interne parfois de nature psychoacoustique) -- expérience du son par le corps et l'esprit -- musique spectrale (dynamisme des transitions continu) -- canon (ad infinitum) -- MOTIF RESULTANT

%subtilite -- nuances infime quelquefois reduit a une intention

% succession de motifs mettant à jour une organisation formelle

%inferieur à 15Hz 

%battements "differentiel" (proximite de deux frequences -- il peut s'agir d'harmoniques

%vibrato --> battement de surface par modulation d'amplitude ou de frequence

%musique stochastique et de texture

La pulsation est un élément déterminant sur le statisme. Elle peut être régulière comme hypnotisante. Elle peut être `flottante', non-pulsé (voire polychronie ou distribution stochastique). Alors, le temps émerge du son. Le son en tant qu'énergie est \textit{nécessairement d’ordre spirituel}; et un chemin vers la transcendance se dessine sur les éléments statique de notre écoute.

%le "temps pulsé" d'une
%musique formelle et fonctionnelle fondée sur les valeurs, le "temps non-pulsé"
%pour une musique flottante, flottante et machinique, qui n'a plus que des vitesses
%ou des différences de dynamique. (boulez)

%Scelsi : le temps émerge du son et le son est énergie. Cette énergie est nécessai-
%rement d’ordre spirituel, le son a un pouvoir, il agit sur l’âme.
%• Reich : le temps indéfini de l’événement et le son comme jouissance, qui est
%une jouissance de l’instant dans l’instant, sans but déterminé.
%• Feldman : le temps d’avant que l’homme ne pense au temps, une abstraction
%de l’être où le son existe par son écoute
% Le statisme, par ses spécificités, nous pa-
%raît être un chemin vers une transcendance réalisée à travers le son. 

\subsection{Le phénomène émergent}
 
 L'émergence est une caractéristique -- entendu un saut qualitatif -- d'un système en tant qu'unité, induit par l'interaction de ses éléments le constituant et avec son environnement. Par définition, cette caractéristique ne peut être déduite (\textit{a priori}) à partir de ses constituants, soit parce que le système a atteint un degré de complexité ne permettant pas d'en prédire le comportement dans son ensemble autrement que par une évaluation empirique (\textit{ergo aposteriori}), soit parce que cette caractéristique est une propriété structurelle qui dépend de la façon dont le système est agencé. 
% Une des propriété connexe d'un tel système est l'auto-organisation ou l'ordre spontané.
 L'un étant corrélé à l'autre.

Cependant, certaines théories émettent l'idée d'une émergence dite `faible' ou `forte'. Dans le premier cas, la dynamique du tout est déterminée par l'ensemble de ses parties `actives'. Malgré cela, la prédiction du phénomène reste le plus souvent une modélisation aposteriori et théorique, et qui se heurte exponentiellement à des calculs combinatoires insurmontables. Enfin, dans ce cas, le phénomène émergent est le concept définissant un saut qualitatif. Celui-ci est épistémologique et épistémique, reposant aussi bien sur l'\textit{affect} que sur le \textit{percept}. Tandis que pour l'émergence `forte', le système est censé rompre le lien avec ses constituants en tant qu'entité autonome, opérant un saut qualitatif non-causal\endnote{Non-causal, autrement dit il n'y a pas de causalité \textit{down-top}, soit une action causale ascendante des parties sur le tout. La condition \textit{sine que non} de  la causalité dans le cas de l'émergence `forte' est de type \textit{top-down}, c'est à dire que le phénomène émergent a une action causale descendante ou rétroactive sur ses parties. Noter que ces considérations terminologiques restent spéculatives, pas très bien définis, et sujettes à controverses, notamment à cause de la transitivité des causes suffisantes. De plus, l'interaction avec l'environnement multiplie de façon aléatoire les acteurs influents sur le système considéré; expliquant pour l'essentiel pourquoi le tout est plus que la somme de ses parties.\\ \indent Pour mémoire, la transitivité des causes suffisantes consiste à dire que: si la situation N des éléments du système produit l'état C du phénomène émergent, et que cet état suffit à produire C', lequel induit un changement N' par rétroaction, cela implique que N suffit à produire N'.}, soit ontologique -- autrement dit émergent; l'exemple type est l'émergence de la conscience. L'émergence `forte' se réalise avec l'interaction de l'environnement (lequel semble être la clé pour comprendre les phénomènes émergents en général) sur le système (reposant sur des principes collectifs d'auto-organisation), induisant une complexité qui rompt tout effet causal -- bien que celui-ci existe peu ou prou.
%  Vaste sujet que l'émergence, dont la musique est le phénomène émergent du monde sonore.
%  
%def ---> Le phénomène émergent est une propriété, une caractéristique d'un système complex, 
%...
%
%La musique en tant que système complex, répond aux principes d'émergence. 
%
%phénomène d’emergence dans la superposition d’objet sonore
%
%fusion/fission

%interprétation/composition -- phénomène d'incarnation (empreinte d'un agent humain au sens écoutez, il y a quelques choses à entendre)
\bigskip

La phénoménologie de l’émergence en musique résulte de deux systèmes pour le moins complexes, tels que la sensibilité auditive, soit `l'encodage perceptif'\endnote{Dans les grandes lignes, `l'encodage' perceptif est le médiateur entre les ondes sonores en tant que variation dépressionnaire du milieu provoqué par une significative perturbation mécanique et le traitement de l'information encodée dans le cerveau. Ce médiateur est -- vous l'aurez deviné -- l'oreille. Celle-ci présente la particularité -- notamment dû au canal auditif externe -- d'être plus sensible autour de la fréquence de 3\,500 Hertz. Cette sensibilité décroit dans les graves et les aigus selon les courbes isosoniques (niveau sonore perçu en fonction de la fréquence) qui rendent compte de manière empirique de la sensibilité du système auditif humain, car cette sensibilité n'est pas linéaire. De plus, cette sensibilité varie d'un individu à l'autre, notamment en fonction de l'age et de l'exposition à risque endurée. Aussi, il existe une bande de fréquence, appelé bande critique, dans laquelle une fréquence peut perdre de son intensité jusqu'à devenir inaudible lorsqu'une fréquence voisine dépasse un certain seuil d'intensité -- communément appelé effet de masque -- (à noter que la largeur de bande augmente avec la fréquence `masqueuse' et avec son intensité), ou bien perçue au dessus de ce seuil en termes de battement ou de rugosité (ce qui constitue en soi un phénomène émergent) -- sachant que la rugosité concerne la modulation de fréquence ou d'amplitude avec un effet maximal aux alentours de 70 Hertz, et avec un effet de battements pour les fréquences basses.\\ \indent Ainsi, le signal est décomposé dans la cochlée pour y être converti en un signal électrique transmit dans le cerveau par le nerf cochléaire composé d'environ 30\,000 axones.\\ \indent À noter aussi, que la pression de l'air -- que notre appareil auditif `encode' en termes de variation -- est elle-même un phénomène émergent, car c'est bien les relations entre les molécules d'air, et non les molécules d'air elles-mêmes, qui déterminent ce phénomène.}, et la psychologie du sujet; ou autrement dit, le son comme porteur d'information potentiellement utile au regard de la cognition du sujet pour le traitement de cette information. 

Concernant les processus mis en œuvre par le cerveau pour traiter l'information auditive, ceux-ci sont déterminés par deux modes d'apprentissage que l'on peut distinguer par l'acquis et par l'inné -- ou respectivement supervisé et non-supervisé selon la terminologie des modèles informatiques en intelligence artificielle. Le premier est un apprentissage qui conditionne une téléologie sociale et culturelle, tandis que le second se fait de manière implicite, sans que l'on en ai conscience, selon des récurrences structurelles de notre expérience avec l'environnement tel qui nous ai rapporté par nos perceptions sensorielles, et selon notre architecture cognitive. Cela se traduit pour l'un, d'assimiler et de perpétuer une connaissance, une pratique; et pour l'autre, de s'intégrer et de s'adapter dans un environnement qui nous ai donné d'expérimenter selon nos propres capacités en tant qu'être humain\endnote{\textsl{Basically, the brain processes implied can be depicted and understood according to different levels of integration of the sonic object as functional units. First, the echoic memory discriminates events; it is the ‘axiomatisation’ process. Second, the short-term memory groups in patterns any kind of melody and rhythm; here we talk about categorisation. Then, the long-term memory does the same at a higher level as section or form;  that is a symbolisation process. These processes are implicit and do not need to reach or communicate with the consciousness to be effective. The implicit learning is self-referent and obviously interacts with the ‘explicit learning’ -- sharing the same memory -- and reciprocally, improving} \textit{de facto} \textsl{and over time the knowledge and the recognition of  the musical phenomenon.}}.

Il en résulte une certaine perception et compréhension du monde sonore, ou plutôt devrais-je dire vibratoire, dont la résonance, et par conséquent sa non-résonance, deviennent les critères d'appréciation, provoquant une réponse émotionnelle plus ou moins marquées, et plus ou moins consciente.

\bigskip

Le phénomène émergent en musique consiste alors à évaluer ou à expérimenter un processus génératif proposé par le compositeur. Ce processus est le concept. Il peut être tonal et prendre la forme d'une sonate. Il peut être sémantique (selon des éléments linguistiques) ou ambiant. Il peut être mathématique ou algorithmique et se déployer en temps de manière déterminée ou chaotique. Il peut être le résultat d'une fission ou d'une fusion d'objets sonores par la psycho-acoustique. Il peut être aussi le résultat d'un acte volontaire ou contrôlé, ou bien non-attendu, contingent. Il est harmonie, tel que chaque objet sonore constituant le tout, la forme, se doit d'intervenir `au bon moment' dans un contexte opportun, de telle sorte à obtenir une cohésion globale -- soit une propriété structurelle de ce tout -- qui dépasse la somme de ses objets constituants ou `sonèmes'. Il peut être bien sûr tout cela à la fois. Bref, la musique est, en tant que telle, un phénomène émergent.

Dans le cadre qui nous intéresse, je parlerai plus volontier, de prime abord et indépendamment du concept lui-même, de l'émergence de moments `magiques' partagés, d'harmonie\endnote{Au sens étymologique du terme, du grec \textit{harmozō}, qui signifie joindre, faire coïncider, adapter, emboiter; et qui se rapporte à la racine Indo-Européenne \textit{\degree are-} ou \textit{\degree re-} pour adapter, ajuster. Et plus largement, je voudrais souligner une des acceptions par analogie, telle qu'établir une relation digne et respectueuse.} et de complicité.
%Pour ce qu'il est de l'apprentissage implicite, 
%$bien que cela reste un sujet d'étude en psychoacoustique et en cognition, 

%---------
%
%Outre le fait que l'activité musicale sollicite les ressources du cerveau dans son entier, en corrélant notamment le son à l'image pour le plus évident, mais aussi à tous nos sens, ainsi que nos connaissances acquises (notamment abstraites et symboliques),  il existe un schéma fonctionnel et structurel qui nous permet de comprendre ces deux modes d'apprentissages. 
%
%...
%
%Au niveau de la perception, un phenomene emergent consiste à percevoir un son dans un contexte sonore...
%
% D'abord, il s'agit d'identifier et de circonscrire l'object sonore en tant qu'unité irréductible selon des indices acoustiques tels que la régularité spectrale, le destin commun, et la contextualisation. La régularité spectrale est l'empreinte auditive ou plus simplement le timbre de la voix, d'un instrument, ou d'un son connu ou à connaitre. Le plus souvent, l'attaque ou le mordant, c'est à dire la courte période permettant la mise en vibration de l'air par l'instrument jusqu'à cette régularité spectrale, et la brillance, caractérisée principalement par le centre de gravité spectral, participent à la discrimination et à la reconnaissance de la source et de ses caractéristiques, permettant des groupements distincts et le traitement de structures temporelles. Le destin commun ...
%
%%..., l'audition musicale est un processus d'apprentissage implicite, c'est à dire que l'acquisition de connaissance de notre environnement via nos sens
%
%Pour ma part, je distingue deux modalités émergentes; à savoir la \textsl{fission}, qui est la capacité à distinguer un son spécifique de l'ensemble, et la \textsl{fusion}, comme le son résultant d'un complexe sonore en termes de texture ou de variation périodique ou continue.
%
%\bigskip
%
%%en psychoacoustique il y a trois phénomène induit...
%%La compréhension et la prévisibilité du phénomène émergent requièrent quelques connaissances psychoacoustique telles que `l'encodage' perceptif. Celui-ci présente deux caractéristiques que sont le `filtrage' qui amplifie certaines fréquences et qui en réduit d'autres, et ...
%
%
%timbre
%
%hauteur musicale
%
%hauteur virtuelle ---> périodicité lié à la régularité spectrale 
%
%hauteur spectrale ---> brillance estimé par le centroîd
%
%ambiguité voire illusion auditives entre hauteur virtuelle et hauteur spectrale
%
%indice acoustique
%-- regularité spectrale (timbre de la voie ou de l'instrument ou d'un son connu)
%-- destin commun --> probable source commune
%mecanisme cognitifs interagissant
%contextualisation (effet de contexte -- semantique, symbolique, ... -- completion d'information partielle par inference bayesienne dans le principe), sollicitation multi-sensorielle notamment visuelle
%
%décryptage par la probabilité d'une séquence, autrement dit de ce qui peut ou va suivre plus indice visuel (mouvement de la bouche pour le chant ou la parole, instrumentiste ou le faiseur de son en action)
%
%---> apprentissage et expérience
%
%+ Effet McGurk 
%
%Du reste, de manière générale l’oreille n’aime pas la monotonie et se focalise sur tout ce qui est
%non-stationnaire : vibrato, effet choral, transitoires, variations du spectre dans le temps.
%
%la brillance, caractérisée principalement par le centre de gravité spectral
%-- le mordant, caractérisée par la durée de l’attaque : de l’ordre de 1 à 2 ms pour le piano, 50 ms
%pour le violon
%Brillance et mordant rendent compte de 85\% de l’information caractéristique des timbres.

\subsection{\textsl{The score}}

Aussi bien destinée à exprimer ou à exploiter une idée musicale pour le compositeur que de faire office d'aide-mémoire pour l'interprète, la partition -- \textsl{the score} -- permet aussi la transmission d'un savoir, tant dans l'élaboration scripturale du projet compositionnel, que dans l'interprétation que peut en faire le lecteur ou le musicien en tant qu'œuvre.

%Il convient de distinguer deux approches. L'une, spécifique aux compositeurs européens, s'inspire et s'exprime selon des recherches essentiellement littéraires, tandis que l'autre, concerne les compositeurs américains qui sont surtout animés par des recherches picturales.

Cependant, il arrive que ce processus intervienne, non pas apriori, mais à posteriori. Lorsqu'une œuvre non écrite se cristallise ou bien lorsqu'il est opportun de préciser une modalité `de faire' pour l'interprète, la partition au sens large du terme devient alors le médiateur privilégié, sorte de trace, d'héritage ou de reconnaissance selon les cas.

Bien sûr, pour satisfaire une certaine lisibilité, il y eut une volonté pragmatique de standardisation de la notation musicale -- encore en vigueur dans l'enseignement musicale --, notamment initié en Europe avec l'imprimerie et la typographie durant la renaissance jusqu'au romantisme. Cependant, essentiellement destinées aux musiques modales et tonales, seul quelques paramètres musicaux étaient considérés, tels que la hauteur, la durée, puis la dynamique, le tempo, le caractère, les accents, etc. évoluant sans cesse, élargissant un répertoire de signes convenus selon l'esthétique musicale du moment.  
%Trop orientée sur la tonalité et le tempérament égale, malgré les tentatives atonales et micro-tonales du sérialisme, cela ne pouvait satisfaire les besoins et les nécessités des compositeurs d'avant-garde. 
Seulement, une certaine disparité `notationnelle' concernant certain paramètre (les sonorités `exotiques' par exemple) tendait depuis lors (notamment à partir du début du  \textsc{xx}\ieme ~siècle) vers une  spécialisation par type, par compositeur, voire par œuvre.
D'une extrême à l'autre, il est difficile d'imaginer un universalisme applicable dans ce domaine. Et cela est d'autant plus vrai pour la musique conceptuelle, électroacoustique et algorithmique, où il est tout simplement impossible d'utiliser la notation traditionnelle. Il est tout autant difficile d'envisager une notation plus ou moins ésotérique ou pragmatique convenue, qui répondrait à toutes les situations musicales.% tant esthétique, historique, sociale ou politique sans se plonger dans une herméneutique pluridisciplinaire, nous dépassant la plupart du temps.
%Étroitement lié à une esthétique musicale 
%Lorsque la `magie' opère lors de la reception de l'œuvre, il est naturel de vouloir la reproduire, pour nous mêmes, pour nos contemporains, voire pour les générations futures. La solution la plus évidente qui nous vient à l'esprit est bien entendu l'enregistrement. Quoique ce dernier requiert une logistique matériel pour le moins significative, avec une pérennité du support pour le moins discutable. Cependant, cela reste un instantané, qui dans le meilleur des cas, devient l'œuvre, tandis qu'il reste un ersatz de l'expérience vécue dans les autres cas. La partition s'avère alors nécessaire, voire indispensable, pour reproduire un effet voulue. Cet effet est le concept.

%Bien que le partage et la transmission peuvent se faire de manière orale, induisant une proximité en temps et lieu des acteurs et une inévitable évolution, transformation, adaptation, la partition ouvre des perspectives conceptuelles et d'élaboration, insoupçonnées autrement. D'ailleurs, en tant que concept, la partition est bien souvent l'œuvre.  

%Bien sûr, il y a l'enregistrement. Quoique ce dernier requiert une logistique matériel pour le moins significative, avec une pérennité du support pour le moins discutable. Cependant, cela reste un instantané, qui dans le meilleur des cas, devient l'œuvre, tandis qu'il reste un ersatz de l'expérience vécue dans les autres cas. La partition s'avère alors nécessaire, voire indispensable, pour reproduire la musique en question ou un effet voulue. 

Cela dit, la partition reste le médiateur entre l'intention musicale du compositeur et l'interprète (le singulier n'empêche aucunement le pluriel, autant pour l'acte compositionnel et créatif que pour le nombre de musiciens). À ce titre, cela permet des combinaisons improbable autrement (notamment pour les formation de type orchestre, ou dans certaines situations où un repérage synchrone est indispensable ou exigé; en règle générale, aussi bien `en temps' que `hors temps', la coordination requiert pour le moins une directivité; celle-ci est plus ou moins ouverte en degré de liberté selon l'œuvre), et ouvre des perspectives conceptuelles et d'élaboration, insoupçonnées et inatteignables autrement. D'ailleurs, en tant que concept, la partition est bien souvent l'œuvre.  

\bigskip

Maintenant et cela étant dit, il reste quelques pistes de notations utiles selon les contextes que les compositeurs et interprètes peuvent prendre en considérations telles que:

\begin{itemize}
\item la notation solfégique `traditionnelle' (faite essentiellement de portées, de notes, de durées, de mesures, ...), voire la tablature;
\item les pages volantes réorganisables  (forme mobile, ...), voire superposables;
\item la représentation textuelle (le plus souvent lié à une action, à un résultat sonore attendu, ...), voire du code pour les musiques algorithmiques;
\item la représentation graphique (symbolique, métaphorique, ...), voire shématique;
\item la représentation multidimentionnelle (2D, 3D, ...).
\end{itemize}

Loin d'être exhaustif, les procédés de cette liste sont bien entendu non exclusif et peuvent se combiner à loisirs, ce qui est d'ailleurs souvent le cas.

%Concernant la représentation textuelle, celle-ci pourrait n'être qu'un enregistrement 

\bigskip

À ce stade, deux autres modalités sont possible, mais non scripturales. Il y a naturellement la tradition orale, vivante par définition.  L'œuvre évolue, se transforme, au gré des interprétations plus ou moins fidèles et créatives, dont le concept même ou l'idée initial peut se perdre ou se dénaturer. Cela dit, la plupart des musiques dites traditionnelles ont été retranscrites, donc figées -- pour le meilleur et pour le pire. 

Il reste évidemment l'enregistrement\endnote{L'enregistrement va créer un changement de paradigme plutôt radical par la déterritorialisation des musiciens, transformant les usages sociaux, en individualisant l'écoute, et paradoxalement va uniformiser, normaliser les musiques populaires -- qui d'ailleurs s'auto-alimentent par la publicité et par conformisme -- par des diffusions intrusives et omniprésentes. Dans le premier cas, l'auditeur devient le musicien, en sélectionnant la musique qu'il veut entendre, et en appuyant sur le bouton \textsl{play} (réduisant l'effort à sa plus simple expression), ou bien sur un mode de choix multiples faisant intervenir parfois le hasard en recherchant une station de radio par exemple. \textit{In fine}, cette pratique deviendra un art à part entière, éprouvée par les DJs.
Dans le second cas, l'auditeur contraint et bien souvent passif, devient la cible et la victime d'une ingénierie sociale, dans une perspective de contrôle et de conditionnement des populations.}. Quoique ce dernier requiert une logistique matériel pour le moins significative, avec une pérennité du support pour le moins discutable. Cependant, cela reste un instantané, qui dans le meilleur des cas, devient l'œuvre, tandis qu'il reste un ersatz de l'expérience vécue dans les autres cas.%-- voire un pure produit de consommation (comme n'importe quelle autre œuvre d'art, d'ailleurs).

Cependant, rien n'empêche d'inclure ces modalités comme directive(s) par l'exemple, par ce qui doit être fait, ou sur ce qui est possible de faire.

%-- enregistement
%
% La solution la plus évidente qui nous vient à l'esprit est bien entendu l'enregistrement. Quoique ce dernier requiert une logistique matériel pour le moins significative, avec une pérennité du support pour le moins discutable. Cependant, cela reste un instantané, qui dans le meilleur des cas, devient l'œuvre, tandis qu'il reste un ersatz de l'expérience vécue dans les autres cas.
%
%\bigskip
%--  developper l'idée tradition orale, dont le concept même ou l'idée initial peut se perdre ou se dénaturer
%
%Bien que le partage et la transmission peuvent se faire de manière orale, induisant une proximité en temps et lieu des acteurs et une inévitable évolution, transformation, adaptation, la partition ouvre des perspectives conceptuelles et d'élaboration, insoupçonnées autrement. D'ailleurs, en tant que concept, la partition est bien souvent l'œuvre.  
%
%\bigskip
%
%-- developper l'idée de `en temps' et `hors temps'
%
%\bigskip
%
%--  developper l'idée d'une œuvre complexe dont la partition permet de coordonnée les musiciens entre eux ...

\bigskip

%-- petit historique et overview
%
%Aussi, cette période se caractérise par la multiplication des systèmes d'écritures qui peut se résumer par:
%\begin{itemize}
%\item 1914 – Sports et divertissements – Erik Satie → ajout de texte;
%\item 1961 – Available forms I – Earle Brown → forme mobile avec un paramétrage fixe;
%\item 1961/62 – Volumina – György Ligeti → representation visuelle.
%\end{itemize}
%
%« Noter, ce n'est plus nécessairement indiquer une hauteur de son, un rythme ... noter, c'est aussi inventer une écriture. » op. cit. p. 133.
% 
%\bigskip
%pour revenir sur l'idée d'œuvre ouverte, qui permet à tout à chacun d'exprimer une idée musicale par le partition, 

%-- œuvre ouverte et synthese
%
%Le concept d'œuvre ouverte est spécifique aux compositeurs européens, qui sont inspirés par des recherches essentiellement littéraires (alors que les compositeurs américains sont surtout influencés par des recherches picturales).
%Il convient de distinguer deux approches. L'une, spécifique aux compositeurs européens, s'inspire et s'exprime selon des recherches essentiellement littéraires, tandis que l'autre, concerne les compositeurs américains qui sont surtout animés par des recherches picturales.

% Si la science et la philosophie ont la vérité comme objet visé, la grammatologie voit en revanche dans l’écriture un phénomène sans vérité. L’écriture est cela même qui déstabilise la notion de vérité depuis son origine.
%\bigskip
%
%la partition donne des indication plus ou moins precise selon le resultat escompte,il peut s'agir d'un acte precis, ou une directivite, ou les deux À la fois; une directivite vers un acte precis.
%Bien qu'il y eut une volonté de standardisation de la notation musicale  initié en Europe durant la renaissance jusqu'au romantisme, encore en vigueur dans l'enseignement musicale, la disparité `notationnelle' tend depuis lors (notamment à partir du début du  \textsc{xx}\ieme ~siècle) vers une  spécialisation par type, par compositeur, voire par œuvre.
%D'une extreme à l'autre, il est difficile d'imaginer un universalisme applicable dans ce domaine. Et cela est d'autant plus vrai pour la musique conceptuelle, électroacoustique et algorithmique. Il est tout autant difficile de défricher, de décrypter, et d'envisager une notation plus ou moins ésotérique ou pragmatique, qui répondrait à une situation historique, sociale ou politique sans se plonger dans une herméneutique pluridisciplinaire nous dépassant, voire inaccessible à l'échelle humaine. 

%L'idée consiste à trouver un compromis entre l'individu et la masse. Compromis largement compris et pratiquer par les sociétés tribal, pour lesquelles il serait bon de s'inspirer que d'interpreter. il s'agit moins de comprendre que d'établir une relation digne, respectueuse, et harmonieuse au sens étymologique du terme, c'est a dire du grec (armozo), qui veut dire joindre, faire coincider, adapter, emboiter. (Du point de vue philosophique, en particulier dans la Grèce antique, on peut considérer l'harmonie comme le fait pour tous les éléments d'un tout d'être à la place qui leur est destinée, de telle sorte que le tout est meilleur que la somme des parties. L'harmonie est ainsi une propriété structurelle de ce tout. wikipedia)

Un dernier point sur la notation. Quant celle-ci ne se limite plus à décrire la production d'un son ou un résultat musicale entendu ou attendu, mais une action ou une situation, voire un déclencheur d'une chaine d'actions ou de situations, alors la partition devient, non pas le contenu -- c'est à dire la musique, le résultat musical -- mais bien le contenant. C'est une autre approche, conceptuelle, proposant un environnement dans lequel une écologie de sons va pouvoir émerger.  
La musique ne se réduit pas à la perception du son dans un  environnement inerte, mais englobe un ensemble d'activités et de faits.
 
%Représentation symbolique d'une situation musicale, la partition est à la fois ...

% Cage vient de sa conception de la fonction du sonore : la production du son n'est pas le but de l'action. L'action contient la musique et se suffit à elle-même.
%
%De telles partitions proposent des instructions pour susciter une situation dans laquelle les actions consécutives doivent être perçues en tant que musique. Le résultat musical de cette action est auxiliaire par rapport à la partition.

%     Synchroniquement, quelques artistes conceptualisent la musique en termes d’idées formalisables dont la résultante musicale est imaginée ou interprétée hic et nunc, impliquant la propre expérience du sujet.

%Parmi ces artistes, il convient de citer :
%\begin{itemize}
%\item 1913 – L’arte dei Rumori – Luigi Russolo
%\item 1916 – Ready-made – Marcel Duchamp
%\item 1939 – Experimental Music – John Cage
%\end{itemize}
%
%« Dans l'œuvre de Cage, l'acte de notation ne se referme pas sur la constitution d'un objet musical, mais agit plutôt comme le déclencheur d'une chaîne d'actions et de réactions sonores dont le compositeur ne cherche pas nécessairement à prévoir toutes les conséquences. » Jean-Yves Bosseur, \textit{Du son au signe. Histoire de la notation musicale}, Paris, Éditions Alternatives, 2005, p. 121.

%++++++++++++++++++++++++++++++++++++++++++++++++++++++++++++ 
%++++++++++++++++++++++++++++++++++++++++++++++++++++++++++++ 
%++++++++++++++++++++++++++++++++++++++++++++++++++++++++++++ 
 
\section{\textsl{So what}}

\textit{L'univers est vibration.}

\bigskip

 Le changement de paradigme se fait dans la manière de vivre la musique\endnote{À l'encontre du paradigme civilisationnel et mondialiste qui consiste à la division et à la `compartimentation' du travail impliquant pour ce qui concerne la musique, le compositeur maître imposant sa volonté à l’interprète, et l’interprété esclave, soumis, qui veut tirer parti de la composition pour se mettre en valeur; hiérarchisation extrême avec le dieu compositeur, le chef d’orchestre meneur de troupe, et d’obéissant soldats musiciens pour un résultat, certes parfois spectaculaire, mais fanatisant, hégémonique, voire suprémaciste;  et de réduire l'acte artistique à une seule personne pour légitimer une hiérarchie omniprésente, dévalorisant toute activité subalterne, laquelle se réduit à des executants que l'on appelle professionnel pour verrouiller le système. Et enfin le public qui dicte ses volontés sans toutefois intervenir directement dans l’œuvre. 
 \textit{Ainsi, l'art devient une activité réservée à une minorité en possession d'un certain savoir et l'œuvre d'art se doit de représenter une valeur qui l'éloigne des activités `banales' de la vie quotidienne.}\\ \indent  Bien qu'il y eut quelques tentatives de démocratisation. Par exemple par l'acte compositionnel en objectivant les systèmes musicaux par le sérialisme, mais qui se transforma bien vite en un mouvement hégémonique d'avant-garde, et qui se heurta à ses propres limites conceptuelles et qui par conséquent ne perdura pas. C'est aussi ce que vécut Fluxus\footref{efn:fl}, dont la dérive politique quelque peu radicale, en vue d'un ambitieux changement de société par l'art -- ou devrais dire l'anti-art --, divisa les membres sur ces objectifs et les moyens pour les atteindre, initialement focalisés sur un changement d'esprit, d'inspiration dadaïsme.
 Inutile de préciser que cela ne concerne pas seulement la musique, mais pratiquement toutes les activités humaines, où toute pensée alternative remettant en cause le système en place est systématiquement refoulée, écartée, mise à l'index, parfois pervertie, corrompue, assimilée, voire ignorée, méprisée, par conformisme.}, non pas comme un métier comme un autre avec plus ou moins de passion, ou comme un don du ciel, non pas comme un produit de consommation ou une quelconque idolâtrie plus ou moins divertissante, mais faisant partie de notre essence même, au plus profond de nous, et de nous impliquer en conscience de ce qui devrait nous relier, non seulement aux autres, mais à l'univers tout entier.  En substance, il faut apprendre à désapprendre.  

% À developper ----> Beaucoup de société tribale l'ont compris et le pratiquent, ou devrais-je dire le pratiquaient,  

\subsection{Déconstruction}

La catégorisation -- \textit{i.e.} le choix paramétrique ou encore la taxonomie -- est un inévitable et nécessaire carcan -- qui se trouve être dans le même temps un écueil -- de la déconstruction. La focalisation sur un aspect de l'objet d'étude permet des modélisations, voire des archétypes, et de révéler des concepts, ce qui fait la spécificité de l'humain, quant à la compréhension de son environnement -- mais, je le redis -- de manière \textit{partielle et partiale}. Et dans le même temps, cette même focalisation décorrèle les élements entre eux, perdant de la sorte de l'information, par  une axiomatique épistémologique et épistémique. Il en résulte un fragile équilibre entre les tenants et les aboutissants de ce que l'on cherche à déconstruire et à reconstruire. 

Cependant, cette nécessité fait l'identité du groupe, son égrégore, par la confrontation, l'intégration, et la synthétisation de ces choix.  Chaque étape de déconstruction, de construction, ou de reconstruction, sont autant de niveaux de %pratique de compréhension, d'intégration, 
perfectionnement, d'amélioration, de compréhension, nécessaire au concept, pour atteindre un objectif délibéré ou heuristique. 
Dès lors, le choix peut s'étendre sur des aspects extra-auditifs, comme autant d'éléments porteurs d'une musique en puissance, constituant le matériau d'une composition.

En tout état de cause, il s'agit bien d'impliquer chaque individu, dans un esprit bienveillant et avec le soutien de tous, afin de créer un réel dynamisme au sein du groupe et
une réalisation de soi par l’expérience aussi évidente et naturelle que spontanée. Cela doit ou devrait aller sans dire.

%Puisqu'il s'agit bien de cela, en musique, les différentes étapes de déconstruction et de construction, sont autant de niveaux de pratique d'intégration nécessaire au concept pour atteindre un objectif délibéré ou heuristique. 

%Cependant, cette nécessité fait l'identité du groupe, son égrégore, par la confrontation, l'intégration, et la synthétisation de ces choix. Cela doit ou devrait se faire avec le concours et le soutien de l’ensemble des personnes impliquées afin de créer un dynamisme naturel et une réalisation de soi par l’expérience aussi évidente que spontanée.

%Un des concepts le plus important consiste désormais à construire ou à reconstruire la musique que l'on souhaite faire en intégrant les différentes étapes comme autant de niveaux de pratique nécessaire à la réalisation de celle-ci. Les avantages sont que chaque personne quelque soit le niveau peut intégrer l'œuvre et évoluer avec elle de façon à atteindre l'objectif `virtuose' ou musical ou tout simplement expérimental, avec le soutien et le concours de l'ensemble. Cela créé un dynamisme naturel et une réalisation de soi aussi évidente que spontanée.

\subsection{Culture} 

Je voudrais revenir sur ces quelques notions, pour le moins importantes pour ne pas dire fondamentales pour ce qui nous intéresse, que sont le concept et la culture afin d'y apporter un nouvel éclairage, sur une même vision, en ces termes\endnote{Vue par Clifford Geertz, anthropologue de son état.}:
%vue par Clifford Geertz, anthropologue de son état. En effet, selon ses propres mots: 

 « \textit{... l’homme est un animal pris dans les réseaux de signifiance qu’il a lui-même tissés, je considère la culture comme assimilable à une toile d’araignée, et par suite son analyse comme relevant non d’une science expérimentale en quête de loi mais d’une science interprétative en quête de sens.} »

Aussi, la culture est « \textit{un modèle de significations incarnées dans des symboles qui sont transmis à travers l'histoire, un système de conceptions héritées qui s'expriment symboliquement, et au moyen desquelles les hommes communiquent, perpétuent et développent leur connaissance de la vie et leurs attitudes devant elle} ». Quant le symbole représente « \textit{tout objet, acte, événement, propriété ou relation qui sert de véhicule à un concept} ».
% SOURCE L’anthropologie interprétative Clifford Geertz (1973) Bali, interprétation d’une culture, Gallimard. https://dun.unistra.fr/ipm/uoh/anthropologie/fichiers/doc_geertz_int_01_01_fr.pdf

%\subsection{Esthétique} 
%
%...
%---------------------

\bigskip

De la culture à la musique, il n'y a qu'un pas. De la musique à l'art, et de son esthétique, il n'y a qu'une question philosophique à laquelle les philosophes  ont tenté et tentent encore de répondre selon leur \textit{épistémè}.
%que pléthore de traités. tous approchant le sujet de façon différentes sans qu'il y ait . 

La culture conditionne l'esthétique -- au sens de ce qui est beau -- d'une communauté de personne partageant des expériences de vie similaires. 
À l'instar de la musique, le beau est une construction de l'esprit; c'est à dire que la beauté n'est pas inhérente à l'objet, mais à la façon dont nous nous représentons l'objet. Le beau est une sensation d'harmonie, de satisfaction, de complétude, d'attractivité, d'amplification sensorielle, satisfaisant les sens et stimulant l'affect. C'est une affaire de goût.

Le plus souvent compris comme une normalisation, c'est un jugement de valeur qui se voudrait objectif, mais qui en réalité repose sur une culture commune \textit{hic et nunc}, d'un consensus institutionnel ou académique. Bref, la subjectivité d'un groupe admise comme objective.

 En fait, il s'agit d'un sentiment propre à chacun, qui peut être choisi ou subit, suivant ou à l'encontre de sa propre culture. Bref, l'on peut partager des éléments dit esthétique sans que cela soit une réalité objective, et lorsque l'on a la possibilité d'évaluer notre propre esthétique, leur diversité, leur disparité,  qui en font la richesse, témoigne d'une impossibilité consensuel vers une quelconque objectivation. Et pour cause, l'esthétique est l'un de ces concepts insaisissables, un cluster polysémique\endnote{La polysémie dans ce cas n'est pas exclusive, ni opposé, bien qu'elle puisse l'être dans certaine circonstance et selon leur position dans le cluster, et d'une manière générale selon l'étendu de ce même cluster.} nodal fluctuant.

%Ce qui ressort de cette aperçu historique est l'incroyable diversité de concevoir, de composer, d'écouter et de vivre le phénomène sonore que l'on appelle dans certaines circonstances musique. Évidemment, loin d'être exhaustif, ce résumé a pour but de donner des pistes d'investigations afin de découvrir, de redécouvrir ou d'approfondir certaine démarche et d'avoir une connaissance ou un savoir commun, si nous voulons ensemble construire ou vivre notre propre relation avec le monde vibratoire.

\subsection{\textit{Praxis}} 

Maintenant, il est possible d'appréhender la musique, dans une continuité dite de tradition -- perpétuation d'une codification --, ou bien dans une logique de rupture -- recherche de nouveaux paradigmes. La rupture se comprend selon la catastrophe thomienne\footref{efn:thom}, dans une sorte de saut qualitatif, d’émergence ontologique de la (ou d’une) structure sous-jacente. 

Cette rupture a été recherchée et souhaitée de par le passé, avec les futuristes, les dadaïstes, et leurs émules créant un véritable engouement dans les années 60/70. Mais toutes ces approches — comme autant de tentatives — ont été limitées, rattrapées, absorbées, digérées par les institutions\endnote{Toutes confrontations avec un système gérant des masses et non des individus restent le plus souvent vaines ou stériles. Bien que les société de masse peuvent dans certains cas privilégié l'individualité, mais seulement dans le but non dissimulé de diviser pour mieux régner, et en aucun cas de permettre ou de promouvoir l'émergence de mouvements alternatifs dans un but d'émancipation, ou la réalisation de soi.}, par la `réalité économique’ (pour ne pas dire marchande ou politique) empêchant toutes perspectives au delà de l’expérimentation elle-même. Ces modes de pratique musicale n’ont pas pu transcender nos habitudes culturelles, lesquelles sont ancrées en nous, tel un atavisme systémique de soumission pour lequel il est difficile, dans un premier temps d'en avoir conscience, et dans un second temps, de s'en émanciper.
Dans le meilleur des cas, quelques traces de ces pratiques subsistent, cantonnées au cadre pédagogique d’éveil ou de sensibilisation musical, en marge de l'enseignement traditionnel (basé sur le solfège tonal), ou bien être anecdotiquement mentionnées sur tout au plus une page de l’histoire de la musique.

De toutes façons, cela n’a pas fait consensus et bien que pratiqué dans certaines classes de musique au collège ou dans les écoles de musique, cela ne concerne que peu de personnes, car souvent dépendant du bon vouloir et de la compétence du pédagogue. Seul, quelques interprètes initiés `reproduisent’, perpétuent  ces concepts ponctuellement, trop rarement, pour un public aussi initié — et par conséquent peu nombreux -- ayant la chance et la possibilité d’y assister, voire de participer.

Tout ceci constitue une entrave certaine, quoique non définitive. Il nous appartient de s’inspirer, de réitérer, de reprendre à notre propre compte ces tentatives, car elles révèlent notre propre intelligence face à la connaissance acquise, par une constructive déconstruction, et notre capacité à modéliser une pratique issue de ces expériences.

\bigskip

\textit{L’obligation de produire aliène la passion de créer. 
%Le travail productif relève des procédés de maintien de l’ordre.
}%Raoul Vaneigem
%\textsl{We have to break the diptych consumer/product and be the actor of the `re-ritualisation/re-territorialisation' of `our' music as a didactic purpose and as an initiatory experience of life.} 
Nous devons avant tout rompre le diptyque consommateur/produit et être l'acteur de la `re-ritualisation/ re-territorialisation' de `notre' musique à des fins didactiques, ou autrement dit à l'initiation d'une expérience de vie.
%
L’idée est de donner l’opportunité à certain d’entre nous de se reconstruire ou de s’adapter autour d’un projet communautaire, (1) pour échapper notamment à la propagande et au conditionnement étatique et mondialiste, (2) afin d’initier un mouvement plus global capable de faire face à l’inéluctable effondrement du système marchand. Créer est un acte de résistance. Le \textit{pousse-toi-d'là-que-j'm'y-mette} doit faire place à l'engagement de partager sa vérité et de s'ouvrir sur celle des autres en vue d'un accomplissement en tant qu'être humain, induisant un soutien mutuel. En ce sens, la musique est le médiateur idéal, et devient le moyen de donner un sens à l’existence et d'ouvrir des horizons insoupçonnés. 

%Aussi, l'obligation de repetitions est une imposture au sens ou

\bigskip

Aussi, souvent, l’intimidation provoqué par notre entourage, voire par l’instrument lui-même, nous mène à l’abandon de tout projet en ce sens, quant bien même nous aurions voulu aller plus loin. Souvent lié un manque de confiance en soi, en ses capacités, souvent insoupçonnées, il nous appartient de reconquérir cet espace de liberté. Cet espace de liberté que l’on nous a privé, dérobé, dépossédé, comme autant de signes de soumission pour lesquels vous êtes prié de ne pas aller au delà de ce que l’on vous demande. La pression est grande. Il ne s’agit pas de piétiner les plates-bandes des `professionnels’\footref{efn:bnw}, qui ont besoin de votre ignorance et de votre bourse pour exister. Il nous faut simplement s’en détacher, faire abstraction de tout cela, et de s’accomplir, d’être soi-même. %[ and just do it, just be yourself ]
Certes on peut acquérir cette confiance par un `forcing’ requérant répétitions rébarbatives et entrainement plus ou moins intensif, mais en cela on force la nature -- à moins qu’il ne s’agisse de la nature même de la personne en question, et que l'on considère la vie comme une lutte permanente, une conquête du plus fort, du plus performant, une compétition favorisant les favorisés, véritable \textit{credo} de notre civilisation, et entériner par le darwinisme primaire\endnote{Au sens `doxique’ du terme qui consiste à généraliser une idée formulée par Charles Darwin en 1859 (\textit{On the Origin of Species}), dont l'idée première relate d'une sélection naturelle et graduelle induite par une concurrence dite vitale dans le processus de l'évolution des espèces. Ce qui peut en constituer une possible cause -- quoique le terme d'évolution me parait plutôt excessif quant il s'agit seulement d'une adaptation -- bien qu'il existe des organismes qui se reproduisent mécaniquement sans dessein (\textit{i.e.} gène égoïste) ou qui ne possèdent pas de lien généalogique (\textit{i.e.} gène orphelin). Et évoluer vers quoi? Si l'on considère la complexité organique comme une évolution (vers l'émergence de la conscience par exemple), certes, l'espèce humaine pourrait en être l'aboutissement (provisoire) de l'évolution, ce qui implique un dessein téléologique, plus idéologique et anthropo-centré que scientifique et objectif, car  nous partageons cette complexité organique pour le moins avec la plupart des mammifères (dont certains sont manifestement dotés d'une réelle forme de conscience). De plus, il existe des théories dont la pertinence est avérées, telles que par exemple l'idée ou le concept de niche, qui induit la notion d'harmonie (et non de concurrence) avec l'environnement selon les limites et les capacités du sujet en tant qu'entité biologique, ce qui semble plus `conforme' à la réalité, notamment archéologique (\textit{i.e.} la discontinuité du registre fossile). En tout état de cause, ces différentes approches, qu'elles soient liées à l'évolution darwinienne ou aux principes de niches écologiques, ne sont en aucun cas exclusives.}. Alors qu'il suffit de se focaliser autour d’un but commun, et la confiance, notre perfectionnement, notre développement, tant en pratique que spirituel, ira de soi, portée par la dynamique bienveillante du groupe, ce qui peut aussi être vrai dans les relations inter-groupes.



\bigskip

En termes de \textit{praxis}, il s’agit de donner de la place à tout à chacun, un espace de liberté totale -- du moins y tendre -- et de réserver quelques directives que ce soient uniquement dans le but d’exercer cette liberté, tel le funambule sur un fil. Il ne s’agit pas de servir un compositeur ou un public, mais de servir une idée, un concept, avec lequel nous devons jouer librement et en experimenter le potentiel au delà du convenu, au gré de notre fantaisie.


\subsection{Composition} 

Ce qui ressort de tout cela est l'incroyable diversité de concevoir, de composer, d'écouter et de vivre le phénomène sonore que l'on appelle dans certaines circonstances musique.


Composer consiste à prendre part activement, par le corps et l'esprit, au rituel consistant à interagir avec l'autre en conscience. L'autre étant l'entité avec laquelle on souhaite interagir. Il ne s'agit pas de signifier mais de communiquer au sens premier du terme 
(\textit{i.e.} `rendre commun')
par l'objet pour ce qu'il est. 

Maintenant, les modalités expressives s'inscrivent dans une situation  d'espace et de temps propre à chacun, et qu'il faut mettre à l'épreuve d'autrui, sorte de quête d'une complicité implicite. Ces modalités sont souvent d'ordre culturel, comme la tonalité, mais aussi d'ordre cognitif, sur la façon dont nous interagissons avec la musique. Ce sont des révélateurs circonstanciels, spontanées ou délibérées, révélant l'identité de l'ensemble des personnes concernées. 

Certes, tout le monde n’est pas compositeur ou musicien tel que nous le concevons, cependant, la participation active, dans le processus même et dans l’interprétation de l’objet sonore doit se faire pour tous. 
Cela peut s'adresser à une communauté ou à un groupe d'amis ou tout simplement un groupe d'individu partageant la même envie d'expérimenter le monde sonore et de s'immerger dans l'action de composer,
en conscience, comme le fît  Monsieur Jourdan en découvrant qu'il faisait de la prose sans le savoir\endnote{Référence à la comédie-ballet de Molière \textit{Le Bourgeois gentilhomme} (1670). À l'acte II scène IV,  Monsieur Jourdan apprend lors d'un échange avec le maître de philosophie, que \textit{\hbox{« tout} ce qui n'est point prose est vers; et tout ce qui n'est point vers est \hbox{prose »}};  et Monsieur Jourdan de répondre plus loin: \textit{\hbox{« Par} ma foi ! il y a plus de quarante ans que je dis de la prose sans que j’en susse rien, et je vous suis le plus obligé du monde de m’avoir appris \hbox{cela. »}}}.

Pour ma part, la musique consiste à explorer d'autres mondes, plutôt que de rivaliser ou de s'inspirer de modèle plus ou moins désuet ou ne correspondant pas à mes propres vibrations. Cela dit, l'un n'empêchant pas l'autre, je voudrais toutefois distinguer deux approches que sont la musique expérimentale et la musique de recherche. La première consiste principalement à expérimenter des appareillages, des procédés, des méthodes, tandis que la seconde se focalise plus particulièrement sur une reflexion au sujet de la musique en tant que telle. L'une recherche un résultat inouï, l'autre recherche les liens cognitifs. Elles s'auto-alimentent. \textit{Idée à partir de musique, musique à partir des idées}.

Cela reste dans tout les cas une démarche purement philosophique pour laquelle il convient de rester humble. Penser la musique autrement est prospecter au delà du pays fertile, en quête d’un savoir insaisissable avec lequel je vous invite à composer. 

%\subsection{Algorithmie} 
%On peut aussi l'appréhender respectivement comme acte d'expression ou de créativité, lesquels ne sont évidemment pas exclusifs.

%appropriation de l’idée dans le respect avec la source (ponctuelle tant la “véritable source” se perd par l’appropriation de l’idée sans le respect de la source — quoique l’idée peut se trouver dans “l’air” auquel cas il est possible de parler de source d’inspiration).

%on peut se laisser porter par la musique ...

%[ … ] dans un contexte sociétal qui se doit d'évoluer; pour lors, ce nouveau paradigme a pour vocation d'initier un mouvement culturel en tant que tel et social ethnologiquement -- voire éthologiquement -- parlant au regard du contexte historique.%\footref{efn:pr}. 

%The next endnote\endnote{This endnote gets a label.}\label{en:test} has  the number~\ref{en:test}. Let's now test endnotemark\endnotemark[\ref{en:test}].

%\bigskip
%Le but est alors de penser la musique comme concept en termes de formalisation et de modélisation. Ainsi le concept s’inscrira dans une praxis%\footref{efn:pr}  musicale délibérée (paysage sonore, synthèse sonore, ensemble musical, etc.).

%Autrement dit, cet atelier est un lieux de discussion philosophique, voire scientifique, sur le comment retranscrire une idée musicale. C'est alors un lieu d'échange de compétences expérientielles et techniques, tant dans la pratique compositionnelle qu'instrumentale dont le dessein et bien sûr de stimuler notre propre créativité. Il s'agit aussi de créer une dynamique de recherche en marge des institutions.

%Aussi, cet atelier s’intéresse aux travaux de « polyphonie spatiale » et plus particulièrement dans un espace quadriphonique usant de l'algorithmie.
%La syntaxe spatiale est ainsi partie prenante de la « mise en scène » du son.

%La composition ouverte est une approche conceptuelle qui se présente sous la forme d’une partition qui peut être graphique, textuelle et/ou traditionnelle en termes de directives formelles et/ou structurelles. La subjectivité et l’expérience de l’interpréte font de la partition un objet « sur mesure ». Autant dans le fond que dans la forme, ce type de composition s’inspire largement des travaux compositionnels de John Cage et de Jean-Yves Bosseur.

%l’œuvre ouverte pour ce qui concerne mon approche peut être vu comme un processus de cristallisation …
%La cristallisation peut se faire au niveau de la structure et/ou au niveau de l'object sonore lui-même.

%La pertinence de cette approche repose sur la prédictibilité ou la constance du ou des concept(s) émergent(s) s’inscrivant dans le contexte systémique de la partition. Autrement dit, les relations entres les propositions compositionnelles du compositeur, l’interpréte et l’intention artitisque sont intimement liés et délibérés.

\bigskip

\textit{Nous n’avons d’autre alternative que d’oser l’impossible.}

\section{…}
Cet essai aurait pu ou aurait dû être présenter sous la forme d'une thèse de doctorat en philosophie, en sociologie, en anthropologie, ou (ou devrais-je dire et) en musicologie, mais bon! Rien n'empêche de vous y coller. Je pense avoir dit l'essentiel\endnote{Seulement, il y aurait beaucoup à dire sur le conditionnement des masses, par une éducation orientée sur la hiérarchisation, voire la militarisation, des tâches sensées servir le peuple et la démocratie qui -- dans le meilleur des cas -- n'en a que le nom. Croissance, pouvoir d'achat, plein emploi, retraite, voilà où nous en sommes, et voilà où nous allons; une pensée unique pour une idéologie pour le moins douteuse quant à ses objectifs philanthropiques. Avec la meilleur volonté du monde, il est difficile de ne pas céder à la corruption, au bien matériel, et il est évident que les privilégiés -- c'est à dire les gens de pouvoir -- feront tout pour que cela ne change pas. Il est alors peu probable que ce travail soit soutenue par une institution quelconque -- mais pas impossible.}. Et le but reste moins la `bénédiction institutionnelle' ou une quelconque crédibilité sociale, sinon la recherche désespéré d'une reconnaissance plus ou moins vaine, au sein d'un impérieux mercantilisme le plus souvent servile,  que la possibilité d'entrevoir un monde autre. Il appartient à chacun d'expérimenter le sonore pour élever de ce qui nous reste d'âme, par l'expérience musicale, qui -- n'en déplaisent aux professionnels, spécialistes et autres experts\endnote{Lesquels ne déméritent aucunement, mais ils accaparent à eux seuls le peu d'espace dédié sur des critères marchand ou artistiques le plus souvent discutable. Et même si cela n'est pas le cas, ils sont à leurs dépens exploiter `jusqu'à la moelle' dans le but d'en extraire le \textit{je-ne-sais-quoi} qu'ils n'auraient simplement pas fait de leur propre chef, au détriment de ce qu'ils auraient pu accomplir librement. Est-ce un bien? Est-ce un mal? Décidément notre civilisation esclavagiste, favorisant les favorisés (les plus doués, les plus riches, les plus performants, les plus motivés, les plus ...), me désespère profondément sur ce point.\\ \indent  « \textit{Nous parvenons au portillon où un employé de la Senecefe récolte des billets afin de les périmer. En général on ne prend pour cette délicate opération que des spécialistes ayant subi un entraînement très intensif et des cours dans différentes universités. À l’université des perforations d’abord, avec comme culture physique complémentaire des séances de casse-noisettes (sur une musique de Tchaïkovski) ; à l’université des éphémérides ensuite où, pour vous assouplir le poignet, on commence par vous faire effeuiller des artichauts bretons. Périmer n’est pas à la portée de tout le monde, croyez bien. Il faut posséder une montre consciencieuse, avoir un calendrier à jour, savoir lire la date portée sur les billets pour se rendre compte s’ils sont du jour comme les œufs coque. Bref, c’est un métier !} »}, qui pour la plupart s'approprient le savoir pour le revendre comme ils l'ont acheté, pour le meilleur et/ou pour le pire -- se doit d'être accessible à tout à chacun\endnote{Vous l'aurez compris, l'accessibilité est une condition \textit{sine qua non}, qui se doit de l'être; et le cas échéant de conseiller, soit dans une attitude à adopter pour intégrer un groupe, soit dans le choix d'un groupe qui serait à même d'être intégrer, et ce, sans restriction d'aucune nature, pourvu qu'il existe ou doit se faire exister une structure pour.}, au même titre que la table de multiplication (pour ne pas dire les mathématiques), la conjugaison (pour ne pas dire la linguistique), et autres affaires, qui aident peu ou prou, à l'être pensant que nous sommes à s'accomplir en lui donnant la possibilité justement de penser par lui-même. On peut rêver.

%\bigskip
%
%Initialement, je voulais éviter toutes les références pompeuses,  
%
\newpage
  
\nolinenumbers
\setenotez{list-name = {Notes}}
\printendnotes

\newpage

\begin{titlepage} 
\clearpage
\thispagestyle{empty}
\setstretch{.8}
\pagecolor{yellow!10}

\vspace*{\baselineskip} 	
	
	\vspace*{0.2867\textheight} 
	
\centerline{\textsl{The truth is out there.}$^*$}
  \vfill
  {\scriptsize \noindent $^*$ La vérité est un concept définit par ce que l'on pourrait appeler -- et que j'appelle dans cet essai -- un \textit{cluster polysémique nodal fluctuant}, et est par conséquent loin d'être universelle et absolue; n'en déplaise aux gens de pouvoir pétris de certitudes, souvent au service de dogmes ou d'oligarques dont ils n'en comprennent que rarement les conséquences destructrices induites.\\ \indent J'aime particulièrement l'image que la Vérité serait comme un miroir brisé. Ainsi, chaque morceau reflète une partie de cette Vérité, et notre propre vérité en tant qu'individu. Si bien que pour accéder à cette `Vérité absolue', il faudrait reconstituer ce miroir brisé de la Vérité à la façon d'un puzzle avec l'ensemble des vérités avérées de chaque individu, si tant est que cela soit possible...
  %\\ \indent \textsl{The truth is} \textit{always} \textsl{out there.}
  }
  
 % {\footnotesize  $^*$ La vérité est un \textit{cluster polysémique nodal fluctuant}, et est loin par conséquent d´être universelle et absolue; n'en déplaise aux gens de pouvoir pétrient de certitudes, souvent au service de dogmes dont ils n'en comprennent que rarement les conséquences destructrices induites. }
%\vspace{-0.2cm} 
%\centerline{  {\footnotesize  ... }}

\end{titlepage}

\end{document}