music is about predictitbilty and satisfaction about it like the prediction is satisfied or the alternative make a surprise allowing emotion and so on...

music 
noise fiting a field of possibilities
that means how where why the bounderies of your field... 



Paradigme civilisationnelle qui consiste en la division du travail
impliquant le compositeur maitre imposant sa volonte a l’interprete
et l’interprété esclave qui veut tirer parti de la composition pour se mettre en valeur 
et enfin le public qui dicte ses volonté sans intervenir directement dans l’œuvre


souligné l’aspect philosophique et social de l’egemonie tonale (hierarchisation extreme avec le dieu compositeur, le meneur de troupe pour le chef d’orchestre et d’obéissant soldat pour les musicien pour un résultat fanatisant), puis sa tentative de démocratisation du système musicale avec l’atonalisme qui se transforme en suprématie avant gardiste, pour être redéfinit comme objet sonore “à vivre” ….


appropriation de l’idée dans le respect avec la source (ponctuelle tant la “véritable source” se perd par l’appropriation de l’idée sans le respect de la source — quoique l’idée peut se trouver dans “l’air” auquel cas il est possible de parler de source d’inspiration).


L’obligation de produire aliène la passion de créer. 

Face à cette médiocrité et ce conservatisme, Francesco Balilla Pratella déploie le « drapeau rouge du futurisme, et appelle auprès de ce symbole flamboyant les jeunes compositeurs qui ont à cœur d'aimer et de combattre, des esprits pour concevoir, et le front dégagé de couardise ».
Voici son programme1 :
* les jeunes doivent se tenir loin des conservatoires et étudier indépendamment ;
* il faut fonder une revue musicale indépendante des académies et des critiques ;
* on doit s'abstenir de participer à toute compétition qui ne serait pas complètement ouverte ;
* il faut se libérer du passé et de la musique « bien faite » ;
* la domination des chanteurs doit cesser, ils doivent devenir comme n'importe quel membre de l'orchestre ;
* les compositeurs d'opéras doivent écrire leurs propres livrets, et ceux-ci doivent être en vers libres ;
* il faut abandonner les ballades et « les chansons napolitaines qui donnent la nausée » ;
* il faut aussi abandonner la musique sacrée qui n'a plus de sens au regard de la disparition de la foi ;
* et enfin préférer les nouvelles œuvres aux anciennes.


certes, tout le monde n’est pas compositeur ou musicien tel que nous le concevons, cependant, la participation active, dans le processus même et/ou dans l’interprétation de l’objet sonore doit être accessible à tous 






Cela dit, cette démarche était encadrée dans un context institutionnel permettant de promouvoir une pratique musicale \textit{hic}  et \textit{nunc} et une approche compositionnelle autre pour enfin créer des liens avec l'enseignement traditionnel. Bien que cela n'est pas fait consensus dans le temps et dans l'espace, Il en reste des traces de nos jours dans les ateliers dit d'éveil musical pratiqués dans certaines classes de musique au collège ou dans les écoles de musique. 




C’est autant de pistes qui nous appartient de suivre selon la situation, de nous en inspirer selon notre humeur, et d’explorer au delà des sentier battu ce que le monde sonore peut nous offrir pour notre plus grand plaisir. 

Une œuvre conçue sur ce principe est incontestablement dotée d'une certaine « ouverture » le lecteur sait que chaque phrase, chaque personnage, enveloppent des significations multiformes qu'il lui appartient de découvrir. Selon son état d'esprit, il choisira la clef qui lui semblera la meilleure et « utilisera » l'oeuvre dans un sens qui peut être différent de celui adopté au cours d'une précédente lecture. Or, ici encore, « ouverture » ne signifie pas « indétermination » de la communication, « infinies » possibilités de la forme, liberté d'interprétation. Le lecteur a simplement à sa disposition un éventail de possibilités soigneusement déterminées, et conditionnées de façon que la réaction interprétative n'échappe jamais au contrôle de l'auteur.


de l’œuvre ouverte, une possible et parfois souhaitable cristallisation peut émerger spontanément.

la musique se présente souvent comme un langage universelle, mais cette “évidence” ne l’est pas. Pourquoi l’art devrait exprimer quelque chose au lieu de n’exister que pour lui-même? Pourquoi recherche à communiquer une émotion, quant celle-ci peut se manifester à tout moment, et de manière impromptu, ou bien quant celle-ci n’est pas le but de l’œuvre, mais plutôt dans une forme de perception autre que celle du message de son auteur. 



toutes ces approches ont été limité par la `réalité économique’ empêchant toutes perspectives au delà de l’expérimentation groupusculaire, qui fut une véritable libération, qui suscita un véritable engouement dans les années 60/70, pour s’éteindre petit à petit, digéré par les institutions pour devenir une page de l’histoire de la musique, et au mieux dans un cadre pédagogique d’éveil musical ou de sensibilisation ne concernant que peu de personnes. Seul, quelque initié `reproduisent’ ces concepts pour quelques représentations pour un public ayant la chance d’y assister. 

réalisé dans un cadre institutionnelle pour les uns, et de recherche pour les autres, qui n’impliquait que des musiciens `volontaires’ et en un petit comité de quelques personnes. Difficile d’échapper à la `réalité économique’ imposant une perspective pédagogique
——

il s’agit aussi de contenter les musicien aguerri  en leur donnant un espace de liberté selon le role que chacun veut jouer dans la pratique…
de donner des directives dans le but d’exercer la liberté de chacun selon des ligne directrice comme support tel le funambule sur un fil 

chaque groupe élaborant ses principes musicaux et ses méthodes de travail de manière personnelle et relativement autonome.

L'art doit descendre dans la rue et les artistes descendre de leur piédestal. Ce n’est qu’à cette condition que chacun pourra être à la fois acteur et spectateur, choisir librement l’attitude qui lui convient suivant ses désirs et ses aspirations au lieu de se voir imposer de l’extérieur un rôle prédéterminé

des démystifications de l’œuvre, de l’art, de l’artiste, du culte et du maniement du pouvoir sous toutes ses formes »

Écouter devient une activité aussi expérimentale que celle de composer ou d'interpréter la musique. 

Par conséquent, elle doit se libérer de sa fonction de transmission d’un message, ne plus chercher à communiquer à tout prix une idée, une information
…
ne cherchant pas à exprimer quoi que ce soit, elle peut tout signifier.

La musique devient le moyen de donner un sens à l’existence et d'ouvrir des horizons insoupçonnés.





in situ

La musique ne se réduit pas à la perception du son dans un  environnement inerte, mais englobe un ensemble d'activités et de faits.

Dès lors, les artistes isolent divers aspects extra-auditifs en tant qu’ éléments susceptibles d’être porteurs d’une musique en puissance et de devenir le matériau  d'une performance. 


Cette composition suggère que la préparation d'un  spectacle est
lui-même un spectacle à part entière. N’importe quel débutant ou non professionnel peut accomplir une telle activité, qui ne nécessite aucune virtuosité particulière.

Ici, ce qui est montré, c’est une tâche des plus banale.

Si Fluxus s’est efforcé très consciemment de résister à la commercialisation, comme bon nombre de mouvements artistiques considérés comme radicaux, il n'a toutefois pas pu ou pas su échapper à la récupération par les musées et les galeries d'art. 

https://fr.quora.com/Quelle-est-la-diff%C3%A9rence-entre-le-mental-et-l%C3%A2me
Aucune les matérialistes disent le mental, les spiritualistes disent l’âme.
Si l’on peut faire une petite métaphore sur ce sujet, on pourrait dire que notre corps est le véhicule, notre cerveau est l’ordinateur de bord de ce véhicule et l’âme ou le mental est le pilote. Quand le véhicule est détruit le pilote (âme ou mental) l’abandonne, pour les matérialistes le pilote (mental) disparaît avec le véhicule, pour les spiritualistes ou religieux le pilote (âme) continue son périple laissant son véhicule qui n’a plus dès lors d’utilités.

----------------------

egregore

...

----------------------

esthetique


le plus souvent compris comme la normalisation, un jugement de valeur qui se voudrait objectif, mais qui en realite depent d'une epoque en terme politique ou social, d'un consensus institutionnel ou academique, bref, la subjectivite d'un groupe admise comme objective.

En fait, il s'agit d'un sentiment d'auto satisfaction, de completude, de plenitude, propre a chacun, qui peut être choisi ou subit, suivant ou à l'encontre de sa propre culture, de son experience en ces temes. Bref, l'on peut partager des element dit esthetique sans que cela soit une realite objective, et lorsque l'on a la possibilite d'evaluer notre propre esthetique, leur diversite, leur disparite  qui en fait la richesse temoigne d'une impossibilite consensuel vers une quelconque objectivation. Et pour cause, l'esthetique est l'un de ces consepts insaisissable, un cluster polysémique nodal fluctuant,


 